%Results
\FloatBarrier
\section{Results}
This chapter is divided in three parts. The first part deals with the qualitative results. This includes response to high power ultrasound, intensity growth curves, and phase-shift bubble behaviour. The second part provides a quantitative validation of the counting algorithm. This validation is based upon the randomly synthesized data set. The last part presents the results from the processing and counting of the real data set. 
 
\subsection{Qualitative results}
This section provides qualitative results, following the prejudices stated in Section \ref{sec:qualitative}.
\subsubsection{Effect of high power ultrasound}

The counted number of bubbles before and after two high ultrasound bursts in non-linear and linear imaging mode were available for 8 and 13 animals, respectively. Relative change after two bursts is shown in Figure \ref{High power US non-lin} and \ref{High power US lin}.
% Note that the high power ultrasound burst are extended (\SI{\sim 2}{\second}) in non-linear mode, while they are extremely short in linear mode (less than \SI{0.1}{\second}). Single bubble intensity curves subject to two short bursts in linear imaging mode are shown in Figure \ref{Fig:high_power_US}. 
%Sett inn bilde av aktiverte boble fra flash
\begin{figure}
\centering
\begin{minipage}[t]{.5\textwidth}
\centering
\includegraphics[width=\textwidth,height=4cm]{relative change after high power US non linear.png}
\caption{Relative change in the counted density before and after two high power ultrasound bursts in non-linear imaging mode. Note that 0\% indicates no change.}
\label{High power US non-lin}
\end{minipage}\hfill
\begin{minipage}[t]{.5\textwidth}
\centering
\includegraphics[width=\textwidth,height=4cm]{relative change after high power US linear.png}
\caption{Relative change in counted density before and after two high power ultrasound bursts in linear imaging mode. Note that 0\% indicates no change.}
\label{High power US lin}
\end{minipage}
\end{figure}

\begin{figure}[h]
  \centering
  \includegraphics[width=\linewidth]{time_intensity_high_powerUS.png}
  \caption{Time intensity curves for a few bubbles subject to a short high power ultrasound burs in linear imaging mode. The two bursts are marked, and some bubble are immediately activated.}
  \label{Fig:high_power_US}
\end{figure}
 

\subsubsection{Visibility in non-linear and linear imaging modes} 
%\subsection{bubble zoom}

\subsubsection{Kinetics}
Bubble kinetics are shown in Figure \ref{Fig:bubble_kinetic}. A specific event occur in the location marked by the yellow circle 36 seconds into the corresponding video. A phase-shift bubble is released from its current location and relocated. 

\begin{figure}[h]
  \centering
  \includegraphics[width=\linewidth]{bubble_kinetics_scr.png}
  \cprotect\caption{Corresponding video shows a stuck phase-shift bubble release from current location, travel a short distance and fasten in another location. The area is marked with yellow and the specific event occur at 36 seconds. Corresponding video is located at \path{F:\usb\avi\2014-05-01-11-23-04_count_and_color_1_to_1000dilate_1_intensity_1000ct_0.85running_avg.avi}}
  \label{Fig:bubble_kinetic}
\end{figure}

The phase-shift bubble dynamics are also visible in Figure \ref{bubble_zoom} and \ref{bubble_zoom_tic}, where we see inflow and activation of a single phase-shift bubble.

\begin{figure}
	\centering
	\begin{minipage}[t]{.45\textwidth}
		\centering
		\includegraphics[width=.8\textwidth,height=4cm]{bubble_zoom.png}
		\cprotect\caption{Inflow and activation of a single phase-shift bubble is visible in the corresponding movie \path{F:\usb\avi\bubble_zoom_b4_10_28_44.avi}.}
		\label{bubble_zoom}
	\end{minipage}\hfill
	\begin{minipage}[t]{.45\textwidth}
		\centering
		\includegraphics[width=.8\textwidth,height=4cm]{bubble_zoom_tic_4_10_28_44.png}
		\caption{The corresponding time intensity curve for the phase-shift bubble to the left.}
		\label{bubble_zoom_tic}
	\end{minipage}
\end{figure}

\clearpage
\subsection{Counting of synthesized data}
A total of 81 videos were synthesized and processed. The result is found for a low (7) and high (258) number of phase shift bubbles in Figure \ref{synthesized low} and \ref{Fig:saturation}, respectively. We see that the program is counting too few bubbles (17) in the high number video (Figure \ref{Fig:saturation}) while the result is almost correct (5 of 7) in the low number video (\ref{Fig:synthesized low}).

\begin{figure}[h]
	\centering
	\includegraphics[width=\linewidth]{synthesized_low.png}
	\cprotect\caption{Counting of a synthesized data set The corresponding movie is found at  \path{F:\usb\avi\2014-05-02-09-51-16_PS_counted_7.avi}.}
	\label{Fig:synthesized low}
\end{figure}

\begin{figure}[h]
	\centering
	\includegraphics[width=\linewidth]{saturation.png}
	\cprotect\caption{Saturation is clearly seen in as large clusters of bubbles recognized as one bubble. The corresponding movie is found at  \path{F:\usb\avi\2014-05-01-11-44-15_PS_counted_258.avi}.}
	\label{Fig:saturation}
\end{figure}

The synthesized data is based on three different background videos. The three data sets and respective non-linear best fit are found in Figure \ref{close comparison}. The data is log-transformed and fitted to a third degree polynomial. The curve fitting is performed on the log-transformed data because of the multiplicative nature of speckle noise. A logarithmic transformation converts multiplication to addition, and makes the noise almost normally distributed (Figure \ref{Fig:log_normal}). The entire set, composed of all data from the three backgrounds, is presented in Figure \ref{Fig:counted_vs_inserted_all} and \ref{Fig:counted_vs_inserted_all_small}.

\begin{figure}[h]
  \centering
  \includegraphics[width=\linewidth]{close_comparison.png}
  \cprotect\caption{The three different data sets. The log-transformed data is fitted to a third degree polynomial. The corresponding background videos are found in the following directories. Background 1: \path{F:\usb\avi\2014-05-01-10-00-05_PS_counted_0.avi}. Background 2: \path{F:\usb\avi\2014-05-01-11-44-15_PS_counted_0.avi}. Background 3: \path{F:\usb\avi\2014-05-02-09-51-16_PS_counted_0.avi}.}
  \label{Fig:close comparison}
\end{figure}
%\begin{figure}[h]
%  \centering
%  \includegraphics[width=\linewidth]{2014-05-01-11-44-15counted_vs_inserted.png}
%  \caption{Non-linear regression on the first synthesized data set, file reference 2014-05-01-11-44-15.}
%  \label{Fig:counted_vs_inserted2}
%\end{figure}
%\begin{figure}[h]
%  \centering
%  \includegraphics[width=\linewidth]{2014-05-01-10-00-05counted_vs_inserted.png}
%  \caption{Non-linear regression on the first synthesized data set, file reference 2014-05-01-10-00-05.}
%  \label{Fig:counted_vs_inserted3}
%\end{figure}

\begin{figure}[h]
  \centering
  \includegraphics[width=\linewidth]{counted_vs_inserted.png}
  \caption{Plot and non-linear fit of the counted number density as a function of inserted density. Note that the fit is performed on the log-transformed data. This make the residuals of the fit almost normally distributed (\ref{Fig:log_normal}).}
  \label{Fig:counted_vs_inserted_all}
\end{figure}

\begin{figure}[h]
	\centering
	\includegraphics[width=\linewidth]{log_normal.png}
	\caption{The Normal probability plot for the residuals of the counted data and the fit in Figure \ref{Fig:counted_vs_inserted_all}. We see that the residuals are almost normally distributed. Remember that the fit is performed on log-transformed data.}
	\label{Fig:log_normal}
\end{figure}

\begin{figure}[h]
  \centering
  \includegraphics[width=\linewidth]{counted_vs_inserted_small.png}
  \caption{Plot and non-linear fit of the counted number density as a function of inserted density. Same plot as in Figure \ref{Fig:counted_vs_inserted_all}, but zoomed onto the range from 0 to 4 bubbles per square millimetre.}
  \label{Fig:counted_vs_inserted_all_small}
\end{figure}

If we swap the axes, we can plot the actual number density as a function of the counted number density. If we only consider the near-linear range, we can fit a linear curve to the log-transformed data (Figure \ref{Fig:counted_vs_inserted_inverse}). The spread of the counted number density has been evaluated (Figure \ref{Fig:rsd}). The average relative standard deviation is 0.016.

\begin{figure}[h]
  \centering
  \includegraphics[width=\linewidth]{counted_vs_inserted_small_inverse.png}
  \caption{Plot and linear fit of real number density as a function of counted density.}
  \label{Fig:counted_vs_inserted_inverse}
\end{figure}

\begin{figure}[h]
	\centering
	\includegraphics[width=\linewidth]{rsd_inserted.png}
	\caption{Relative standard deviation as a function of inserted number density of phase-shift bubbles. Linear fit to log-transformed data. The mean relative standard deviation is 0.016.}
	\label{Fig:rsd}
\end{figure}


\subsection{Counting of phase-shift bubbles in tumor}
Image of counting during ACT\texttrademark{} administration is found in Figure \ref{Fig:counting_administration}. The contrast agent is coloured in green. The blue boundary defines the region of interest, and counting is only performed within this area. Areas which are identified as stuck phase-shift bubbles are circumscribed with red.

Counted number density for the 16 animals are presented together with the manually counted results (Figure \ref{Fig:Number of counted bubbles} and \ref{Fig:Number density of counted bubbles}). There may be small differences between the ROI used for the automatic and manually counted data. This affects the number of counted bubbles, N. The bubble count representing each animal is the maximum count during the video of ACT\texttrademark{} injection. Results for all 125 video sequences are found in Appendix (\ref{raw counting}). Although the largest tumor holds the most phase-shift bubbles, it is no obvious relationship between the tumor area and the number of phase-shift bubbles (Figure \ref{Fig:area_vs_N}).

The average and standard deviation of the four different combinations of dose/activation (Section \ref{sec protocols}) is presented in Figure \ref{Fig:group avg}. From a two-way ANOVA analysis we find that there is a significant difference between the two doses given (p = 0.023), but no significant difference between the two transducers (p = 0.146).


\begin{figure}[h]
  \centering
  \includegraphics[width=\linewidth]{counting_scrshot_10_40_23.png}
  \cprotect\caption{Counting during administration of ACT\texttrademark{}. The corresponding video is found at \path{F:\usb\avi\2014-05-02-10-40-23_count_and_color_1_to_1000dilate_1_intensity_1000ct_0.85running_avg.avi}. All contrast agent is coloured in green. The blue boundary defines the region of interest, and counting is only performed within this area. Areas which are identified as stuck phase-shift bubbles are circumscribed with red.}
  \label{Fig:counting_administration}
\end{figure}



\begin{figure}[h]
  \centering
  \includegraphics[width=\linewidth]{manual_vs_auto_count.png}
  \caption{Number of counted bubbles in the 16 mice. Note that there may be differences in the ROI used for the manual(blue) and automatic(red) counting.}
  \label{Fig:Number of counted bubbles}
\end{figure}

\begin{figure}[h]
  \centering
  \includegraphics[width=\linewidth]{manual_vs_auto_count_density.png}
  \caption{Number density of counted, stuck phase-shift bubbles in the 16 mice. Automatic counting is shown in red, while manual counting is shown in blue.}
  \label{Fig:Number density of counted bubbles}
\end{figure}


\begin{figure}[h]
	\centering
	\includegraphics[width=\linewidth]{area_vs_N.png}
	\caption{Plot of number of counted phase-shift bubbles versus tumor area. The tumor area is the area inside the defined region of interest.}
	\label{Fig:area_vs_N}
\end{figure}

\begin{figure}[h]
	\centering
	\includegraphics[width=\linewidth]{gropu_avg.png}
	\caption{Average and standard deviation of number of counted phase-shift bubbles for the four different combinations of dose and transducer.}
	\label{Fig:group avg}
\end{figure}


For all 16 animals, time-intensity curves showing the integrated intensity within the ROI as a function of time is plotted. The count number density is also plotted with respect to time. For one animal, this is presented in Figure \ref{Fig:tic} and \ref{Fig:tic_count}. Similar results for all animals are found in Appendix \ref{tic appendix} and \ref{App:number_density_curves}.


\begin{figure}[h]
	\centering
	\includegraphics[width=\linewidth]{tic_10.jpg}
	\caption{Time-intensity curve in animal nr. 10. These numbers represent both non-linear(red) and linear(black) contrast imaging.}
	\label{Fig:tic}
\end{figure}

\begin{figure}[h]
	\centering
	\includegraphics[width=\linewidth]{bubble_count5.jpg}
	\caption{Number density as a function of time in animal nr. 5. These numbers represent both non-linear(red) and linear(black) contrast imaging.}
	\label{Fig:tic_count}
\end{figure}
\clearpage
%Table with andys count vs my count



