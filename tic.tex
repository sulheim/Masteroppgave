\section{Time-intensity curves}
\label{tic appendix}
Time-intensity curves for all 16 animals are presented below. Black and red data points represent data from linear contrast and non-linear imaging modes, respectively.
\begin{figure}
	\includegraphics[width=\linewidth]{tic_1.jpg}
	\caption{Time-intensity curve for animal number 1. Black and red data points represent data from linear contrast and non-linear imaging modes, respectively.}
\end{figure}
\begin{figure}
	\includegraphics[width=\linewidth]{tic_2.jpg}
	\caption{Time-intensity curve for animal number 2. Black and red data points represent data from linear contrast and non-linear imaging modes, respectively.}
\end{figure}
\begin{figure}
	\includegraphics[width=\linewidth]{tic_3.jpg}
	\caption{Time-intensity curve for animal number 3. Black and red data points represent data from linear contrast and non-linear imaging modes, respectively.}
\end{figure}
\begin{figure}
	\includegraphics[width=\linewidth]{tic_4.jpg}
	\caption{Time-intensity curve for animal number 4. Black and red data points represent data from linear contrast and non-linear imaging modes, respectively.}
\end{figure}
\begin{figure}
	\includegraphics[width=\linewidth]{tic_5.jpg}
	\caption{Time-intensity curve for animal number 5. Black and red data points represent data from linear contrast and non-linear imaging modes, respectively.}
\end{figure}
\begin{figure}
	\includegraphics[width=\linewidth]{tic_6.jpg}
	\caption{Time-intensity curve for animal number 6. Black and red data points represent data from linear contrast and non-linear imaging modes, respectively.}
\end{figure}
\begin{figure}
	\includegraphics[width=\linewidth]{tic_7.jpg}
	\caption{Time-intensity curve for animal number 7. Black and red data points represent data from linear contrast and non-linear imaging modes, respectively.}
\end{figure}
\begin{figure}
	\includegraphics[width=\linewidth]{tic_8.jpg}
	\caption{Time-intensity curve for animal number 8. Black and red data points represent data from linear contrast and non-linear imaging modes, respectively.}
\end{figure}\begin{figure}
	\includegraphics[width=\linewidth]{tic_9.jpg}
	\caption{Time-intensity curve for animal number 9. Black and red data points represent data from linear contrast and non-linear imaging modes, respectively.}
\end{figure}
\begin{figure}
	\includegraphics[width=\linewidth]{tic_10.jpg}
	\caption{Time-intensity curve for animal number 10. Black and red data points represent data from linear contrast and non-linear imaging modes, respectively.}
\end{figure}
\begin{figure}
	\includegraphics[width=\linewidth]{tic_11.jpg}
	\caption{Time-intensity curve for animal number 11. Black and red data points represent data from linear contrast and non-linear imaging modes, respectively.}
\end{figure}
\begin{figure}
	\includegraphics[width=\linewidth]{tic_12.jpg}
	\caption{Time-intensity curve for animal number 12. Black and red data points represent data from linear contrast and non-linear imaging modes, respectively.}
\end{figure}\begin{figure}
	\includegraphics[width=\linewidth]{tic_13.jpg}
	\caption{Time-intensity curve for animal number 13. Black and red data points represent data from linear contrast and non-linear imaging modes, respectively.}
\end{figure}
\begin{figure}
	\includegraphics[width=\linewidth]{tic_14.jpg}
	\caption{Time-intensity curve for animal number 14. Black and red data points represent data from linear contrast and non-linear imaging modes, respectively.}
\end{figure}
\begin{figure}
	\includegraphics[width=\linewidth]{tic_15.jpg}
	\caption{Time-intensity curve for animal number 15. Black and red data points represent data from linear contrast and non-linear imaging modes, respectively.}
\end{figure}
\begin{figure}
	\includegraphics[width=\linewidth]{tic_16.jpg}
	\caption{Time-intensity curve for animal number 16. Black and red data points represent data from linear contrast and non-linear imaging modes, respectively.}
\end{figure}
\clearpage