%Teori.tex
\section{Theory}
\subsection{A start}
The basis of ultrasound imaging is the reflection of ultrasound at tissue boundaries within the body. The ultrasound is generated by a transducer, and as the wave travels through the body an echo is generated by partial reflection at every boundary. The amount of reflection depends on the difference in acoustic impedance. The echo is recorded by the transducer, and it is then displayed in the image according its spatial origin. The speed of sound is approximately 1540 $m s^{-1}$ for all soft tissue in the body, and it is thus easy to calculate the origin by measuring the time of travel of the echo.  The brightness in the image is proportional to the strength of the echo, and this is thus called B-mode(brightness mode). 



\subsection{Contrast agents}


\subsection{Transducer}


\subsection{Image processing of B-mode images}
After the echoes have reached the transducer a signal is produced by making an image with the brightness at each pixel determined by the strength of the echo from that corresponding distance and direction. The first step in the image processing is to convert the signal from analogue to digital. The digital signal is less vulnerable to noise and distortion, and it enables further digital image processing. Then a linear amplifier apply the same amount of gain to the entire signal, to make the signal strong enough for further processing. Time-gain compensation is then applied to make echoes from similar interfaces equal, regardless of the depth of their origin. This is performed by increasing the gain with increasing depth of echo. The depth of the echo is identified by the arrival time at the transducer. The rate of attenuation of ultrasound with depth is determined by the frequency and tissue.

After amplification and time-gain compensation the dynamic range of the signal is about 60 dB. The dynamic range of a signal is defined as the ratio between the largest amplitude that can be recorded without causing distortion and the lowest amplitude that can be distinguished from noise. The dynamic range of a common screen is about 20 dB. The signal must therefore be compressed before it can be displayed. To compress the dynamic range from 60 to 20 dB, an amplifier with non-linear gain is applied. Low amplitudes are amplified more than high, and the dynamic range is therefore decreased. Compression allows weak echoes from scattering within tissue to be displayed together with strong echoes from tissue interfaces.

\subsubsection{RF and IQ data}
RF is short term for radio frequency data which is used in ultrasound as a description for unprocessed data. IQ is short term for in quadrature, and refers to a modulation of the RF data to reduce the amount of storage space without loosing information. IQ modulation converts the signal from the real to the imaginary space.

\subsubsection{Hilbert transform}
IQMODULATION?????
The hilbert transform is a linear operator which acts on a signal $u(t)$ to derive an analytic signal. The hilbert transform convert the signal from real to complex space by adding or subtracting 90 degrees. It is therefore also known as a phase-shift operator. An analytic signal has by definition only positive frequencies in its fourier transform, and is related to the hilbert transform through 

\begin{equation}
\tilde{x}(t) = x(t) + x_h(t),
\end{equation}

where $x(t)$ is the signal, $x_h(t)$ the hilbert transform of the signal, and $\tilde{x}(t)$ the analytic signal. The hilbert transform can be written as a convolution, 

\begin{equation}
x_h(t) = x(t)*\frac{1}{\pi t},
\end{equation}
which can be interpreted as a filtering operation with a quadrature filter which shifts all sinusoidal components by a phase shift of $\frac{\pi}{2}$.

\subsection{Harmonic Imaging}

\subsection{Matlab}
\subsubsection{Removing image artifacts}
The operation of removing movement artifacts are based on the Matlab toolbox Image Processing and the use of image registration. \textit{imregister} and \textit{imregconfig} are the first two functions that have been applied and tested. This is intensity based automatic registration. Control point registration may be another option.

Image registration in Matlab aligns one image to a fixed reference image by applying geometric transformations. In sonography displacement of images occur due to breathing and movement of the image target, and it is important to remove these movements to be able to do quantitative measurements on the images. Intensity based image registration find similar intensity patterns in the    the two images, and use this to map the misplaced image into correct position. 

The process is initiated by making a \textit{metric} and an \textit{optimizer} object using the \textit{imregconfig} function. The \textit{metric} object measures the similarity of the two images. The \textit{optimizer} contains the optimization parameters such as maximum number of iterations, initial step length, optimization algorithm etc. Which optimization algorithm and how the image similarity is measured can be chosen in the \textit{imregconfig} function. The options for optimization algorithm are either a regular step gradient or a one-plus-one evolutionary method. For the metric object the similarity can be measured either by a mean square error approach or by making a mutual information metric. The mutual information metric maximizes the number of pixel with the same relative pixel value, and is best suited for images with different brightness ranges.[REF MATLAB]










