% Først spesifiserer vi hvilken dokumentklasse vi vil ha og noen 
% globale opsjoner. Bytt ut 'article' med 'book' hvis du vil ha med kapitler.
\documentclass[b5paper, twoside, titlepage, 10pt]{article}

% Så sier vi fra om hvilke tilleggspakker vi trenger
% til dokumentet vårt. De som du ikke trenger (se kommentaren) 
% kan det være en fordel å kommentere ut (sett prose
%%%%REGEX FINN litenbokstav etter punktum : \.\s[:lower:]
\usepackage[T1]{fontenc}
\usepackage{cite}
\usepackage[english]{babel}
\usepackage[lmargin=25mm,rmargin=25mm,tmargin=27mm,bmargin=30mm]{geometry}                
\usepackage{amsmath,amsfonts,amssymb} % matematikksymboler
%\usepackage{amsthm}                   % for å lage teoremer og lignende.
\usepackage{graphicx}                 % inkludering av grafikk
\graphicspath{{figurer/}}
\usepackage{subcaption}                   % hvis du vil kunne ha flere
                                      % figurer inni en figur
\usepackage[space]{grffile}
\usepackage{siunitx}
\sisetup{range-units = single}
\usepackage{fancyvrb}
\sisetup{separate-uncertainty}%
\DeclareMathOperator{\sgn}{sgn}
\usepackage{morefloats}
\usepackage[numbers]{natbib}
\usepackage[titletoc,toc,title]{appendix}
%\usepackage{underscore}

\usepackage{longtable}
\usepackage{multirow}
\usepackage{mathtools}
\usepackage{units}%for nicefrac
\usepackage{textcomp}%For trademark
\usepackage{commath}%for derivatives
\usepackage{booktabs}
\usepackage{cprotect}
\usepackage{placeins}
\renewcommand{\arraystretch}{1.2} 
%
%\DeclarePairedDelimiter\abs{\lvert}{\rvert}%
%\DeclarePairedDelimiter\norm{\lVert}{\rVert}%

\usepackage{listings}                 % Fin for inkludering av kildekode
\usepackage[hyphens, obeyspaces]{url}
\Urlmuskip=0mu plus 1mu


%\usepackage{hyperref}                % Lager hyperlinker i evt. pdf-dokument
                                      % men har noen bugs, så den er kommentert
                                      % bort her.
                                 
% Indeksgenerering er kommentert ut her. Ta bort prosenttegnene
% hvis du vil ha en indeks:
%\usepackage{makeidx}     
%\makeindex              

%\includeonly{forside, introduction, teori, metode, results, Rayleigh-plesset}
% Selve dokumentet begynner:

\begin{document}

% På forsida skal vi ikke ha noen sidenummerering:

\pagestyle{empty}
\pagenumbering{roman}

% Inkluder forsida:
% Enkel forside som bruker latex sin \titlepage kommando:
% NB: Bruken av \and mellom navn!
\titlepage
\title{ Method development and automated analysis of ultrasound images of phase-shift bubbles.}
\author{Snorre Sulheim}
\date{\today}
\maketitle
%abstract
%Summary of Background/motivation for work
%Material and methods
%results
%Conclusion
%Approx 1 page.
%No references
\section{Abstract}
Ultrasound mediated drug delivery is an important tool in the fight against cancer. A new concept known  as ACT\texttrademark{} is under development, and two pilot imaging studies have been performed on prostate cancer xenografts in mice.  A large amount of raw ultrasound data has been recorded, but existing software can not perform the required image processing. The ACT\texttrademark{} concept is based on clusters of microbubbles and microdroplets that experience a phase-shift from liquid to gas when exposed to ultrasound. The phase-shift increases the bubble size and, make they are caught in the vasculature of the tumor. 

A complete program has been developed in Matlab\textsuperscript{\textregistered} to process the raw ultrasound data. The program is tailored to the unique properties of the phase-shift bubbles, and is able to reduce noise and motion artefacts, to visualize the contrast agent, and to count the number of ultrasound activated phase-shift bubbles. The program produces high quality videos, displaying both free flowing contrast agent and identified, stuck phase-shift bubbles. 

The program showed a very good correlation to a manually counted data set. The program was validated against a synthesized data set, and we found that the program counted accurately up to \SI{\sim2}{bubbles\per\milli\meter}. A saturation was experienced above this threshold, and too few bubbles were counted.  


\include{sammendrag}

% Romerske tall på alt før selve rapporten starter er pent.
\pagenumbering{roman}

% For å ikke begynne innholdslista på baksida av forsida:
\cleardoublepage
% (kun aktuelt når man har twoside som global opsjon)

% Nå vi vil ha noe i topp- og bunnteksten
\pagestyle{headings}

% Si til LaTeX at vi vil ha ei innholdsliste generert akkurat her:
\tableofcontents

% Pass på at neste side ikke begynner på baksida av en annen side.
\cleardoublepage

% Arabisk (vanlige tall) sidenummerering. Starter på side 1 igjen.
\pagenumbering{arabic}

% Inkluder alle de andre kildefilene:

% NB: Vi trenger ikke ta med filendelsen .tex her. Den vet
%     LaTeX om selv!
\section{Introduction}
%Skriv om annen forskning innen samme område. ref mail fra andy 30.okt. 
%Skriv noe om forskning ved NTNU
%short about background/motivation
%Migth mension methods used
%1-1 1/2 page

%Review of background information enabling the reader to understandt objective and significance
Cancer is currently the most common cause of disease, with more approximately 14.1 million new incidents worldwide every year (per 2012)\cite{cancer1}. In the United states, 1 of 4 deaths are caused by cancer\cite{Siegel2014}, and only 50\% of the people diagnosed with cancer survive 10 years or more. 

The available treatments are rarely satisfying, giving limited results and significant side effects. Chemotherapy is one of the major treatments, in addition to surgery and radiotherapy. Chemotherapy is based on administration of toxic drugs, and the limited effect is related to the lack of specificity. The toxic drug destroy healthy as well as cancerous tissue, limiting the dose to a survivable amount. This amount may be too low for the cancer to be fully treated[ref].

Targeted cancer treatment is a hot field of research, where the goal is a large uptake of drug in cancerous tissue without harming healthy tissue. Several targeted treatments are already available, but none have so far been able to produce a precise,  satisfying treatment. A common weakness is that the drug does not travel far enough into the tumor after leaving the vasculature. This leave remote cancerous cells untreated. Slow accumulation is another drawback. 

Ultrasound may be the solution for precise, effective drug delivery. Microbubbles are already used as contrast agents for ultrasound, and work has been conducted to use these as drug carriers as well. The main idea is to use ultrasound to trigger the release of the carried drug. The drug can be loaded onto the shell encapsulating the microbubble, but this research has given limited results due to low drug load capacity\cite{Ibsen2011}(obs. review article). Another concept is the use of nanoparticles to achieve the necessary properties. In \cite{Eggen2013} the drug is encapsulated in nanoparticles. The nanoparticles are then used to encapsulate the gas microbubbles. The shell of the microbubble can also be developed with ligands able to attach to receptors present in the tumor vasculature. This is known as active targeting and can increase accumulation of drug in the tumor vasculature.

Phoenix Solution are developing a new concept for ultrasound mediated drug delivery. This concept use clusters, consisting of encapsulated gas microbubbles and drug-loaded emulsion droplets. When exposed to ultrasound a phase-shift occurs, and the drug is released immediately. The emulsion droplets vaporizes, creating new gas bubbles of \SI{\approx 30}{\micro\meter}. Roughly ten times the initial size. These bubbles are large enough to block small vessels, keeping the drug present for an extended time. Low frequency ultrasound is applied, initiating oscillations of the large bubbles. Applied ultrasound together with microbubbles are known to enhance drug delivery through enhanced vessel permeability and sonoporation. This new drug delivery concept is called ACT\textregistered, and can be regarded a theragnostic product. It is possible to image the clusters and phase-shift bubbles while performing drug delivery.

The function of these microbubbles has to be proved \textit{in vivo}. This is carried through on mice with prostate cancer xenografts, where the administrationa and activation is imaged using high-frequency(\SIrange{16}{18}{\mega\hertz}) ultrasound. The main goal of this thesis is to process the recorded data to estimate the number of activated and stuck phase-shift bubbles within the tumor. Image processing involves motion correction and background subtraction. The algorithm performing counting of phase-shift bubbles should be completely validated, including estimate of accuracy and precision. The algorithm should be applied to a data set containing ultrasound images of 16 mice. A proper evaluation of the performance of the ACT\textregistered concept is essential for further development toward clinical trials. The phase-shift bubbles have previously been counted manually. 

Counting of stuck bubbles in ultrasound images is a field of image processing with limited available literature and publications. One approach is described in \cite{Needles2009}, where microbubbles bound to vessel walls are differentiated from tissue and flowing microbubbles. Subharmonic imaging is used to separate tissue and microbubbles, before a low-pass inter-frame filter is applied to remove the free flowing microbubbles. 






%What have been done?

%Summary of conflicting findings in literature

%What I want to do

%Purpose and significance of study




%\subsection{History}
%Ultrasound is sound with frequency above the upper limit of human hearing, considered to be at 20 kHz, and has a wide range of use. The first work on ultrasound related to spatial orientation was written in 1794 by Lazaro Spallanzini, after discovering bats ability to navigate, only using ultrasound. Yet, almost hundred years passed before Jacques and Pierre Curie discovered the piezoelectric effect and Sir Francis Galton invented a machine able to produce ultrasound at 40 kHz, both during 1880. The piezoelectric effect is the ability of some crystals to generate electrical charge, when subjected to mechanical stress.
%
%During the beginning of the 20th century the echo-locator was invented, and the first application was detecting submarines during World War 1. Use of ultrasound in medical imaging, also known as sonography, was first used in 1956 when Ian Donald measured the parietal diameter of a fetal head. Seven years later commercial sonography devices were available.
%
%The last decades there has been continuous development, and from the simple display modes used in the beginning, there is now possible to get real-time imaging in both two and three dimensions, and Doppler imaging enables continuous measurement and visualization of blood movement in vessels.




\subsection*{Outline of thesis}

 
   
%Teori.tex
\section{Theory}
\subsection{Basics of ultrasound images}
The basis of ultrasound imaging is the reflection of ultrasound at tissue boundaries within the body. Ultrasound is sound waves generated by a transducer, and as the wave travels through the body, echoes are generated by partial reflection at every boundary. The amount of reflection depends on the relative change in acoustic impedance at the boundary. The echo is recorded by the transducer, and it is then displayed in the image according its spatial origin. The speed of sound is approximately \SI{1540}{\metre\per\second} for all soft tissue in the body, and it is thus easy to calculate the origin by measuring the time of travel of the echo. The brightness in the image is proportional to the strength of the echo. This is called B-mode(brightness mode), and is the most common imaging mode. 

\subsection{Physics of ultrasound}
Ultrasound is sound waves with frequency above \SI{20}{\kilo\hertz}, which is the upper limit of audible sound. For the purpose of medical imaging, frequencies between \SI{1}{\mega\hertz} and \SI{18}{\mega\hertz} are common. Sound waves are pressure waves, and the pressure fluctuations cause temporal displacement of the medium in which the wave is traveling. For most purposes the displacement is along the direction of travel, this is know as longitudinal waves. Solids can support transverse sound waves, but that is outside the scope of this thesis.  The wavelength , $\lambda$, is determined by the frequency of the source and the phase velocity in the medium, $c$, i.e. $\lambda= \frac{c}{f}$. The phase velocity $c$ is given by

\begin{equation}
\label{phace velocity}
c = \sqrt{\frac{1}{\rho \kappa}},
\end{equation}

where $\rho$ and $\kappa$ are the tissue density and compressibility. For soft human tissue the phase velocity is about \SI{1540}{\metre\per\second}.%, while it is about \SI{4000}{\metre\per\second} for bone.
The compressibility is a measure of the relative change in volume from change in pressure, that is $\kappa = -\frac{1}{V}\frac{\partial V}{\partial p}$. Here $V$ is volume and $p$ pressure. 

The propagation of a sound wave traveling in the x-direction can in a fluid be described by the wave equation, i.e.
\begin{equation}
\label{wave equation}
\frac{\partial^2W}{\partial x^2} = \frac{\rho_0}{G}\frac{\partial^2W}{\partial t^2}.
\end{equation}

Here $W$ is the particle displacement, $\rho_0$ the density of the medium and $G$ the bulk modulus. The bulk modulus is the reciprocal of the compressibility, and is a measure of the volume stiffness. A general solution to this equation is 
\begin{equation}
\label{particle displacement}
W = W_0 \exp^{i(kx - \omega t)},
\end{equation}

where $\omega$ is the angular frequency, $k = \frac{\omega}{c}$ the wave number, and $c$ the phase velocity as given in Equation \eqref{phace velocity}, with $\kappa$ substituted by $\frac{1}{G}$. The pressure variations is then given by particle velocity $u_x = \frac{\partial W}{\partial t}$ as 

\begin{equation}
\label{pressure wave}
p_z = \rho_0 c u_x.
\end{equation}

It is important to emphasize the difference between the phase velocity and the particle velocity. The phase velocity is the velocity of energy carried through the medium, while the particle velocity is the velocity of the local displacement of particles.

The acoustic impedance $Z$ of the medium is defined as the ratio of the pressure to the particle velocity,
\begin{equation}
\label{acoustic impedance}
 Z = \frac{p_x}{u_x} = \rho c.
\end{equation}
   %%%Sjekk fortegn?????????????????

 
\subsection{Reflection}
Reflection provide the basis for ultrasound imaging, and occurs when the wave encounter a planar surface with a change in acoustic impedance. Across this surface both the pressure and particle velocity have to be continuous. From these boundary conditions 
%The acoustic impedance $z$ is given by  
%\begin{equation}
%z = \rho c.
%\end{equation}
the intensity coefficients of the reflected $r_i$ and transmitted $t_i$ wave can be given as \cite{wells1969physical}
\begin{equation}
\label{fresnel}
r_i = \left(\frac{z_2 \cos \theta_i - z_1 \cos \theta_t}{z_1 \cos \theta_t + z_2 \cos \theta_i}\right)^2
\end{equation}

\begin{equation}
\label{fresnel2}
t_i = \frac{4z_2 z_1 \cos^2 \theta_i}{(z_2 \cos \theta_i + z_1 \cos \theta_t)^2}.
\end{equation}

Here $z_1$ and $z_2$ are the acoustic impedance in medium 1 and 2, shown in Figure ?????. The angle of reflection $\theta_r$ is equal the angle of incidence $\theta_i$, while the angle of transmission $\theta_t$ is given by Snell's law\cite{blackstock2000fundamentals}, $ c_2 \sin \theta_i = c_1 \sin \theta_t$. 

\subsection{Absorption}
Loss of kinetic energy to heat from the propagating wave to the surrounding medium is known as absorption. Absorption occur continuously and reduce the amplitude of the wave. For linear propagation of a pressure wave, this can be described as 

\begin{equation}
p(x) = p(0)\exp^{-\alpha_A(\omega)x},
\end{equation} 

where $p(0)$ is the initial pressure amplitude, $alpha_A$ the absorption coefficient and $\omega$ the angular frequency.

For non-linear propagation the absorption depends on the local amplitude, and in the diagnostic intensity range the non-linear interaction is proportional to the square of the intensity\cite{:/content/asa/journal/jasa/97/3/10.1121/1.412091}. Non-linear propagation also affects the shape and frequency spectrum of the propagating wave, and can usually not be ignored for contrast agent applications \cite{Healey2012}. 

%\subsection{Non-linear imaging??}
 
\subsection{Scattering}
Scattering is reflection that occur when the size $d$ of the surface encountered is comparable or smaller than the wavelength. Scattering can arise from inhomogenities in compressibility or density. Scattering reflects the wave in a large range of directions, and the backscattered signal received at the transducer is therefore weak compared to reflected echoes. The magnitude and direction of the scattering depends on the size of the scatterer and increase strongly with the frequency.  

\begin{figure}[h]
  \centering
  \includegraphics[scale=0.8]{figurer\scattering2.png}
  \caption{Fig:Scattering}
\end{figure}
If we consider an incoming plane wave propagating in the direction $hat{i}$, see Figure \ref{Fig:scattering}, the incident pressure, $p_i$, at the scatter located at $\textbf{r_0}$ is then

\begin{equation}
p_i(\textbf{r_0}, t) = p_0\exp^{i(\textbf{r_0k_i}-\omega t)}.
\end{equation}

If we only consider the far field the scattered wave, $p_s$, at the observer at $\textbf{r}$ is given by \cite{Healey2012}

\begin{equation}
p_s(\textbf{r}, t) = f(\hat{r},\hat{i})\frac{\exp^{ik_s(\textbf{r-r_0}}{\abs*{\textbf{r-r_0}}}p_i(\textbf{r_0}, t).
\end{equation}

The $f(\hat{r},\hat{i})$ is the scattering amplitude function. The scattered intensity is

\begin{equation}
I_s = \frac{1}{2}\frac{\abs*{p_s}^2}{\rho c} =\frac{1}{2}\frac{\abs*{p_i}^2}{\rho c}\frac{\abs*f(\hat{r},\hat{i})}^2}{\abs*{\textbf{r-r_0}}^2} = I_i \frac{\abs*{f(\hat{r},\hat{i})}^2}{\abs*{\textbf{r-r_0}}^2},
\end{equation}

where $I_i$ is the incident intensity. The differential scattering cross-section is defined as $\sigma_d = |f(\hat{r},\hat{i})|^2$. The scattering cross section is the integral of the differential cross section over all solid angles, i.e.

\begin{equation}
\label{solid angle}
\sigma_s = \int_{4\pi}\sigma_d \mathrm{d}\Omega.
\end{equation}

The Rayleigh scattering model is the simplest model for scattering of small particles, i.e. particles with a diameter $d << \lambda$. This model do not take damping or resonance effects into account. If we again consider the plane wave, the differential scattering cross section is given by \cite{morse1986theoretical}

\begin{equation}
\label{rayleigh cross section}
\sigma_d = k^4a^6\abs*{\frac{G-G_0}{3G}-\frac{\rho-\rho_0}{2\rho+\rho_0}\cos\theta}^2.
\end{equation} 

The total cross-section is obtained using Equation \eqref{solid angle} and \eqref{rayleigh cross section},

\begin{equation}
\label{total cross-section}
\sigma_s = 4\pi k^4 a^6 \left[\left(\frac{G-G_0}{3G}\right)^2 +\frac{1}{3}\left(\frac{\rho-\rho_0}{2\rho + \rho_0}\right)^2\right].
\end{equation}
In these equations $k$ is the wavenumber, $a$ the radius of the scatterer, and the zero subscript refer to properties of the surrounding medium. The angle $\theta$ is the angle between the incident wave and the scattered wave, i.e. $\theta = 180\deg$ is the direction of backscatterering. Although these equations represent a coarse approximation, it demonstrates why gas bubbles are excellent contrast agents. The density term in Equation \eqref{total cross-section} is limited to $\nicefrac{1}{3}$, while the bulk modulus term has no upper limit as $G$ gets small compared to $G_0$. Thus, it is the compressibility and not the acoustic impedance which is the main cause to the scattering off a gas bubble. 

It is also important to keep in mind that it is the backscattering and not the total scattering which determines the signal at the transducer, and Equation \eqref{rayleigh cross section} give therefore a more correct image than Equation \eqref{total cross-section}. 
%Include table from andys Ultrasound paper?
%Include figures os differntial cross-sections?


\subsubsection{Resolution and depth of view}
The ultrasound waves used in medical imaging are emitted as short pulses, where the pulse length determines the longitudinal resolution. It is not possible to distinguish two points in the longitudinal direction separated by a distance shorter than half the pulse length. A short pulse is therefore desirable in order to produce good longitudinal resolution. The theoretical minimum pulse length is one wavelength, although this is difficult to achieve in practice, and a few wavelengths is a more realistic minimum. A short pulse will contain a wider band of frequencies than a long pulse. 

The required depth of view will depend on how deep the tissue subject to the imaging is located, but there will always be a finite required depth penetration. The attenuation of ultrasound is proportional to the frequency, so a good depth of view implies a low frequency. Hence, there will always be a trade off between good longitudinal resolution and depth of view, and the diagnostic frequency will be chosen according to the patient and application. 

Lateral resolution is the minimum distance perpendicular to direction of propagation required to distinguish two objects. This is determined by the width of the beam, and may vary with the depth of view. It is common to focus the beam to obtain the best resolution at the depth of interest. Strong focusing will give very good resolution at a very limited depth range, while weak focusing give medium resolution through most of the image. Temporal resolution is determined by the frame rate, which is typically between 10 and 30 frames per second. 

\section{Tumor/cancer}
\subsection{Tumor growth and properties}

 A tumor is a mass of tissue with abnormal growth, and may either be benign or malignant. A benign tumor is localized with a  well-defined boundary and do usually not pose any health threat. A malignant tumor is what we know as cancer, and can invade adjacent tissue, perform metastasis and be life threatening. 

A cancer starts off from one abnormal cell, which proliferates through cell division resulting in uncontrolled growth. To maintain growth beyond a critical size of about \SI{1}{\milli\metre}\cite{king2006cancer}, new blood vessels are induced through angiogenesis to meet the need of nutrients and oxygen. 

The cell growth is regulated by two types of genes. Proto-oncogenes regulate normal cell proliferation, while the tumour suppressor genes control repair of cell damage, cell death and growth inhibition. Mutation or loss of any of these regulatory systems can cause cancer growth. 

The invasion of adjacent, normal tissue distinguish the malignant tumour from the benign, and this happens along along the pathways of least resistance, i.e. along vessels or fascia. This invasion is enhanced by increased amount of proteases outside the tumor boundary, which increase mobility of the cancerous cells. This also enhance the metastasis, where the cancer is spread along the lymphatic vessels, blood vessels or the peritoneal cavity to a new site.
 
An important feature which cause the possibility of localized drug delivery, is the structure of the cancer vasculature. This vasculature is developed through the angiogenesis, which is enhanced by Vascular endothelial growth factors(VEGFs). The VEGFs diffuse through the extracellular matrix(ECM) and connect to receptors on the inner surface of the vessel walls(Endothelial cells)\cite{Koumoutsakos2013}. This stimulate both the production of more endothelial cells and the construction of new vessels through the extracellular matrix\cite{Nishida2006}. This results in a structure which differ from normal vasculature. The tumour vasculature has increased vessel density and vessel size, a large amount of dead-ends and a disordered branching pattern.    

The abnormal level of endothelial cells leads to vascular walls which lack in coverage of perivascular cells and tight adherens junctions which stabilize the vessel. Hence, there will be large intracellular spaces in the vessel wall, and the vessel becomes leaky. This allow the cancer cells to enter the vasculature, enhance macromolecular transport through the vessel wall, and increase interstitial pressure within the tumour. This is known as the \textit{enhanced permeability and retention}(EPR) effect. The EPR effect leads to accumulation of macromolecules within the tumour, and enhance local delivery of cancer drugs.



\subsection{Cancer treatment}
Despite the massive amount of research on the field of cancer, all existing treatments suffer from side-effects and limited results. Most types of cancer are strongly related to lifestyle. Prevention is therefore the most important action in the battle against cancer. Another important action is screening, which can reveal cancer in an early stage. Early detection has shown to be decisive for the outcome of the cancer treatment\cite{king2006cancer}\cite{Jordan1986}.

The current treatments can be divided into chemotherapy, radiotherapy or surgery. Surgery aim to remove the entire tumour and, if possible, any metastases in regional lymphatics. Surgeries may be difficult to perform without damaging adjacent tissue. Radiotherapy is performed by radiating the tumour with high-energy X-rays to kill the malignant cells by causing fatal damage to their DNA. Some healthy tissue such as the lens of the end the spinal cord are very vulnerable to radiation, and can be damaged it exposed to a high radiation dose. Both of these treatments are best suited for not-metasized tumours.  

\subsection{Ultrasound mediated drug delivery}



\subsection{Contrast agents}
\label{contrast agents}
%SHell: Albumin, lipid, polymer, Nanoparticles??
%%%Ta med et avsnitt om forskjellige oscillasjoner til bobler. Kollaps, stabil osciallsjo, non-linear
%%Targeted microbubbles

Contrast agents are used to increase the image sensitivity and signal-to-noise ratio in medical imaging. In ultrasound imaging, this is accomplished by intravenous injection of a solution containing gas-filled microbubbles. Microdroplets are also used\cite{Soman2006}, but in much less extent than gas bubbles and is outside the scope of this thesis. Microbubbles are favourable because of the higher echogenicity\cite{Talu2008}.

The microbubbles have to fulfill several requirements to perform as a contrast agent. The contrast agent has to be delivered to the area of interest, and this set reguirements to lifetime and size. The diameter has to be smaller than \SI{8}{\micro\metre} to pass the pulmonary capillary\cite{Tickner1980}, which is the size limiting factor in the circulatory system. Strong backscattering of ultrasound is important. Absorption is an unwanted effect as it attenuates the ultrasound wave without contributing to the received signal. It is also important that the contrast agent is well tolerated by the body and is able to leave the circulatory system either by dissolving or by being phagocytosed by the kupfer cells in the liver\cite{Healey2012}.

The first contrast agents were made by saline, and was put to use by cardiologists in the 1960s for identification of mitral valve echoes. The saline was shaken before injection to create the microbubbles. Current available contrast agents have been able to fulfill the requirements stated, and consist of a gas enclosed in a suited shell. The shell has to be biocompatible and is made from fat, proteins or polymers. The advantage of a shell is increased lifetime and scattering of ultrasound. The size of the microbubbles is approximately equal to the size of red blood cells, i.e. (\SIrange{2}{6}{\micro\metre}).

The microbubbles have two important properties which cause large scattering. The first reason is the big difference in acoustic impedance between microbubbles and blood. The second property is the compressability of the gas inside the microbubble. The pressure fluctuations caused by the ultrasound forces the microbubble to expand during rarefaction and contract during compression. The microbubble will oscillate with the same frequency as the ultrasound source, and the scattering will be strongest at the resonance frequency of the microbubble. The resonant frequency is determined by the properties and the size of the shell.

The use of contrast agents has been limited to diagnostics, but these microbubbles possess properties suited therapeutic use. The microbubbles can serve as carriers for ultrasound mediated drug delivery \cite{Dijkmans2004}, see section ?????. 

\subsection{Phase-shift bubbles}
%Kanskje dette skal under metode???
The drug carrier used in this project is a two component particle composed of negatively charged gas microbubbles and positively charged droplets, known as Acoustic cluster therapy (ACT). Both the gas microbubble and the droplet have an initial size about \SIrange{2}{3}{\micro\metre}. The gas microbubble consist of a low solubility perfluorocarbon gas encapsulated in a negatively charged phospholipid membrane, e.g. Sonazoid\texttrademark. The drug is dissolved in a perfluorated oil phase and stabilized by a positively charged phospholipid membrane. When the bubbles and droplets are mixed, clusters of droplets and bubbles will form due to electrostatic attractive forces.


\begin{figure}[h]
  \centering
  \label{Fig:Sonazoid}
  \includegraphics[scale=0.8]{figurer\PS_compound.png}
  \caption{Illustration of phase-shift bubble, with the gas microbubble(left) and microdroplet(right).}
\end{figure}


When the clusters are exposed to ultrasound of standard medical frequency and intensity(??),the microbubble will oscillate and transfer energy to the droplet through mechanical interactions at the boundary. This initiate a fusion into a gas and liquid mixture, encapsulated by a mixed surfactant membrane. The fluid will vaporize and expand to a gas bubble of approximately \SI{30}{\micro\metre}. The enlarged gas bubble can block the capillary network and maintain the local concentration of the released drug. 

If we assume rapid thermal conduction from surrounding blood, the partial pressure of pf-MCP will be close to the vapour pressure at body temperature. This vapour pressure is lower than the local hydrostatic pressure, and the difference is initially equalized by the gas from the microbubble. An inward diffusion of $\mathrm{O_2, N_2, CO_2, H_2O and Ar}$ will exist simultaneously, and these gases will also contribute to the equalization of the hydrostatic pressure. The inward diffusion is driven by partial pressure gradients of the respective gases. 

The evaporation will not occur if the blood is under-saturated of the aforementioned gases. If the hydrostatic pressure inside the bubble is too large, there will be no pressure gradient to drive the inward diffusion. This is caused by too high surface tension or hydrostatic pressure in the surrounding blood, and this stop the evaporation. This initial evaporation occurs in a second or less\cite{Healey2013}. The inward diffusion may continue after the evaporation, until a maximum size is reached after approximately 20-30 seconds(Ref?).

Application of low MI and low frequency(\SIrange{0.1-2}{\mega\hertz} ultrasound will drive an oscillation of the large phase-shift bubbles and increase the permeability of the adjacent vasculature(REF?). This allow diffusion of the drug through the vessel walls to the cancer cells. The increased permeability will cease when the ultrasound is turned off leaving the drug trapped within the tumour.

A more mathematical description of the aforementioned evaporation is given now\cite{Healey2013}. We assume we can use the simplification that the gases are ideal gases, i.e. we can use the ideal gas law
\begin{equation}
\label{ideal gas law}
 PV = nRT. 
\end{equation}  
Hence, the volume of evaporated gas, $v_{pf}$, is a function of the initial volume and pressure in the oil droplet, i.e.

\begin{equation}
\label{gas volume}
V_g(V_{pf}, p_{pf}) = \frac{n_{pf}RT}{p_{pf}}=\frac{V_{pf}\rho_{pf}RT}{M_{pf}p_{pf}}.
\end{equation}

We assume the body temperature $T$ to be \SI{310}{\kelvin}. If also assume that both the gas and liquid bubble are spherical, we have the simple relation between diameter and volume, $V = \frac{\pi d}{6}$, and using this and Equation \eqref{gas volume}, we get an expression for the diameter of the gas bubble, 

\begin{equation}
\label{diameter}
d_g(p_{pf}, d_{pf}) = d_{pf}\sqrt[3]{\frac{\rho_{pf}Rt}{M_{pf}p_{pf}}}.
\end{equation}

From this equation we can get the size of the gas bubble after the initial evaporation. For an initial diameter of \SI{4}{\micro\metre} we get a diameter about \SI{23.4}{\micro\metre}, knowing that the vapour pressure of pf-MCP at body temperature is \SI{76}{\kilo\pascal}\cite{Healey2013}.
To include surface tension we use the Young-Laplace equation for a sphere,
\begin{equation}
\label{Young-Laplace}
\Delta p = \gamma\frac{4}{d},
\end{equation}

to get an expression for the pressure in the surrounding fluid, 

\begin{equation}
p_w = p_{partial} + p_{pf} - \gamma frac{4}{d_g(p_{pf}, d_{pf})}.
\end{equation}
Here $p_w$ is the total pressure in the surrounding fluid, while $p_{pf}$ is the partial pressure from the gases in the surrounding fluid. Combining equations above we get a cubic expression for the gas bubble diameter,  

\begin{equation}
\label{cubic}
(p_w-p_{partial})D_g^3 + 4\gammaD_g^2 - \frac{6v_{pf}\rho_{pf}RT}{\pi M_pf}.
\end{equation}

This equation has a real solution, and can be used to calculate an upper limit for the phase-shift gas bubble diameter. This is shown in Figure ???.
%Insert figures.

%Dynamic calcultions 
In addition to the static description above, we can derive an expression for the dynamic bubble growth. The following is a simplified model.

There exist a mechanical equilibrium(pressure) at the bubble boundary, and from the ideal gas law (Equation \eqref{ideal gas law}) we have that

\begin{equation}
\label{mec eq}
p_A + p_{pf} = \frac{2\gamma}{r} + p_{atm}+p_{blood}, 
\end{equation}
where
\begin{equation}
p_A + P_{pf}= (C_A+C_{pf})RT.
\end{equation}

Using both Fick's first law of diffusion and that a change in mass have to cause a flux through the boundary ($J = -\od{n}{t}$), we have that

\begin{equation}
\label{flux}
J_{pf} = -\od{}{t}\left(\frac{4\pi C_{pf}r^3}{3}\right) = 4\pi r(c_{pf}(r)-c_{pf}(\infty)),
\end{equation}
and a similar expression for $J_A$. Note that the diffusion flux $J$ is given in mol per second. We can assume that the concentration of pf-MCP goes to zero for far from the bubble, i.e. $ c_{pf}(\infty)=0$. We get two differential equations, 
\begin{equation}
\label{diff1}
-\od{}{t}(C_{pf}r^3) = 3rD_{pf}L_{pf}C_{pf},mathrm{ and } -\od{}{t}(C_Ar^3)=3rD_AL_A(C_A-\frac{p_{air}}{RT}. 
\end{equation}
Here $L$ is the concentration Ostwald coefficient describing the solubility of a gas, $L = \left(\frac{c}{C}\right)_{equilibrium}$\cite{Equilibria1984}. Lower-case $c_x$ is the concentration $x$ in the liquid phase, while upper-case $C_x$ is the concentration of $x$ in the vapour phase. Under saturation of air is incorporated in the term $\frac{p_{air}}{RT}$.

We rewrite the these equations with dimensionless variables to get
\begin{equation}
F + A = \mu \rho^2+(1+\vartheta)\rho^3,\quad \od{F}{\Gamma}=-\frac{3L_{pf}}{\rho^2}F\quad mathrm{and}\quad \od{A}{\Gamma}=-\frac{3\delta L_A}{\rho^2}(A-p_d\rho^3),
\end{equation}
with the dimensionless variables
\begin{multline}
\label{dim}
\mu=\frac{2\gamma}{p_{atm}r_0}, \quad \vartheta = \frac{p_{blood}}{p_{atm}}, \quad \rho = \frac{r}{r_0}, \quad \chi_A =\frac{C_ART}{p_{atm}},\\
\chi_{pf} = \frac{C_{pf}RT}{p_{atm}}, \quad \Gamma = \frac{D_{pf}}{r_0^2}t, \quad A = \chi_A\rho^3, \quad F = \chi_{pf}\rho^3, \quad and \quad p_d = \frac{p_{air}}{p_{atm}}.
\end{multline}

Combining equations we get a differential equation for $\rho$, 
\begin{equation}
\label{diff3}
\od{\rho}{\Gamma} = \frac{-3\delta L_A(A-p_d\rho^3)- 3L_{pf}(\mu\rho^2+(1+\vartheta)\rho^3-A)}{\rho^3(2\mu+3(1+\vartheta)^rho)}.
\end{equation}

These three differential equations can be solved using appropriate initial conditions. Defining the variable $X_{pf}$ to be the initial mole fraction of pf-MCP, we get the initial conditions

\begin{equation}
F(0) = X_{pf}(\mu +\vartheta +1),\quad A(0)=(1-X_{pf})(\mu+\vartheta+1)\quad
\mathrm{and}\quad \rho(0)=1.
\end{equation}

The growth of the bubble radius and volume is calculated from these differential equations and shown in Figure ???.


\subsection{Transducer}
??

\subsection{Image processing of B-mode images}
After the echoes have reached the transducer a signal is produced by making an image with the brightness at each pixel determined by the strength of the echo from that corresponding distance and direction. The first step in the image processing is to convert the signal from analogue to digital. The digital signal is less vulnerable to noise and distortion, and it enables further digital image processing. Then a linear amplifier apply the same amount of gain to the entire signal, to make the signal strong enough for further processing. Time-gain compensation is then applied to make echoes from similar interfaces equal, regardless of the depth of their origin. This is performed by increasing the gain with increasing depth of echo. The depth of the echo is identified by the arrival time at the transducer. The rate of attenuation of ultrasound with depth is determined by the frequency and tissue.

After amplification and time-gain compensation the dynamic range of the signal is about 60 dB. The dynamic range of a signal is defined as the ratio between the largest amplitude that can be recorded without causing distortion and the lowest amplitude that can be distinguished from noise. The dynamic range of a common screen is about 20 dB. The signal must therefore be compressed before it can be displayed. To compress the dynamic range from 60 to 20 dB, an amplifier with non-linear gain is applied. Low amplitudes are amplified more than high, and the dynamic range is therefore decreased. Compression allows weak echoes from scattering within tissue to be displayed together with strong echoes from tissue interfaces.

\subsubsection{RF and IQ data}
RF is short term for radio frequency data which is used in ultrasound as a description for unprocessed data. IQ is short term for in quadrature, and refers to a demodulation of the RF signal to reduce the amount of storage space without loss of information. IQ modulation converts the signal from the real to the imaginary space. The IQ signal is obtained using a IQ-demodulator to down-mix, low-pass filter and decimate the RF signal. The IQ signal can also be computed by through a Hilbert transform\cite{Kirkhorn1999}.

\subsubsection{Hilbert transform}
The Hilbert transform is a linear operator which acts on a signal $u(t)$ to derive an analytic signal. The Hilbert transform convert the signal from real to complex space by adding or subtracting 90 degrees. It is therefore also known as a phase-shift operator. An analytic signal has by definition only positive frequencies in its Fourier transform, and is related to the Hilbert transform through 

\begin{equation}
\tilde{x}(t) = x(t) + x_h(t),
\end{equation}

where $x(t)$ is the signal, $x_h(t)$ the Hilbert transform of the signal, and $\tilde{x}(t)$ the analytic signal. The Hilbert transform can be written as a convolution, 

\begin{equation}
x_h(t) = x(t)*\frac{1}{\pi t},
\end{equation}

which can be interpreted as a filtering operation with a quadrature filter which shifts all sinusoidal components by a phase shift of $\frac{\pi}{2}$. The envelope is the amplitude of the analytic signal, and a B-mode image is created from the envelope of the signal. 

\subsubsection{Downsampling}
Because it is the envelope that is used to create the B-mode images, further reduction in data size can be achieved by downsampling of the RF signal. The RF signal is downsampled by a decimation factor $M$, by only recording every $M$th sample. 

The RF signal is a strictly bandlimited signal, limited by the bandwidth of the transducer. The envelope is thus also a bandlimited signal, with a finite maximum frequency. According to the Nyquist-Shannon sampling theorem, this signal can be sampled without aliasing or loss of information by a sampling rate twice the maximum frequency. Hence, the decimation factor, $M$, is determined so that

\begin{equation}
\label{deciamtion}
\frac{Fs_{RF}}{M} > 2f^{max}_{envelope},
\end{equation}

where $Fs_{RF}$ is the sampling frequency of the RF signal, and $f^{max}_{envelope}$ is the maximum frequency of the envelope. For a thorough explanation, see \cite{Crochiere1981}.



\subsection{Harmonic Imaging}

\subsection{Matlab}
The most important to write here is how the methods used in matlab work IN THEORY, not how they are implemented in MATLAB.


\subsubsection{Removing image artifacts}
The operation of removing movement artifacts are based on the Matlab toolbox Image Processing and the use of image registration. \textit{imregister} and \textit{imregconfig} are the first two functions that have been applied and tested. This is intensity based automatic registration. Control point registration may be another option.

%DETTE ER KANSKJE IKKE VIKTIG
%The process is initiated by making a \textit{metric} and an \textit{optimizer} object using the \textit{imregconfig} function. The \textit{metric} object measures the similarity of the two images. The \textit{optimizer} contains the optimization parameters such as maximum number of iterations, initial step length, optimization algorithm etc. Which optimization algorithm and how the image similarity is measured can be chosen in the \textit{imregconfig} function. The options for optimization algorithm are either a regular step gradient or a one-plus-one evolutionary method. For the metric object the similarity can be measured either by a mean square error approach or by making a mutual information metric. The mutual information metric maximizes the number of pixel with the same relative pixel value, and is best suited for images with different brightness ranges.[REF MATLAB]

\subsection{Speckle}
Speckle is a random, deterministic speckle pattern present in all types of coherent imaging, thus also in ultrasound images. The speckle is formed by scatterers smaller than the resolution of the imaging system, and the shape and size of the speckle pattern is determined by the dimensions of the imaging system and the structure of the imaged tissue.

The speckle is an interference phenomenon created by coherent waves with different phase and amplitude added together. If several echo waves arrive at the same piezoelectric element within a time span shorter than the emitted pulse, the piezoelectric element will not be able to distinguish the waves, and their impact will be added. 

\subsection{Image registration}
Image registration is the process where one image is spatially aligned to a reference image. The image to be aligned is called the moving image, while the reference image is called the fixed image. Image registration  is an important part of image processing and can both be used to remove motion artifacts or to fuse images of the same object, captured with different image modalities or from different directions. 

Image registration can initially be divided into extrinsic and intrinsic methods. Extrinsic methods are based on foreign objects placed into the scene before the image is captured. This has the advantage of simple, feature-based registration, but the preplacement and removal of these objects may not be trivial. 

Intrinsic methods can be divided into feature and intensity based registration. Features can either be easily recognizable points identified by the user, or structures which can be extracted from the image by image segmentation. These methods are mostly used in rigid transformations, and have the advantage of being simple computations once the features are determined. One drawback is that the registration often is limited by the reliability of the segmentation or identification of the features. Although these methods are applicable to both multi- and monomodal registration, and to different body parts, their use have in general been limited to neuroimaging and orthopedic imaging\cite{Maintz1998}.

Intensity based registration differs from the other methods by using the intensity pixel values directly to compare the fixed and moving image. To compare the images a suited similarity measure is used, see section\ref{subsec:similarity}. Using different similarity measures, this method is suitable for both multi- and monomodal registration. The image registration is then performed using an optimization routine to find the spatial translation of the moving image which minimizes the chosen similarity measure. Choosing the right similarity measure and optimization scheme is essential to get a satisfying result. For a full review of this topic, see \cite{Maintz1998}.

\subsubsection{Basic theory of image registration}
\cite{Mainstream}
%Should non-parametric transformations be mentioned?

%Transformation of an image is the basis for image registration, and can be described as a function $y$ mapping the image coordinates from $\Real_d \arrow \Real_d$ using a linear combination of basis functions and coefficients.

\subsubsection{Image transformations}

Transformation of an image is the basis for image registration, and can be described as a mapping of a coordinate vector $\vec{x}$ from the space $X$ to a new coordinate vector $\vec{y}$ in the space $Y$. The transformation is performed by a transformation matrix $A$, i.e. $\vec{y} = A\vec{x}$.

In 2D a rigid transformation can be written as 

\begin{equation}
	\label{rigid}
	\begin{pmatrix}
		y_1 \\
		y_2 \\
		1 
	\end{pmatrix}
	=
	\begin{pmatrix}
		R_{11} & R_{12} & T_1\\
		R_{21} & R_{22} & T_2\\		
		0 & 0 & 1
	\end{pmatrix}
	\begin{pmatrix}
		x_1\\
		x_2\\
		1
	\end{pmatrix},
\end{equation}
where $R_{ij}$ are elements in the rotation matrix 

\begin{equation}
	R = 
	\begin{pmatrix}
	\cos \theta & -\sin \theta\\
	\sin \theta & \cos \theta
	\end{pmatrix}.
\end{equation}
The rotation matrix rotates the coordinates an angle $\theta$ around the origo. The matrix elements $T_{ij}$ determines the translation of the coordinates. An affine translation is described by the matrix 
\begin{equation}
\begin{pmatrix}
 a_{11}&a_{12}&a_{13}\\
 a
 
\end{pmatrix}
\end{equation} This enables shear and scaling of the image.

If we apply a transformation matrix to an image, we get the new pixel coordinates of transformed image, but these points will not be on the grid coordinates of the image. For that reason, an interpolation scheme is applied to the transformed image coordinates to get the new pixel values at the grid coordinates. A suitable interpolation scheme must be chosen according to the given problem, but the most common are described in section ??????

Image transformation can be divided into linear and non-linear transformation. For a linear transformation the same transformation matrix is applied to the whole image, whereas this is not the case for a non-linear transformation. Linear transformation is computationally faster and simpler, and less affected by noise. On the other hand, non-linear transformations can correct for local deformations out of reach for the linear transformation. 

The difficult part of image registration is not to apply the transformation, but to obtain the right transformation matrix. The simplest case is the point or feature based method, where easily recognized feature are localized in both the fixed and moving image, either by interaction from the user, or by using a feature detection algorithm. In the case of an affine transformation, six pair of corresponding coordinates are needed to solve the set of linear equations to obtain the the six unknown elements in the transformation matrix. Usually, more points are obtained, and the transformation matrix is calculated using a least-squares approach. 

\subsubsection{Similarity measure}
\label{subsec:similarity}
There are several ways of measuring the similarity between two images, where different approaches enhance different image properties, and are suitable for different problems. The choice of similarity measure will determine the minimum and the rate of convergence for the optimization scheme.

When the images to compare are from the same modality, they will be in the same intensity range. They will only differ because of noise, geometric transformation and changes in imaged object. Common similarity measures are then the sum of squared differences (SSD), the sum of absolute differences or the cross correlation. If the changes in the imaged object is sufficiently small, and we assume the noise to be Gaussian, it is shown in \cite{Viola1997} that SSD give the optimal result.
For two images $A$ and $B$ the SSD is given as
\begin{equation}
\label{SSD}
\mathrm{SSD} = \frac{1}{N}\sum_{i}^N \abs{A_i -B_i}^2, \forall i \in A \cap B,   
\end{equation}

where $i$ is an image pixel. 

In multimodal image registration, the images have neither similar intensities or even a linear relationship between the intensities. For this type of problems, mutual information is the most common similarity measure. Mutual information is a measure of the statistical dependence of the two images, and the alignment is optimal the moving image contains the maximal amount of information about the fixed image. For a detailed review of mutual information in medical multimodal imaging, see \cite{563664}.  

\subsection{Regular step gradient descent}
An optimization scheme determines the optimal alignment by optimizing the chosen similarity measure. The optimization scheme is usually chosen through an empirical approach, where computational demand, reliability and stability are important factors. 

The gradient descent method is a first-order algorithm which finds the local minimum of a multivarible, differentiable function $F(x)$. From a given initial state $x_k$, the optimizer moves a distance $\gamma$ in the direction opposite to the gradient, i.e.

\begin{equation}
\label{gradient descent}
x_{k+1} = x_k - \gamma \Delta F(x_k).
\end{equation}

This method suffers from the reliability of the step distance $\gamma$. Too long or short step will give a slow, or no convergence. The \textit{regular step gradient descent} is a variation where the step length is halved every time there is a significant change in the direction of the gradient. The optimization terminates after reaching a the minimum, at a minimum step length, or after a maximum number of iterations.

To speed up the image registration, a pyramidal method can be applied together with the gradient descent. A smoothing filter is applied to the images, before they are decimated by a factor 2. This is performed for a given number of pyramid levels, resulting in a set of images with decreasing size. The scheme starts by optimizing the translation for the smallest image, and the optimal translation is used as an initial translation in the next pyramid level. In addition the increased speed, there is also less chance of getting stuck in a local minimum due to the smoothing which is applied before each decimation.
 









%metode
%Materials
%Procedures
%Instrumentation
\section{Materials and methods}
\subsection{Image recording}
All ultrasound images have been recorded by Annemieke van Wamel and Andrew Healey prior to this study.
\subsubsection{The ACT\texttrademark{} clusters}
The ACT\texttrademark{} clusters are composed of \SI{3}{\micro\liter\per\milli\liter} microdroplets (pf-MCP) and \SI{8}{\micro\litre\per\milli\litre} (\SI{1.2e8}{\per\milli\litre}) of microbubbles (HEPS-Na stabilized PFB microbubbles[Sonazoid\texttrademark{}]). Two different doses were used:

\begin{enumerate}
	\item Low dose: \SI{50}{\micro\litre} of the mixture, diluted 1:4. contains \SI{0.72}{\micro\liter\per\milli\liter} oil droplets and \SI{2}{\micro\liter\per\milli\liter} microbubbles.
	\item High dose: \SI{50}{\micro\litre} of the mixture, neat, contains \SI{3}{\micro\liter\per\milli\liter} oil droplets and \SI{8}{\micro\liter\per\milli\liter} microbubbles.
\end{enumerate}  
 
The Sonazoid\texttrademark{} microbubbles are stored lyophilized, and reconstituted with sterile water to create a liquid solution. They are used both as a contrast agent, and as microbubbles together with microdroplets to compose the ACT\texttrademark{} clusters (Section \ref{sec:phase-shift bubbles}).  %Anyhting more here?

\subsubsection{Imaging setup}
%16Mhz
All ultrasound images evaluated during this work are imaged with the Vevo\texttrademark{} 2100 imaging system from Visual Sonics~\cite{Coulthard2009}. Two different imaging modes have been used, linear and non-linear contrast mode. Linear contrast mode was used to emphasize the ACT\texttrademark{} bubbles, while non-linear mode enhances the contrast from the Sonazoid\texttrademark{} microbubbles. The imaging settings are presented in Table \ref{tab:Vevo Ultrasound settings}.

\begin{table}[htb]
\caption{Settings used in the Vevo 2100 ultrasound apparatus.}
\label{tab:Vevo Ultrasound settings}
\begin{center}
\begin{tabular}{@{}l l l @{}}\toprule
& \multicolumn{2}{c}{Value} \\ \cmidrule(r){2-3}
Parameter & Linear contrast mode & Non-linear contrast mode\\
\midrule
Application & Cardiac & Cardiac\\
Preset & Contrast & Cardiac\\
Transmit:\\
Frequency & \SI{16}{\mega\hertz} & \SI{18}{\mega\hertz}\\
Power & 3\% & 10\%\\
Gate & & 6\\
Beamwidth & & Standard
Acquisition:\\
Gain & 10 dB & 10 dB\\
2D Gain & & 10 dB\\  
Frame rate & 10 fps & 10 fps\\
Depth & \SI{21.00}{\milli\meter} & \SI{21.00}{\milli\meter}\\
Width & \SI{23.04}{\milli\meter} & \SI{23.04}{\milli\meter}\\
Sensitivity & &1\\
Line density & Standard & High \\ 
Persistence & None & Off \\
Display:\\
Dynamic range &45 dB &40 dB \\
Display Map & MB1 & MB2\\
Brightness & 50 & 50\\
Contrast & 50 & 50 \\
  \bottomrule
\end{tabular}
\end{center}
\end{table}

For activation of the ACT\texttrademark{} clusters a transducer was placed in a water bath a fixed distance from the tumor. This transducer was either a V-scan or a Vivid MS5. The settings used are given in Table \ref{tab:V-scan vivid}.

 
\begin{table}[htb]
\caption{The settings used for the V-scan and the Vivid MS5 to activate the phase-shift bubbles.}
\label{tab:V-scan vivid}
\begin{center}
\begin{tabular}{@{}l l l @{}}\toprule
& \multicolumn{2}{c}{Value} \\ \cmidrule(r){2-3}
Parameter & V-scan & Vivid E9\\
\midrule
Preset & Cardiac & \\
Frequency & & \SI{2}{\mega\hertz}\\
Image distance & \SI{8}{\centi\meter} & \SI{2}{\centi\meter}\\
Focus depth & \SI{5}{\centi\meter} & \SI{1.5}{\centi\meter}\\
Frames per second & &Maximum\\
Angle & & \ang{45}\\
Mechanical index (MI) & \num{0.5} & \num{0.28} \\
  \bottomrule
\end{tabular}
\end{center}
\end{table}

The mice received anesthesia before injection of ACT\texttrademark{} or Sonazoid\texttrademark{}. The mice were placed on a rat handling table, with the left leg (location of tumor) lifted horizontally and fixed. Ultrasound gel and a water-bath bag was put on top of the tumor. The imaging and activation transducers were fixed inside the water-bath-bag (Figure \ref{Fig:setup}).

\begin{figure}[h]
  \centering
  \includegraphics[width=\linewidth]{experimental setup.png}
  \caption{Experimental setup for activation and imaging. The Vevo MS250 transducer (A) and the V-scan transducer (B) is placed within the water-bath. The tumor side (C) of the mouse faces the camera.}
  \label{Fig:setup}
\end{figure}

\subsubsection{High ultrasound bursts}
\label{sec:high US bursts}
In some of the ultrasound recordings, high power ultrasound were applied by the Vevo 2100 transducer. This was utilized because the high power ultrasound disrupts the Sonazoid\texttrademark{} microbubbles, while the large phase-shift bubbles are unaffected. This difference is utilized to distinguish the phase-shift bubbles from the Sonazid\texttrademark microbubbles. The high power ultrasound were set to the maximum duration in linear contrast and non-linear mode. That is \({<}\SI{0.1}{\second}\) and \SI{~2}{\second}, respectively~\cite{Coulthard2009}.


\subsubsection{Experimental protocol}
\label{sec protocols}
Four different combinations of ACT\texttrademark{} dose and transducer were used in this study:

\begin{enumerate}
  \item Animal 1 to 4. Vivid transducer and low dose.
  \item Animal 5 to 8. V-scan transducer and low dose.
  \item Animal 9 to 12. Vivid transducer and high dose.
  \item Animal 13 to 16. V-scan transducer and high dose.
\end{enumerate}
 
Sonazoid\texttrademark{} microbubbles were also injected alone as 1:4 dilution, \SI{50}{\micro\litre}, for assessment of vascularity. All injections were administrated as bolus. The general protocol applied is given below~\cite{Healey2014}. A detailed list, including all exceptions, is found in Appendix (\ref{raw counting}).
\FloatBarrier
\begin{enumerate}
	\item Imaging vasculature using Sonazoid\texttrademark{}.
		\begin{enumerate}
		\item Pre-injection images. Non-linear contrast mode.
		\item Sonazoid\texttrademark{} administration. Non-linear contrast mode.
		\item Post-administration images. In general 5 minutes after the injection. Non-linear contrast mode.
		\end{enumerate}
	\item Imaging and activation of ACT\texttrademark{}.
		\begin{enumerate}
		\item Pre-injection images. Non-linear contrast mode.
		\item Pre-injection images. Linear contrast mode.
		\item ACT\texttrademark{} administration. Linear contrast mode.
		\item Post-administration images. Linear contrast mode.
		\item Post-administration images. Non-linear contrast mode.
		\item Post-administration images. Linear contrast mode.
		\item Two bursts of high power ultrasound. Non-linear contrast mode.
		\item Two bursts of high power ultrasound. Linear contrast mode.
		\end{enumerate}
	\item Imaging vasculature using Sonazoid\texttrademark{}.
		\begin{enumerate}
		\item Pre-injection images. Non-linear contrast mode.
		\item Sonazoid\texttrademark{} administration. Non-linear contrast mode.
		\item Post-administration images. In general 5 minutes after the injection. Non-linear contrast mode.
		\end{enumerate}
\end{enumerate}  
 

\subsubsection{The animals}
Prostate adenocarcinoma cells (PC-3) were used in this research. The tumor cells were implanted in 40 female Balb/c mice at the age of 9-11 weeks. The tumor was implanted on the left hind limb, on the lateral side between the knee and the hip. The tumor cells were injected subcutaneously as a \SI{100}{\micro\meter} suspension containing \num{3e6} PC-3 cells.

\subsection{Image processing}
Image processing is performed on the acquired RF-data to enable image registration, background subtraction and counting of ACT\texttrademark{} microbubbles. The envelope of the RF-data is obtained from the absolute value of the IQ-modulated data. The RF-data are IQ-modulated through a Hilbert transform. A logarithmic compression is performed to reduce the dynamic range before image registration. The images are resized from $13568\times 256$ to $512\times 512$ through decimation and bicubic interpolation.  

\subsubsection{Image registration}
Motion correction is performed on all videos. All frames are aligned to a reference frame through an affine transformation. A regular step gradient descent optimization scheme has been utilized to determine the transformation matrix. The maximum number of iterations in the optimization scheme was set to 2000, using only one pyramid level. The reference frame was constructed from an average of the three first frames of the video used for background subtraction.

\subsubsection{Background subtraction}
Before background subtraction the data is linearized. For each processed video a background is computed from a set of frames. The set of frames is chosen individually to minimize motion artefacts through a heuristic approach. The background is filtered with a maximum filter of size $3\times 3$. The background is then subtracted from all frames to segment the signal caused by ACT\texttrademark{} or Sonazoid\texttrademark{} bubbles. Negative values after subtraction are set to 0. 

\subsubsection{Counting}
A region-of-interest (ROI) is defined prior to counting. The ROI is drawn to include most of the tumor, while avoiding skin and other healthy tissue. Only phase-shift bubbles within the ROI are counted. The counting of stuck phase-shift bubbles is based upon a temporal coherence filter to distinguish between stuck and free-flowing bubbles. A correlation matrix is constructed from the 40 previous frames, to form the base for this temporal coherence filter. A correlation matrix $d$ of two consecutive images, $A$ and $B$, is defined as
\begin{equation}
d_{ij} = 1-\frac{\abs{(A_{ij}-B_{ij})}}{A_{ij}+B_{ij}}\quad \forall i,j.
\end{equation}


%% INSERT FIGURE CORRELATION AND RUNNING AVERAGE

A running average of the correlation matrix is performed, and pixel indexes where the running average exceeds a temporal coherence threshold, $d_{min}$ are considered as stuck bubbles. The temporal coherence threshold is set to 0.85. This value is determined through a heuristic approach where the temporal coherence of manually identified bubbles is considered. A minimum intensity threshold is set to avoid counting of low-intensity noise. The minimum intensity threshold is set to 1000. It is important to emphasize that this algorithm counts the number of stuck phase-shift bubbles present, and not the total number of activated phase-shift bubbles.  

\subsection{Counting number of phase-shift bubbles in tumor}
For 16 animals, 125 data sets are processed in total, i.e. motion correction, background subtraction and automatic counting are performed. For each animal, one of the data sets images administration of ACT\texttrademark{} dilution. In these 16 videos, stuck phase-shift bubbles have been counted manually by Andrew Healey~\cite{Healey2014}. The manual counting is performed visually by careful identification of the phase-shift bubbles in the ultrasound videos. The results from manual and automatic counting are compared.

\subsubsection{Choosing representative number} 
For each data set, the maximal counted number density is the representative value when the automatically and manually results are compared. The first 20 frames are excluded because they may be considerably affected by the previous data set (Figure \ref{Fig:first_frames}). The reason for this is that the correlation matrix is continuously computed across consecutive data sets. The maximum value is chosen because it is closest to the total number of phase-shift bubbles activated throughout the video. This follows from counting algorithm which count number of present phase-shift bubbles. The number of bubbles present has to be lower or equal to the actual number of bubbles activated throughout the sequence.

\begin{figure}[h]
	\centering
	\includegraphics[width=\linewidth]{first_frames_corruption.png}
	\cprotect\caption{In the corresponding video we observe how the first 10-20 frames are affected by the correlation matrix from the previous video. The number of identified bubbles decreases quickly. The video is found at \path{F:\usb\avi\2014-05-03-09-56-48_count_and_color_1_to_100dilate_1_intensity_1000ct_0.85running_avg.avi}}
	\label{Fig:first_frames}
\end{figure}

\subsection{Qualitative validation}
\label{sec:qualitative}
A qualitative validation of the counting algorithm was performed using the following approach. The counted video sequences were evaluated with the following prejudices. In video sequences with only Sonazoid\texttrademark{} bubbles, no phase-shift bubbles should be counted. After the administration of the ACT\texttrademark{} dilution the phase-shift bubbles should stick and stay up to five minutes before decaying and disappearing. After the burst of high frequency ultrasound, all Sonazoid\texttrademark{} bubbles should be destroyed, but none of the phase-shift bubbles, i.e. the bubble count should not be affected by the high frequency ultrasound. The qualitative prejudices used to distinguish Sonazoid\texttrademark{} and phase-shift bubbles are stated in Table \ref{tab:qualitative}.


\begin{table}[H]
\caption{Prejudices used to distinguish Sonazoid\texttrademark{} and phase-shift bubbles.}
\label{tab:qualitative}
\begin{center}
\begin{tabular}{@{}l l l @{}}
  \toprule
  Property & Sonazoid\texttrademark{} & Phase-shift bubbles \\
  \midrule
  Non-linear imaging & Strong signal & Weak signal \\
  Linear imaging & Weak signal & Strong signal \\
  Kinetics & Free flowing & Stuck \\
  High power ultrasound & Destroyed &Unchanged\\
  \bottomrule
\end{tabular}
\end{center}
\end{table}


\subsection{Quantitative validation}
\subsubsection{Synthesized data set}
A quantitative validation of the algorithm was carried out by counting a synthesized data set with a known number density of activated phase-shift bubbles. Three different videos of administration of Sonazoid\texttrademark{} microbubbles were used as a background, and artificial phase-shift bubbles were added to create a data set with a known number density phase-shift bubbles. The number density ranged from \SIrange[per-mode=symbol]{0}{15}{bubbles\per\milli\meter\squared}. For each background, 27 data sets with increasing number density were constructed. The background images were acquired using linear contrast mode.

The algorithm adding the synthesized phase-shift bubbles to a background is described in the following paragraph. First, N random positions within the ROI is drawn from a uniform distribution. Then N intensities are drawn from a Gamma distribution (see Section \ref{PS intensity distribution}), and N frame numbers are drawn from a Poisson distribution to determine when the bubbles enter the tumor. N bubbles are then generated by applying the point spread function (PSF) to the drawn intensities. For each bubble the maximum intensity follows the slope of the bubble growth, shown in Figure \ref{Fig:growth slope}. These N bubbles are then log compressed and inserted into the video data. 


\begin{figure}[h]
  \centering
  \includegraphics[width=\linewidth]{fit of bubble growth.jpg}
  \caption{A growth slope is fit to data from an identified bubble in a true dataset.}
  \label{Fig:growth slope}
\end{figure}

\subsubsection{Phase-shift bubbles intensity distribution}
\label{PS intensity distribution}
A heuristic approach was used to estimate the intensity distribution for the activated phase-shift bubbles. In a set containing 63 identified phase-shift bubbles, the maximum intensity was plotted for each frame (Figure \ref{Fig:bubble_tic}). For each bubble the overall maximum value was used in a set, from which a gamma distribution was estimated (Figure \ref{Fig:gamma_fit}). 

The intensity distribution could probably been estimated through a more theoretical approach. However, the size distribution of activated emulsion droplets is not known accurately. The large phase-shift bubbles will also be deformed by the surroundings, and accurate backscattering and received signal from these bubbles are difficult to predict. A theoretical approach would therefore be based assumptions with unknown legitimacy. The heuristic approach was therefore considered to provide better results, and was chosen in favour. 


\begin{figure}[h]
  \centering
  \includegraphics[width=\linewidth]{bubble_tic.jpg}
  \caption{Time-intensity curves for a set of phase-shift bubbles.}
  \label{Fig:bubble_tic}
\end{figure}
\begin{figure}[h]
  \centering
  \includegraphics[width=\linewidth]{gamma_fit.jpg}
  \caption{The fit of a Gamma distribution to the distribution of maximum phase-shift bubble intensities.}
  \label{Fig:gamma_fit}
\end{figure}

\subsubsection{Measuring the point spread function}
The point spread function was measured from two different images. One image of a low solution of Sonazoid\texttrademark{} microbubbles in water, and another image of contamination in tap water. The size of the imaged microbubbles are smaller than the image system resolution. Thus, the bubbles visible in the image is effectively the PSF. A two dimensional, normalized Gaussian function was fit to a single, strong and easily identifiable bubble. 

% 
%\subsubsection{Measuring the scan plane height}
%The height of the ultrasound scan plan is necessary to calculate the number of phase-shift bubbles per volume. This height was measured by imaging a thin wire, running diagonally across the scan plane. The image is then a mapping of the string down on the x-axis. The height, H,  can the be calculated by measuring the length L, for a given angle $\Theta$.
%
%\begin{figure}[h]
%  \centering
%  \includegraphics[width=0.6\linewidth]{Transducer_scan_plane_height.pdf}
%  \caption{The measurement of transducer scan plane height}
%\end{figure}
%


\clearpage


%Results
\section{Results}
This chapter is divided in three parts. The first part deals with the qualitative results. This includes response to high power ultrasound, intensity growth curves, and phase-shift bubble behaviour. The second part provides a quantitative validation of the counting algorithm. This validation is based upon the randomly synthesized data set. The last part presents the results from the processing and counting of the real data set. 
 
\subsection{Qualitative results}
This section provides qualitative results, following the prejudices stated in Section \ref{sec:qualitative}.
\subsubsection{Effect of high power ultrasound}

The counted number of bubbles before and after two high ultrasound bursts in non-linear and linear imaging mode were available for 8 and 13 animals, respectively. Relative change after two flash pulses is shown in Figure \ref{High power US non-lin} and \ref{High power US lin}. Note that the high power ultrasound burst are extended (\SI{\sim 2}{\second}) in non-linear mode, while they are extremely short in linear mode (less than \SI{0.1}{\second}). Single bubble intensity curves subject to two short flashes in linear imaging mode are shown in Figure \ref{Fig:high_power_US}. 
%Sett inn bilde av aktiverte boble fra flash
\begin{figure}
\centering
\begin{minipage}[t]{.45\textwidth}
\centering
\includegraphics[width=.8\textwidth,height=4cm]{relative change after high power US non linear.png}
\caption{Relative change in the counted density before and after two high power ultrasound bursts in non-linear imaging mode. Note that 0\% indicates no change.}
\label{High power US non-lin}
\end{minipage}\hfill
\begin{minipage}[t]{.45\textwidth}
\centering
\includegraphics[width=.8\textwidth,height=4cm]{relative change after high power US linear.png}
\caption{Relative change in counted density before and after two high power ultrasound bursts in linear imaging mode. Note that 0\% indicates no change.}
\label{High power US lin}
\end{minipage}
\end{figure}

\begin{figure}[h]
  \centering
  \includegraphics[width=\linewidth]{time_intensity_high_powerUS.png}
  \caption{Time intensity curves for a few bubbles subject to a short high power ultrasound burs in linear imaging mode. The two bursts are marked, and some bubble are immediately activated.}
  \label{Fig:high_power_US}
\end{figure}
 

\subsubsection{Visibility in non-linear and linear imaging modes} 
%\subsection{bubble zoom}

\subsubsection{Kinetics}
Bubble kinetics are shown in Figure \ref{Fig:bubble_kinetic}. A specific event occur in the location marked by the yellow circle 36 seconds into the corresponding video. A phase-shift bubble is released from its current location and relocated. 

\begin{figure}[h]
  \centering
  \includegraphics[width=\linewidth]{bubble_kinetics_scr.png}
  \cprotect\caption{Corresponding video shows a stuck phase-shift bubble release from current location, travel a short distance and fasten in another location. The area is marked with yellow and the specific event occur at 36 seconds. Corresponding video is located at \path{F:\usb\avi\2014-05-01-11-23-04_count_and_color_1_to_1000dilate_1_intensity_1000ct_0.85running_avg.avi}}
  \label{Fig:bubble_kinetic}
\end{figure}

The phase-shift bubble dynamics are also visible in Figure \ref{bubble_zoom} and \ref{bubble_zoom_tic}, where we see inflow and activation of a single phase-shift bubble.

\begin{figure}
	\centering
	\begin{minipage}[t]{.45\textwidth}
		\centering
		\includegraphics[width=.8\textwidth,height=4cm]{bubble_zoom.png}
		\cprotect\caption{Inflow and activation of a single phase-shift bubble is visible in the corresponding movie \path{F:\usb\avi\bubble_zoom_b4_10_28_44.avi}.}
		\label{bubble_zoom}
	\end{minipage}\hfill
	\begin{minipage}[t]{.45\textwidth}
		\centering
		\includegraphics[width=.8\textwidth,height=4cm]{bubble_zoom_tic_4_10_28_44.png}
		\caption{The corresponding time intensity curve for the phase-shift bubble to the left.}
		\label{bubble_zoom_tic}
	\end{minipage}
\end{figure}

\clearpage
\subsection{Counting of synthesized data}
The synthesized data is based on three different background videos. The three data sets and respective non-linear best fit are found in Figure \ref{close comoarison}. The data is log-transformed and fitted to a third degree polynomial. The curve fitting is performed on the log-transformed data because of the multiplicative nature of speckle noise. A logarithmic transformation converts multiplication to addition, and makes the noise almost normally distributed (Figure \ref{Fig:log_normal}). The entire set, composed of all data from the three backgrounds, is presented in Figure \ref{Fig:counted_vs_inserted_all} and \ref{Fig:counted_vs_inserted_all_small}.
\begin{figure}[h]
  \centering
  \includegraphics[width=\linewidth]{close_comparison.png}
  \cprotect\caption{The three different data sets. The log-transformed data is fitted to a third degree polynomial. The corresponding background videos are found in the following directories. Background 1: \path{F:\usb\avi\2014-05-01-10-00-05_PS_counted_0.avi}. Background 2: \path{F:\usb\avi\2014-05-01-11-44-15_PS_counted_0.avi}. Background 3: \path{F:\usb\avi\2014-05-02-09-51-16_PS_counted_0.avi}.}
  \label{Fig:close comparison}
\end{figure}
%\begin{figure}[h]
%  \centering
%  \includegraphics[width=\linewidth]{2014-05-01-11-44-15counted_vs_inserted.png}
%  \caption{Non-linear regression on the first synthesized data set, file reference 2014-05-01-11-44-15.}
%  \label{Fig:counted_vs_inserted2}
%\end{figure}
%\begin{figure}[h]
%  \centering
%  \includegraphics[width=\linewidth]{2014-05-01-10-00-05counted_vs_inserted.png}
%  \caption{Non-linear regression on the first synthesized data set, file reference 2014-05-01-10-00-05.}
%  \label{Fig:counted_vs_inserted3}
%\end{figure}

\begin{figure}[h]
  \centering
  \includegraphics[width=\linewidth]{counted_vs_inserted.png}
  \caption{Plot and non-linear fit of the counted number density as a function of inserted density. Note that the fit is performed on the log-transformed data. This make the residuals of the fit almost normally distributed (\ref{Fig:log_normal}).}
  \label{Fig:counted_vs_inserted_all}
\end{figure}

\begin{figure}[h]
	\centering
	\includegraphics[width=\linewidth]{log_normal.png}
	\caption{The Normal probability plot for the residuals of the counted data and the fit in Figure \ref{Fig:counted_vs_inserted_all}. We see that the residuals are almost normally distributed. Remember that the fit is performed on log-transformed data.}
	\label{Fig:log_normal}
\end{figure}

\begin{figure}[h]
  \centering
  \includegraphics[width=\linewidth]{counted_vs_inserted_small.png}
  \caption{Plot and non-linear fit of the counted number density as a function of inserted density. Same plot as in Figure \ref{Fig:counted_vs_inserted_all}, but zoomed onto the range from 0 to 4 bubbles per square millimetre.}
  \label{Fig:counted_vs_inserted_all_small}
\end{figure}

If we swap the axes, we can plot the actual number density as a function of the counted number density. If we only consider the near-linear range, we can fit a linear curve to the log-transformed data (Figure \ref{Fig:counted_vs_inserted_inverse}). The spread of the counted number density has been evaluated (Figure \ref{Fig:rsd}). The average relative standard deviation is 0.016.

\begin{figure}[h]
  \centering
  \includegraphics[width=\linewidth]{counted_vs_inserted_small_inverse.png}
  \caption{Plot and linear fit of real number density as a function of counted density.}
  \label{Fig:counted_vs_inserted_inverse}
\end{figure}

\begin{figure}[h]
	\centering
	\includegraphics[width=\linewidth]{rsd_inserted.png}
	\caption{Relative standard deviation as a function of inserted number density of phase-shift bubbles. Linear fit to log-transformed data. The mean relative standard deviation is 0.016.}
	\label{Fig:rsd}
\end{figure}


\subsection{Counting of real data}
Counted number density for the 16 animals are presented together with the manually counted results (Figure \ref{Fig:Number of counted bubbles} and \ref{Fig:Number density of counted bubbles}). The bubble count representing each animal is the maximum count during the video of ACT\texttrademark{} injection. Results for all 125 video sequences are found in Appendix (\ref{raw counting}). Image of counting during ACT\texttrademark{} administration is found in Figure \ref{Fig:counting_administration}. The contrast agent is coloured in green. The blue boundary defines the region of interest, and counting is only performed within this area. Areas which are identified as stuck phase-shift bubbles are circumscribed with red.

\begin{figure}[h]
  \centering
  \includegraphics[width=\linewidth]{counting_scrshot_10_40_23.png}
  \cprotect\caption{Counting during administration of ACT\texttrademark{}. The corresponding video is found at \path{F:\usb\avi\2014-05-02-10-40-23_count_and_color_1_to_1000dilate_1_intensity_1000ct_0.85running_avg.avi}. All contrast agent is coloured in green. The blue boundary defines the region of interest, and counting is only performed within this area. Areas which are identified as stuck phase-shift bubbles are circumscribed with red.}
  \label{Fig:counting_administration}
\end{figure}

There may be small differences between the ROI used for the automatic and manually counted data. This affects the number of counted bubbles, N.

\begin{figure}[h]
  \centering
  \includegraphics[width=\linewidth]{manual_vs_auto_count.png}
  \caption{Number of counted bubbles in the 16 mice. Note that there may be differences in the ROI used for the manual(blue) and automatic(red) counting.}
  \label{Fig:Number of counted bubbles}
\end{figure}

\begin{figure}[h]
  \centering
  \includegraphics[width=\linewidth]{manual_vs_auto_count_density.png}
  \caption{Number density of counted, stuck phase-shift bubbles in the 16 mice. Automatic counting is shown in red, while manual counting is shown in blue.}
  \label{Fig:Number density of counted bubbles}
\end{figure}

For all 16 animals, time-intensity curves showing the integrated intensity within the ROI as a function of time is plotted. The count number density is also plotted with respect to time. For one animal, this is presented in Figure \ref{Fig:tic} and \ref{Fig:tic_count}. Similar results for all animals are found in Appendix \ref{tic appendix} and \ref{App:number_density_curves}.


\begin{figure}[h]
  \centering
  \includegraphics[width=\linewidth]{tic_10.jpg}
  \caption{Time-intensity curve in animal nr. 10. These numbers represent both non-linear(red) and linear(black) contrast imaging.}
  \label{Fig:tic}
\end{figure}

\begin{figure}[h]
  \centering
  \includegraphics[width=\linewidth]{bubble_count5.jpg}
  \caption{Number density as a function of time in animal nr. 5. These numbers represent both non-linear(red) and linear(black) contrast imaging.}
  \label{Fig:tic_count}
\end{figure}

\clearpage
%Table with andys count vs my count




%Discussion
\section{Comparison to existing program(VisualSonics)}
\section{Comparison to Andy's manual counting}

%conclusion
%Short and to the point
%No discussion
%Further work

\section{Conclusion}
A program for motion correction, background subtraction and counting of phase-shift bubbles is successfully developed, validated and applied to a data set containing 125 ultrasound image sequences of 16 different prostate tumor xenografts. The program has been validated qualitatively against prejudices regarding the phase-shift bubble behaviour. Our results have been compared to the results from manual counting, performed by Andrew Healey~\cite{2014}. There is a very good correlation between the results obtained from automatic and manual counting. This support the credibility of the program.

A synthesized data set of in total 81 videos, based on three different backgrounds, was created. For each video, a chosen number of synthesized phase-shift bubbles was inserted at a random time and position, and with a random intensity. The random values were drawn from appropriate distributions based on the real data set. This provided a quantitative validation of the program performance. For number densities below \SI{\sim2}{bubbles\per\milli\meter\squared}, the relation between the inserted and the counted number density is close to linear (Figure \ref{Fig:counted_vs_inserted_inverse}). Above \SI{\sim2}{bubbles\per\milli\meter\squared} the program experience a saturation, and the accuracy decreases as the number density increases. 

The output of the developed program fulfil the project requirements, and include a video (.avi-file) displaying the counted phase-shift bubbles and a data file (.mat-file) containing number and number density of counted phase-shift bubbles. The quality of the video is better than existing software, in terms of motion correction and visualization of phase-shift bubbles.


\subsection{Further work}
If this program is to be applied on data with number density of phase-shift bubbles above \SI{2}{bubbles\per\milli\meter\squared}, effort should be put into reducing the experienced saturation. This saturation occurs because single bubbles are situated too close to each other, and therefore recognized as one large bubble. It should be possible to distinguish these separate bubbles by counting the number of local maxima within each bubble. This was tested without satisfying result, due to the inherent variations and noise present in ultrasound images. A more robust method may be achieved by smoothing the shape of the identified bubbles, before the number of local maxima is counted. Another option is to increasing the imaging frequency, and thereby the spatial resolution. Another obvious improvement of the program is to speed up the motion correction.
	
This program is based on linear contrast images. Some non-linear data were processed, but the model validation is based solely on linear contrast images. Based on our results, the algorithm seems to work equally well on non-linear and linear contrast images. But, in order to accurately determine the performance on non-linear images, a validation based on this imaging mode is necessary.

The high power ultrasound bursts turned out to produce unexpected results. More research and examination of single bubble intensity curves is necessary to understand the physics behind the large increase in counted phase-shift bubbles in non-linear imaging mode. 




% Bibliografi/referanseliste skal komme før appendiks
\bibliography{C:/Users/Snorre/Documents/bibtex/library}
\bibliographystyle{plainnat}

% En latex-kommando for å si fra at kapitlene/seksjonene fra nå 
% av skal nummereres med store bokstaver:
\appendix
%\begin{appendices}

\section{Derivation of the Rayleigh-Plesset equation}
\label{App:R-P}
The Rayleigh-Plesset equation is an ordinary differential equation which describe the non-linear oscillation of a gas bubble suspended in an infinite liquid, subject to an external sound wave. This equation is in the following derived using the energy balance between the liquid and the gas bubble ~\cite{Moss2014}. The equation can also be derived from the Navier-Stokes equations~\cite{leighton2007derivation}.

A few assumptions are required for the following derivation to be valid. We assume the wavelength of the pressure field to be way larger then the size of the gas bubble, i.e. $d \ll \lambda$. The bubble is spherical and spatially uniform conditions within the bubble exist at all times. We can neglect gravity and bulk viscosity. There is no flow of either matter of heat through the boundary of the bubble. The density of the gas is significantly smaller then the liquid density, and the gas within the bubble can be considered an ideal gas. Using these assumptions the oscillation will be an adiabatic process.  Newton's notation for the time-derivative is used.

Consider an oscillating bubble suspended in incompressible fluid. Because of the incompressibility of the fluid, the fluid velocity $u(r,t)$ has to follow the inverse square law, i.e. 

\begin{equation}
\label{eq:1}
u(r,t) = \frac{R(t)^2}{r(t)^2}\dot{R}(t).
\end{equation}
Here $r_b(t)$ is the bubble radius, $\dot{r}_b(t)$ the velocity of the boundary and $r$ any radius larger or equal $r_b$. 

The kinetic energy $E_k$ of the fluid caused by the oscillating bubble is then
\begin{equation}
\label{kinetic energy}
E_k = \frac{\rho_l}{2}\int_{r_b}^\infty u(r,t)4\pi r^2 \dif r = 2\pi\rho_l r_b^3 \dot{r}_b^2,
\end{equation}
where $\rho_l$ is the liquid density.

Far from the bubble the liquid pressure is given by $p_{\infty}(t) = p_0 + p(t)$, where $p(t) = p_a e^{i\omega t}$ is the time varying pressure caused by the sound wave and $p_0$ the hydrostatic pressure. For an adiabatic process we have that $pV^{\gamma}=\mathrm{constant}$. Here $V$ is the volume, $p$ the pressure and $\gamma$ the adiabatic index. The pressure is then only a function the bubble radius $r_b$,
\begin{equation}
p(r_b) = p_{r_0}\left(\frac{V_{r_0}}{V(r_b)}\right)^{\gamma} =  p_{r_0}\left(\frac{r_0}{r_b}\right)^{3\gamma},
\end{equation}

where $r_0$ and $p_{r_0}$ are the equilibrium radius and pressure. 

The work done to expand the bubble is only carried out by the net pressure, $\Delta p = P(r_b)-P_{\infty}(t)$, and the total work $W$ is

\begin{equation}
\label{bubble work}
W = \int \Delta p dV = \int _{r_0}^{r_b} (p(r_b)-p_{\infty(t)})4\pi r_b^2 \dif r_b.
\end{equation}

The kinetic energy of the liquid must equal the work, and the Rayleigh-Plesset equation is obtained by equating and differentiating Equation \eqref{kinetic energy} and \eqref{bubble work} with respect to $r_b$, 

\begin{equation}
\label{RPE}
\frac{p_{r_0}\left(\frac{r_0}{r_b}\right)^{3\gamma}-p_0 - p(t)}{\rho_l} = \frac{3\dot{r}_b^2}{2}+r_b\ddot{r}_b.
\end{equation}

Note that $\frac{d\dot{r}_b^2}{dr-b} = 2\ddot{r_b}$. 
\clearpage

%	\label{Matlab files}
%\end{table}

%\section{Raw counting results}
The raw counting results are stored in Excel-sheets found in \verb|F:\usb\excel\|. The first document (counting sheet.xlsx) contains the result and information about the real data set. The second document (counting sheet synthesized data.xlsx) contains the result and information about the synthesized data set.

%\section{Matlab files}

The program developed in this work is written in Matlab. All written Matlab files and a short description is given in the table below (\ref{longtable}). Note that the key files are written in bold text. The actual Matlab files are found in \path{F:\usb\mfiles\}. A simple description on how to use the program is given in the following paragraph.

The main file in this program is \textit{run\_program.m}. The first input is the full file name of the raw ultrasound data (e.g. \path{D:\Ultrasound image data\2014-04-28-10-23-41.rf}). The second input argument is an array of frame numbers from which the background should be computed (e.g. \verb|1:10|). The third  and fourth argument are true/false statements. The decide if counting or background subtraction is performed, respectively. The fifth and sixth argument may be omitted, if the background file is based on frames within the video given in argument one. The sixth argument is the full file name, of a .mat where the correlation matrix for the previous video is stored (e.g.\path{C:\Phaseshiftcounting\2014-05-02-10-44-08\_PS\_count.mat}). This allow the counting of phase-shift bubble to continue from where the last video ended. The \textit{run\_program.m} perform motion correction, background subtraction and counting, in the respective order. The computation time will depend on the computer performance and the size of the raw ultrasound file, but between one and ten hours per file must be expected. Note that this script load and save several files to directories available on my computer. Directories and file names in the Matlab files may therefore be corrected for the program to work properly. the  An example on how to run through a large set of ultrasound image data is given in \textit{batch\_process.m} 

	\begin{center}
		\begin{longtable}{@{} p{3cm} p{3cm} p{2cm} p{4cm} @{}}
			\caption{List of Matlab files. The most important files are written in bold text.}\label{longtable}\\
			\toprule
			Name & Input & Output & Function\\  
			\hline \endhead
			\textbf{align\_image.m} & fixed image, moving image, transformation type,  max iterations, initial transform, max step length) &  & Align the moving image to the fixed image using MATLAB functions imregtform and imwarp. \\ \hline 
			all\_counted\_vs\_inserted.m & & & Perform non-linear fit of synthesized data \\ \hline
			\textbf{batch\_process.m} &  &  & This script perform batch processing of the rf-data specified in \path{counting sheet.xlsx}. For all file names, the function run\_program is launched. \\ \hline 
			bubble\_count\_curve.m &  & .mat files & Produce .mat files containing data later used for plotting of Number density as a function of time  \\ \hline 
			bubble\_density.m & Number of bubbles, region of interest, Bmode parameters & number density & Compute number density from a given number of bubbles and a ROI. \\ \hline 
			bubble\_growth.m &  &  & Estimate bubble growth curve from existing single bubble tic curve \\ \hline 
			bubble\_tic.m &  & .mat files & Produce .mat files with data later used to plot tic-curves \\ \hline 
			bubble\_zoom.m & frames, chosen pixels, show images(true/false) & maximum intensity & Zoom in on a region defined by the chosen pixels, and produce a smoothed close up video of these pixels. \\ \hline 
			\textbf{color\_code.m} & frame, contrast mask & RGB image & Color the contrast green in the .avi files. \\ \hline 
			compare\_manual\_and\_auto\_count.m &  & Figures & Make histograms comparing manually and automatically counted data. \\ \hline 
			counted\_vs\_inserted.m & & & Compare results from the three different backgrounds\\ \hline
			count\_max\_real\_data.m &  & .txt file & Count the maximum counted number of phase-shift bubble for each video, and store in .txt file. \\ \hline 
			\textbf{count\_PS\_bubbles.m} & motion corrected file name, subtracted file name & .avi file and .mat file & Count the number of phase-shift bubbles and make the final .avi file. \\ \hline 
			\textbf{do\_subtraction.m} & frames & subtracted frames and background frame & Compute background and subtract the backgrounf from all frames. \\ \hline 
			draw\_roi.m & frame & ROI & Allow the user to draw a ROI on a frame. \\ \hline 
			evalc2decimal.m & string & decimal number & Convert a string to a decimal number \\ \hline 
			frame\_counter.m & Full file name of raw US data & Total number of frames in file & Count the number ov video frames in a raw file (.rf/.iq) \\ \hline 
			\textbf{get\_background.m} & Frames & background & Compute background by maximum projection of the given frames \\ \hline 
			get\_correlation.m & frame A, frame B, & correlation & Compute correlation between two frames \\ \hline 
			get\_growth\_intensity.m & max intensity, start frame, length & intensity array & Multiply the intensity obtained with growth\_fun.m to obtain correct intensity function \\ \hline 
			get\_param.m & file name raw US data & image parameter & Obtain image parameters \\ \hline 
			\textbf{get\_RF\_from\_IQ.m} & Full file name of raw US data, frame number & RF data, parameters & Obtain RF-data from IQ data.  \\ \hline 
			get\_tic.m & file\_reference, t & intensity array & Compute time intensity curve for a file given by the file reference.  t(seconds) is a time array.  \\ \hline 
			growth\_fun.m & frame numbers & growth function & Calculate the bubble growth function for the given frame numbers \\ \hline 
			hms2sec.m & d1,h1, m1, s1, d2, h2, m2, s2 & seconds & Compute seconds elapsed between two timestamps.  \\ \hline 
			im\_sub.m & frame A, frame B, type & Difference frame & Subtract frame B from A, and set all values less than zero equal to zero. \\ \hline 
			insert\_bubble.m & Intensity, x-position, y-position, frame & frame with bubble & Insert a synthesized bubble at a given position and and intensity given in the given frame \\ \hline 
			investigate\_pixels.m &  & Figures & Plot the intensity as a function of frame for a given pixel. \\ \hline 
			log\_compress\_2.m & array, 'compress/decompress' & array & Approximation to log-compression(envelope) in log\_compress.m Can compress or decompress. \\ \hline 
			log\_compress.m & RF-data & compressed data & Perform hilbert transform and log  compression on RF- data \\ \hline 
			make\_intensity\_distribution.m &  & probability object & Compute probability density function for the intensity distribution. \\ \hline 
			make\_ref\_frame.m & Full filename, frame indexes, extension(rf/iq) & reference frame and parameters & Make reference frame for motion correction. \\ \hline 
			mat2avi.m & filename & .avi file & Contruct .avi file from .mat file \\ \hline 
			\textbf{motion\_correction.m} & Filename(mat file) & motion corrected .mat file & Compute motion corrected .mat file from a .mat file \\ \hline 
			mse.m & frame A, frame B & Mean square error & Compute mean square error between two frames. \\ \hline 
			plot\_bubble\_count\_curve.m &  & Figures & Plot bubble count curves from .mat files produced with bubble\_count\_curve.m \\ \hline 
			plot\_bubble\_tic.m &  & Figures & Plot tic curves from the .mat files produced with bubble\_tic.m \\ \hline 
			plot\_tic\_curves.m & Counted .mat file name, subtracted .mat file name & time array, intensity curves & Calculate intensity curves for single bubbles and show close-up video of bubbles. \\ \hline 
			point\_spread\_fun.m &  &  & Calculate the PSF from an identified bubble. \\ \hline 
			progressbar.m & ratio & Figure & Show progress bar.  Public available script. \\ \hline 
			psf.m &  &  & return the PSF \\ \hline 
			random\_intensity.m & intensity distribution & intensity & Draw a random intensity from the intensity distribution. \\ \hline 
			random\_position.m & Roi, N & x and y coordinate & Draw N random positions within ROI. \\ \hline 
			ReadRF.m & full filename, mode name, and frame number & RF data, parameters & Read in RF data. Written by A.Healey. \\ \hline 
			\textbf{RF2mat.m} & Full filename & .mat file & Convert raw US data to .mat file. \\ \hline 
			\textbf{run\_program.m} & File name raw US data, background frame index, count(true/false), bg\_subtraction(true/false),  background file name, correlation array file name. & .avi file and . Mat file & Perform all processing from raw US data to counted .avi file and .mat file. If count or background\_subtraction is false, the program will not perform these tasks. Background file name must be included if the background is supposed to be made from another file. Correlation array file name is the name of the previous video sequence. \\ \hline 
			\textbf{subtraction\_fun.m} & File name motion corrected .mat file, indexes for background & .mat file(subtracted data) & Create background file, and subtract the background from all frames. Save subtracted frames as .mat file \\ \hline 
			time\_array\_from\_ time\_stamps.m & time stamp 1, time stamp 2. & time array(seconds) & Make time array from two time stamps \\ \hline 
			VsiBModeIQ.m & Full file name, mode name, and frame number & IQ data and parameters & Compute IQ data from RF-data. Written by A. Needles, J. Mehi. Copyright VisualSonics 1999-2010 \\

		\end{longtable}
		
	\end{center}
	\clearpage
%\end{appendices}

% Indeks for rapporten. Ta bort prosenttegn hvis du vil ha det med.
%\printindex

% Avslutter dokumentet vårt:
\end{document}

% Local Variables:
% TeX-master: "master"
% End: