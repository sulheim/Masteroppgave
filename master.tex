% Først spesifiserer vi hvilken dokumentklasse vi vil ha og noen 
% globale opsjoner. Bytt ut 'article' med 'book' hvis du vil ha med kapitler.
\documentclass[b5paper, twoside, titlepage, 10pt]{article}

% Så sier vi fra om hvilke tilleggspakker vi trenger
% til dokumentet vårt. De som du ikke trenger (se kommentaren) 
% kan det være en fordel å kommentere ut (sett prose
%%%%REGEX FINN litenbokstav etter punktum : \.\s[:lower:]
\usepackage[T1]{fontenc}
\usepackage{cite}
\usepackage[english]{babel}
\usepackage[lmargin=25mm,rmargin=25mm,tmargin=27mm,bmargin=30mm]{geometry}                
\usepackage{amsmath,amsfonts,amssymb} % matematikksymboler
%\usepackage{amsthm}                   % for å lage teoremer og lignende.
\usepackage{graphicx}                 % inkludering av grafikk
\graphicspath{{figurer/}{C:/Users/Snorre/Documents/masteroppgave/Masteroppgave/Phaseshiftcounting/tic/}{C:/Users/Snorre/Documents/masteroppgave/Masteroppgave/Phaseshiftcounting/bubble_count_curves/}}
\usepackage{subcaption}                   % hvis du vil kunne ha flere
                                      % figurer inni en figur
\usepackage[space]{grffile}
\usepackage{siunitx}
\sisetup{range-units = single}
\usepackage{fancyvrb}
\sisetup{separate-uncertainty}%
\DeclareMathOperator{\sgn}{sgn}
\usepackage{morefloats}
\usepackage[numbers]{natbib}
\usepackage[titletoc,toc,title]{appendix}
%\usepackage{underscore}

\usepackage{longtable}
\usepackage{multirow}
\usepackage{mathtools}
\usepackage{units}%for nicefrac
\usepackage{textcomp}%For trademark
\usepackage{commath}%for derivatives
\usepackage{booktabs}
\usepackage{cprotect}
\usepackage{placeins}
\renewcommand{\arraystretch}{1.2} 
%
%\DeclarePairedDelimiter\abs{\lvert}{\rvert}%
%\DeclarePairedDelimiter\norm{\lVert}{\rVert}%

\usepackage{listings}                 % Fin for inkludering av kildekode
\usepackage[hyphens, obeyspaces]{url}
\Urlmuskip=0mu plus 1mu


%\usepackage{hyperref}                % Lager hyperlinker i evt. pdf-dokument
                                      % men har noen bugs, så den er kommentert
                                      % bort her.
                                 
% Indeksgenerering er kommentert ut her. Ta bort prosenttegnene
% hvis du vil ha en indeks:
%\usepackage{makeidx}     
%\makeindex              

%\includeonly{appendix}
% Selve dokumentet begynner:

\begin{document}

% På forsida skal vi ikke ha noen sidenummerering:

\pagestyle{empty}
\pagenumbering{roman}

% Inkluder forsida:
% Enkel forside som bruker latex sin \titlepage kommando:
% NB: Bruken av \and mellom navn!
\titlepage
\title{ Method development and automated analysis of ultrasound images of phase-shift bubbles.}
\author{Snorre Sulheim}
\date{\today}
\maketitle
%\section{Acknowledgements}
I would like to thank:
\begin{enumerate}
	\item Andrew Healey for essential supervising and enthusiasm throughout the project.
	\item Annemieke van Wamel for giving me helpful supervising and providing me with the necessary data. 
%	\item Catharina de Lange Davies 
	\item Friends and family for support and proofreading.
\end{enumerate}
%abstract
%Summary of Background/motivation for work
%Material and methods
%results
%Conclusion
%Approx 1 page.
%No references
\section{Abstract}
Ultrasound mediated drug delivery is an important tool in the fight against cancer. A new concept known  as ACT\texttrademark{} is under development, and two pilot imaging studies have been performed on prostate cancer xenografts in mice.  A large amount of raw ultrasound data has been recorded, but existing software can not perform the required image processing. The ACT\texttrademark{} concept is based on clusters of microbubbles and microdroplets that experience a phase-shift from liquid to gas when exposed to ultrasound. The phase-shift increases the bubble size and, make they are caught in the vasculature of the tumor. 

A complete program has been developed in Matlab\textsuperscript{\textregistered} to process the raw ultrasound data. The program is tailored to the unique properties of the phase-shift bubbles, and is able to reduce noise and motion artefacts, to visualize the contrast agent, and to count the number of ultrasound activated phase-shift bubbles. The program produces high quality videos, displaying both free flowing contrast agent and identified, stuck phase-shift bubbles. 

The program showed a very good correlation to a manually counted data set. The program was validated against a synthesized data set, and we found that the program counted accurately up to \SI{\sim2}{bubbles\per\milli\meter}. A saturation was experienced above this threshold, and too few bubbles were counted.  


\section{Sammendrag}
%%

% Romerske tall på alt før selve rapporten starter er pent.
\pagenumbering{roman}

% For å ikke begynne innholdslista på baksida av forsida:
\cleardoublepage
% (kun aktuelt når man har twoside som global opsjon)

% Nå vi vil ha noe i topp- og bunnteksten
\pagestyle{headings}

% Si til LaTeX at vi vil ha ei innholdsliste generert akkurat her:
\tableofcontents

% Pass på at neste side ikke begynner på baksida av en annen side.
\cleardoublepage

% Arabisk (vanlige tall) sidenummerering. Starter på side 1 igjen.
\pagenumbering{arabic}

% Inkluder alle de andre kildefilene:

% NB: Vi trenger ikke ta med filendelsen .tex her. Den vet
%     LaTeX om selv!
\section{Introduction}
%Skriv om annen forskning innen samme område. ref mail fra andy 30.okt. 
%Skriv noe om forskning ved NTNU
\subsection{History}
Ultrasound is sound with frequency above the upper limit of human hearing, considered to be at 20 kHz, and has a wide range of use. The first work on ultrasound related to spatial orientation was written in 1794 by Lazaro Spallanzini, after discovering bats ability to navigate, only using ultrasound. Yet, almost hundred years passed before Jacques and Pierre Curie discovered the piezoelectric effect and Sir Francis Galton invented a machine able to produce ultrasound at 40 kHz, both during 1880. The piezoelectric effect is the ability of some crystals to generate electrical charge, when subjected to mechanical stress.

During the beginning of the 20th century the echo-locator was invented, and the first application was detecting submarines during World War 1. Use of ultrasound in medical imaging, also known as sonography, was first used in 1956 when Ian Donald measured the parietal diameter of a fetal head. Seven years later commercial sonography devices were available.

The last decades there has been continuous development, and from the simple display modes used in the beginning, there is now possible to get real-time imaging in both two and three dimensions, in and Doppler imaging enables continuous measurement and visualization of blood movement in blood vessels and tissues.

\subsection{} 
   
%Teori.tex
\section{Theory}
\subsection{A start}
The basis of ultrasound imaging is the reflection of ultrasound at tissue boundaries within the body. The ultrasound is generated by a transducer, and as the wave travels through the body an echo is generated by partial reflection at every boundary. The amount of reflection depends on the difference in acoustic impedance. The echo is recorded by the transducer, and it is then displayed in the image according its spatial origin. The speed of sound is approximately \SI{1540}{\metre\per\second} for all soft tissue in the body, and it is thus easy to calculate the origin by measuring the time of travel of the echo.  The brightness in the image is proportional to the strength of the echo, and this is thus called B-mode(brightness mode). 



\subsection{Contrast agents}
Contrast agents are used to increase the image sensitivity, by increasing image contrast between the relevant body tissues. In ultrasound imaging, this is accomplished by injecting a solution containing gas-filled microbubbles. 

The first contrast agents were made by saline, and was put to use by cardiologists in the 1960s for identification of mitral valve echoes. The saline was shaken before injection to create the microbubbles. Current available contrast agents are more complex, and consist of a carefully chosen gas enclosed in a suited shell. The shell has to be biocompatible and is made from fat or proteins. The advantage of a shell is increased lifetime and scattering of ultrasound. The size of the microbubbles is approximately equal to the size of red blood cells (\SIrange{2}{6}{\micro\metre}), and flow easily through the circulatory system. 

The microbubbles have two important properties which cause large scattering. The first reason is the big difference in acoustic impedance between microbubbles and human tissue. The second property is the compressability of the gas inside the microbubble. The pressure fluctuations caused by the ultrasound forces the microbubble to expand during rarefaction and contract during compression. The microbubble will oscillate with the same frequency as the ultrasound source, and the scattering will be strongest at the resonance frequency of the microbubble. The resonant frequency is determined by the properties and the size of the shell. 

\subsection{Our Bubble}
\subsection{Transducer}

\subsection{Image processing of B-mode images}
After the echoes have reached the transducer a signal is produced by making an image with the brightness at each pixel determined by the strength of the echo from that corresponding distance and direction. The first step in the image processing is to convert the signal from analogue to digital. The digital signal is less vulnerable to noise and distortion, and it enables further digital image processing. Then a linear amplifier apply the same amount of gain to the entire signal, to make the signal strong enough for further processing. Time-gain compensation is then applied to make echoes from similar interfaces equal, regardless of the depth of their origin. This is performed by increasing the gain with increasing depth of echo. The depth of the echo is identified by the arrival time at the transducer. The rate of attenuation of ultrasound with depth is determined by the frequency and tissue.

After amplification and time-gain compensation the dynamic range of the signal is about 60 dB. The dynamic range of a signal is defined as the ratio between the largest amplitude that can be recorded without causing distortion and the lowest amplitude that can be distinguished from noise. The dynamic range of a common screen is about 20 dB. The signal must therefore be compressed before it can be displayed. To compress the dynamic range from 60 to 20 dB, an amplifier with non-linear gain is applied. Low amplitudes are amplified more than high, and the dynamic range is therefore decreased. Compression allows weak echoes from scattering within tissue to be displayed together with strong echoes from tissue interfaces.

\subsubsection{RF and IQ data}
RF is short term for radio frequency data which is used in ultrasound as a description for unprocessed data. IQ is short term for in quadrature, and refers to a demodulation of the RF signal to reduce the amount of storage space without loss of information. IQ modulation converts the signal from the real to the imaginary space. The IQ signal is obtained using a IQ-demodulator to down-mix, low-pass filter and decimate the RF signal. The IQ signal can also be computed by through a Hilbert transform\cite{Kirkhorn1999}.

\subsubsection{Hilbert transform}
The Hilbert transform is a linear operator which acts on a signal $u(t)$ to derive an analytic signal. The Hilbert transform convert the signal from real to complex space by adding or subtracting 90 degrees. It is therefore also known as a phase-shift operator. An analytic signal has by definition only positive frequencies in its Fourier transform, and is related to the Hilbert transform through 

\begin{equation}
\tilde{x}(t) = x(t) + x_h(t),
\end{equation}

where $x(t)$ is the signal, $x_h(t)$ the Hilbert transform of the signal, and $\tilde{x}(t)$ the analytic signal. The Hilbert transform can be written as a convolution, 

\begin{equation}
x_h(t) = x(t)*\frac{1}{\pi t},
\end{equation}

which can be interpreted as a filtering operation with a quadrature filter which shifts all sinusoidal components by a phase shift of $\frac{\pi}{2}$. The envelope is the amplitude of the analytic signal, and a B-mode image is created from the envelope of the signal. 

\subsubsection{Downsampling}
Because it is the envelope that is used to create the B-mode images, further reduction in data size can be achieved by downsampling of the RF signal. The RF signal is downsampled by a decimation factor $M$, by only recording every $M$th sample. 

The RF signal is a strictly bandlimited signal, limited by the bandwidth of the transducer. The envelope is thus also a bandlimited signal, with a finite maximum frequency. According to the Nyquist-Shannon sampling theorem, this signal can be sampled without aliasing or loss of information by a sampling rate twice the maximum frequency. Hence, the decimation factor, $M$, is determined so that

\begin{equation}
\label{deciamtion}
\fraction{Fs_{RF}}{M} > 2f^{max}_{envelope},
\end{equation}

where ${Fs_{RF}$ is the sampling frequency of the RF signal, and $f^{max}_{envelope}$ is the maximum frequency of the envelope. For a thorough explanation, see \cite{Crochiere1981}.



\subsection{Harmonic Imaging}

\subsection{Matlab}
The most important to write here is how the methods used in matlab work IN THEORY, not how they are implemented in MATLAB.


\subsubsection{Removing image artifacts}
The operation of removing movement artifacts are based on the Matlab toolbox Image Processing and the use of image registration. \textit{imregister} and \textit{imregconfig} are the first two functions that have been applied and tested. This is intensity based automatic registration. Control point registration may be another option.

DETTE ER KANSKJE IKKE VIKTIG
The process is initiated by making a \textit{metric} and an \textit{optimizer} object using the \textit{imregconfig} function. The \textit{metric} object measures the similarity of the two images. The \textit{optimizer} contains the optimization parameters such as maximum number of iterations, initial step length, optimization algorithm etc. Which optimization algorithm and how the image similarity is measured can be chosen in the \textit{imregconfig} function. The options for optimization algorithm are either a regular step gradient or a one-plus-one evolutionary method. For the metric object the similarity can be measured either by a mean square error approach or by making a mutual information metric. The mutual information metric maximizes the number of pixel with the same relative pixel value, and is best suited for images with different brightness ranges.[REF MATLAB]

\subsection{Speckle}
Speckle is a random, deterministic speckle pattern present in all types of coherent imaging, thus also in ultrasound images. The speckle is formed by scatterers smaller than the resolution of the imaging system, and the shape and size of the speckle pattern is determined by the dimensions of the imaging system and the structure of the imaged tissue.

The speckle is an interference phenomenon created by coherent waves with different phase and amplitude added together. If several echo waves arrive at the same piezoelectric element within a time span shorter than the emitted pulse, the piezoelectric element will not be able to distinguish the waves, and their impact will be added. 

\subsection{Image registration}
Image registration is the process where one image is spatially aligned to a reference image. The image to be aligned is called the moving image, while the reference image is called the fixed image. Image registration  is an important part of image processing and can both be used to remove motion artifacts or to fuse images of the same object, captured with different image modalities or from different directions. 

Image registration can initially be divided into extrinsic and intrinsic methods. Extrinsic methods are based on foreign objects placed into the scene before the image is captured. This has the advantage of simple, feature-based registration, but the preplacement and removal of these objects may not be trivial. 

Intrinsic methods can be divided into feature and intensity based registration. Features can either be easily recognizable points identified by the user, or structures which can be extracted from the image by image segmentation. These methods are mostly used in rigid transformations, and have the advantage of being simple computations once the features are determined. One drawback is that the registration often is limited by the reliability of the segmentation or identification of the features. Although these methods are applicable to both multi- and monomodal registration, and to different body parts, their use have in general been limited to neuroimaging and orthopedic imaging\cite{Maintz1998}.

Intensity based registration differs from the other methods by using the intensity pixel values directly to compare the fixed and moving image. To compare the images a suited similarity measure is used, see section\ref{subsec:similarity}. Using different similarity measures, this method is suitable for both multi- and monomodal registration. The image registration is then performed using an optimization routine to find the spatial translation of the moving image which minimizes the chosen similarity measure. Choosing the right similarity measure and optimization scheme is essential to get a satisfying result. For a full review of this topic, see \cite{Maintz1998}.

\subsubsection{Basic theory of image registration}
\cite{Mainstream}
%Should non-parametric transformations be mentioned?

%Transformation of an image is the basis for image registration, and can be described as a function $y$ mapping the image coordinates from $\Real_d \arrow \Real_d$ using a linear combination of basis functions and coefficients.

\subsubsection{Image transformations}

Transformation of an image is the basis for image registration, and can be described as a mapping of a coordinate vector $\vec{x}$ from the space $X$ to a new coordinate vector $\vec{y}$ in the space $Y$. The transformation is performed by a transformation matrix $A$, i.e. $\vec{y} = A\vec{x}$.

In 2D a rigid transformation can be written as 

\begin{equation}
	\label{rigid}
	\begin{pmatrix}
		y_1 \\
		y_2 \\
		1 
	\end{pmatrix}
	=
	\begin{pmatrix}
		R_{11} & R_{12} & T_1\\
		R_{21} & R_{22} & T_2\\		
		0 & 0 & 1
	\end{pmatrix}
	\begin{pmatrix}
		x_1\\
		x_2\\
		1
	\end{pmatrix},
\end{equation}
where $R_{ij}$ are elements in the rotation matrix 

\begin{equation}
	R = 
	\begin{pmatrix}
	\cos \theta & -\sin \theta\\
	\sin \theta & \cos \theta
	\end{pmatrix}.
\end{equation}
The rotation matrix rotates the coordinates an angle $\theta$ around the origo. The matrix elements $T_{ij}$ determines the translation of the coordinates. An affine translation is described by the matrix 
\begin{equation}
\begin{pmatrix}
 a_{11}&a_{12}&a_{13}\\
 a
 
\end{pmatrix}
\end{equation} This enables shear and scaling of the image.

If we apply a transformation matrix to an image, we get the new pixel coordinates of transformed image, but these points will not be on the grid coordinates of the image. For that reason, an interpolation scheme is applied to the transformed image coordinates to get the new pixel values at the grid coordinates. A suitable interpolation scheme must be chosen according to the given problem, but the most common are described in section ??????

Image transformation can be divided into linear and non-linear transformation. For a linear transformation the same transformation matrix is applied to the whole image, whereas this is not the case for a non-linear transformation. Linear transformation is computationally faster and simpler, and less affected by noise. On the other hand, non-linear transformations can correct for local deformations out of reach for the linear transformation. 

The difficult part of image registration is not to apply the transformation, but to obtain the right transformation matrix. The simplest case is the point or feature based method, where easily recognized feature are localized in both the fixed and moving image, either by interaction from the user, or by using a feature detection algorithm. In the case of an affine transformation, six pair of corresponding coordinates are needed to solve the set of linear equations to obtain the the six unknown elements in the transformation matrix. Usually, more points are obtained, and the transformation matrix is calculated using a least-squares approach. 

\subsubsection{Similarity measure}
\label{subsec:similarity}
There are several ways of measuring the similarity between two images, where different approaches enhance different image properties, and are suitable for different problems. The choice of similarity measure will determine the minimum and the rate of convergence for the optimization scheme.

When the images to compare are from the same modality, they will be in the same intensity range. They will only differ because of noise, geometric transformation and changes in imaged object. Common similarity measures are then the sum of squared differences (SSD), the sum of absolute differences or the cross correlation. If the changes in the imaged object is sufficiently small, and we assume the noise to be Gaussian, it is shown in \cite{Viola1997} that SSD give the optimal result.
For two images $A$ and $B$ the SSD is given as
\begin{equation}
\label{SSD}
\mathrm{SSD} = \frac{1}{N}\sum_{i}^N \abs{A_i -B_i}^2, \forall i \in A \cap B,   
\end{equation}

where $i$ is an image pixel. 

In multimodal image registration, the images have neither similar intensities or even a linear relationship between the intensities. For this type of problems, mutual information is the most common similarity measure. Mutual information is a measure of the statistical dependence of the two images, and the alignment is optimal the moving image contains the maximal amount of information about the fixed image. For a detailed review of mutual information in medical multimodal imaging, see \cite{563664}.  

\subsection{Regular step gradient descent}
An optimization scheme determines the optimal alignment by optimizing the chosen similarity measure. The optimization scheme is usually chosen through an empirical approach, where computational demand, reliability and stability are important factors. 

The gradient descent method is a first-order algorithm which finds the local minimum of a multivarible, differentiable function $F(x)$. From a given initial state $x_k$, the optimizer moves a distance $\gamma$ in the direction opposite to the gradient, i.e.

\begin{equation}
\label{gradient descent}
x_{k+1} = x_k - \gamma \Delta F(x_k).
\end{equation}

This method suffers from the reliability of the step distance $\gamma$. Too long or short step will give a slow, or no convergence. The \italic{regular step gradient descent} is a variation where the step length is halved every time there is a significant change in the direction of the gradient. The optimization terminates after reaching a the minimum, at a minimum step length, or after a maximum number of iterations.

To speed up the image registration, a pyramidal method can be applied together with the gradient descent. A smoothing filter is applied to the images, before they are decimated by a factor 2. This is performed for a given number of pyramid levels, resulting in a set of images with decreasing size. The scheme starts by optimizing the translation for the smallest image, and the optimal translation is used as an initial translation in the next pyramid level. In addition the increased speed, there is also less chance of getting stuck in a local minimum due to the smoothing which is applied before each decimation.
 









%metode
%Materials
%Procedures
%Instrumentation
\section{Image recording}
\subsection{Sonazoid\texttrademark}
Sonazoid\texttrademark is a contrast agent which has overcome all requirements stated in Section \ref{Contrast agents}, with a structure and size distribution as seen in Figure \ref{Fig:Sonazoid}. There is a core of perfluorcarbon gas, with a \SI{4}{\nano\meter} thick lipid monolayer shell with shear modulus and viscosity equal \SI{50(3)}{\mega\pascal} and \SI{0.8(1)}{\newton\second\per\meter\squared}, respectively\cite{Hoff2000}. The density of the perfluorcarbon gas core is \SI{0.0098}{\gram\per\centi\meter\squared}, while the thermal diffusivity is \SI{0.07}{\centi\meter\squared\per\second}\cite{Healey2012}. 

\begin{figure}[h]
  \centering
  \label{Fig:Sonazoid}
  \includegraphics[width=0.8\linewidth]{./figurer/sonazoid.png}
  \caption{Left: The structure of Sonazoid\texttrademark , i.e. perfluorcarbon gas encapsulated in a lipid monolayer. Right: Size distribution of Sonazoid\texttrademark and red blood cells\cite{Healey2012}.}
\end{figure}

Sonazoid\texttrademark microbubbles has been used together with microdroplets to form the ACT\texttrademark clusters, or by itself in the test data sets. The microbubbles are shipped freeze-dried, and reconstituted with sterile water to crate a solution with a gas volume content of about 1\%. %Anyhting more here?

\subsection{Imaging setup}
%16Mhz
All ultrasound images used in this research has been imaged with the Vevo\textregistered 2100 imaging system from Visual Sonics. The imaging settings are presented in Table

\subsubsection{Imaging modes}
%Non-linear contrast????
Two different imaging modes have been used, linear and non-linear contrast mode. Linear contrast mode has been used to visualize the ACT bubbles, while non-linear mode B-mode is used when only Sonazoid\texttrademark is administered. 

 
\subsection{The animals/cancer}
\subsection{Counting 16 animals}
For 16 animals, similar series of images have been captured, and phase-shift bubbles have been counted manually\cite{Healey2014}. The same video sequences have been counted automatically to compare results. 

\section{Image processing}
Image processing is performed on the acquired RF-data to enable image registration, background subtraction and counting of ACT microbubbles. The envelope of the RF-data from the absolute value of the IQ-modulated data. The RF-data are IQ-modulated through a Hilbert transform. A logarithmic compression is performed to reduce the dynamic range before image registration. The images are resized from $13568x256$ to $512x512$ through decimation and bicubic interpolation.  

\subsection{Image registration}
Motion correction is performed on all videos. All frames are aligned to a reference frame through an affine transformation. A regular step gradient descent optimization scheme has been utilized to determine the transformation matrix. The maximum number of iterations in the optimization scheme was set to 2000, using only 1 pyramid level. The reference frame was constructed from an average of the three first frames of the video used for background subtraction.

\subsection{Background subtraction}
Before background subtraction the data is linearized. For each processed video a background is computed from a set of frames. The set of frames is chosen individually to minimize motion artifacts through a heuristic approach. The background is filtered with a maximum filter of size $3\times 3$. The background is then subtracted from all frames to segment the signal caused by ACT\texttrademark or Sonazoid\texttrademark bubbles. Negative values after subtraction are set to 0. 

\subsection{Counting}
The counting of ACT\texttrademark bubbles are based on a temporal coherence filter to distinguish between stuck and free-flowing bubbles. A correlation matrix $d$ of two consecutive images ,$A$ and $B$, is defined as
\begin{equation}
d_{ij} = 1-\frac{\abs(A_{ij}-B_{ij})}{A_{ij}+B_{ij}} \forall i,j .
\end{equation}

A running average over correlation matrix for the last 40 frames is performed and indexes where the running average exceeds a temporal coherence threshold, $d_{min}$ are considered stuck bubbles. The temporal coherence threshold is set to 0.85. This value is determined through a heuristic approach where the temporal coherence of manually identified bubbles are considered. A minimum intensity threshold is set to avoid counting of low-intensity noise. The minimum intensity threshold is set to 1000.

\section{Qualitative validation}
A qualitative validation of the counting algorithm was performed using the following approach. The counted video sequences was evaluated with the following prejudices. In video sequences with only Sonazoid\texttrademark bubbles, no phase-shift bubbles should be counted. After the introduction of phase-shift bubbles the phase-shift bubbles should stick, and stay for up to five minutes before decaying and disappearing. After the burst of high-frequency ultrasound, all Sonazoid\texttrademark bubbles should be destroyed, but none of the phase-shift bubble, i.e. the bubble count should not be affected by the high-frequency ultrasound. 


\section{Quantitative validation}
\subsection{Synthesized data set}
A quantitative validation of the algorithm has been carried through by counting a synthesized data set with a known density of activated phase-shift bubbles. Video sequences of administration of Sonazoid\texttrademark microbubbles were used as a background, and artificial phase-shift bubbles were added to the background to create a data set with a known number of phase-shift bubbles.

The algorithm for adding the phase-shift bubbles are now described. First, N random positions within the ROI is drawn from a uniform distribution. Then N intensities are drawn from a Gamma distribution, and N frame numbers are drawn from a Poisson distribution to determine when the bubbles enter the tumour. N bubbles are then generatedby applying the PSF to the drawn intensities. For each bubble the maximum intensity follows the slope of the bubble growth, shown in Figure \ref{fig:bubble growth}. These N bubbles are then log compressed and inserted into the video data. 


\begin{figure}[h]
  \centering
  \label{Fig:Sonazoid2}
  \includegraphics[width=0.8\linewidth]{./figurer/fit of bubble growth.jpg}
  \caption{A growth slope is fit to data from an identified bubble in a true dataset.}
\end{figure}

%To get an unmasked estimation, we want to apply the validation directly onto the RF-data. Therefore, the transformation matrix from the image registration should be stored and applied to RF-data. Define a region of interest by drawing the boundary of the tumour. Then two uniformly distributed numbers should be drawn, to choose a random position within the ROI. Then draw an ???Intensity from intensity distribution??? and multiply with the point spread function. Add the bubble, and repeat the process until N bubbles have been applied. Apply the same bubbles to every RF-frame, and then run the bubble counting and compare with the number of bubbles applied. N should be $\#/area$, i.e. the density if bubbles, and there should be a varying density to investigate how the counting perform with increasing number of bubbles.

\subsection{Phase-shift bubbles intensity distribution}
A heuristic approach was used to estimate the intensity distribution for the activated phase-shift bubbles. For a set containing 63 identified phase-shift bubbles the maximum intensity was plotted for each frame, see Figure \ref{Fig:bubble_tic}. For each bubble the overall maximum value was used in a set, from which a gamma distribution was estimated, see Figure \ref{Fig:gamma_fit}.

\begin{figure}[h]
  \centering
  \label{Fig:bubble_tic}
  \includegraphics[width=0.8\linewidth]{./figurer/bubble_tic.jpg}
  \caption{Time-intensity curves for a set of phase-shift bubbles.}
\end{figure}
\begin{figure}[h]
  \centering
  \label{Fig:gamma_fit}
  \includegraphics[width=0.8\linewidth]{./figurer/gamma_fit.jpg}
  \caption{Fit of a Gamma distribution to the distribution of maximum phase-shift bubble intensities.}
\end{figure}

\subsection*{Measuring the point spread function}
The point spread function was measured from two different images, image a low solution of Sonazoid\texttrademark microbubbles in water and image of contamination in tap water. The size of the imaged microbubbles are smaller than the resolution of the imaging system, so the the bubbles seen in the image is effectively the point spread function(PSF). A two dimensional, normalized Gaussian function were fit to a single, easily identifiable bubble. 

 
\subsection*{Measuring the scan plane height}
The height of the ultrasound scan plan is necessary for the estimate of the number of phase-shift bubbles per volume. This height was measured by imaging a thin wire, running diagonally across the scan plane. The image is then a mapping of the string down on the x-axis. The height, H,  can the be calculated by measuring the length L, for a given angle $\Theta$.

\begin{figure}[h]
  \centering
  \includegraphics[width=0.8\linewidth]{Transducer_scan_plane_height.pdf}
  \caption{Measurement of transducer scan plane height}
\end{figure}



%Results
\section{Results}
This chapter is divided in three parts. The first part deal with qualitative results. This include response to high power ultrasound, intensity growth curves, and phase-shift bubble behaviour. The second part provides a quantitative validation of the counting algorithm. This validation is based on the randomly synthesized data set. The last part present the results from processing and counting of the real data set. 
 
\subsection{Qualitative results}
\subsection{Counting of Sonazoid\texttrademark microbubbles}
\subsection{Effect of high power ultrasound}
Counted number of bubbles before and after 2 high ultrasound bursts are shown in Table(SETT INN TABLE).
%Sett inn bilde av aktiverte boble fra flash
 

\subsection{Visibility in non-linear and linear imaging modes}
\subsection{bubble_zoom}
\subsection{Kinetics}
Video of bubbles flowing and getting stuck is shown in Video ????.
\section{Counting of synthesized data}
%Figures of insterted bubbles vs counted bubbles


\section{Counting of 16 animals}
Counted number of bubbles for the 16 animals is presented together with the manually counted result in Figure ? and ?. The bubble count representing each animal is the highest, reliable count during the video of administration of compound. A full list of counted bubbles, and frame number for counting is presented in Appendix (EXCEL ark).
There may be small differences between the ROI used for the automatic and manually counted data. This can effect the number of counted bubbles, N.

\begin{figure}[h]
  \centering
  \label{Fig:Number of counted bubbles}
  \includegraphics[width=0.8\linewidth]{Number of counted bubbles.jpg}
  \caption{Number of counted bubbles of the 16 mice. Note that there may be differences in the ROI used for the manual(blue) and automatic(red) counting.}
\end{figure}

\begin{figure}[h]
  \centering
  \label{Fig:Number density of counted bubbles}
  \includegraphics[width=0.8\linewidth]{Number density in data set.jpg}
  \caption{Number density of counted, stuck phase-shift bubbles of the 16 mice. Automatic counting is shown in red, while manual counting in blue.}
\end{figure}

For all 16 animals, time-intensity curves showing the integrated intensity within the ROI as a function of time is plotted. The count number density is also plotted with respect to time. For animal 1, this is presented in Figure 1 and 2. For all other animals, this is found in Appendix?????.



%Table with andys count vs my count




%Discussion
%Discuss methods then results
\section{Discussion}
The developed and validated algorithm is a tool that might increase the understanding of the activation, administration and \texttrademark{} of the ACT\texttrademark{} bubbles. Compared to manual counting this program give increased efficiency, and a result with a known accuracy and precision. It is also shown that the chosen method of motion correction, background subtraction and stuck bubble identification provides satisfying results. Although this program is tailored to the ACT\texttrademark{} bubbles, these methods may apply to similar problems.
%\subsection{Why existing algorithm are not applicable}
%Several methods for distinguish stuck microbubbles from free flowing microbubbles and tissue were described in section \ref{existing algorithms}. The size of the ACT\texttrademark{} phase-shift bubbles distinguish them from other microbubbles, and allow a new approach for segmentation. The method described in ~\cite{Rychak2006} has its obvious drawback of waiting for free bubbles to clear. During this time most of the ACT\texttrademark{} will disappear as well. The first method described in ~\cite{Zhao2007} assume that no bubbles are stuck before the first image. 
\subsection{Performance of motion correction}
The motion correction applied in this work, uses a linear affine transformation to correct for motion in image. This can not correct for local deformations in the tissue or movements out of the image plane. Hence, some videos are not free of motion. A tumor is in general quite rigid, and there is thus rarely local deformations within the tumor region. Non-linear (local) image registration is therefore unnecessary. Linear transformation has the benefit of being faster and more robust than non-linear transformations. The current motion correction program is slow ( one to eight hours per video sequence).  The number of pyramidal steps and iterations can be more closely evaluated to increase the speed, but this was not prioritized in this work. 

\subsection{Qualitative validation}
In Section \ref{sec:qualitative} there are several results which support the claim that the counted bubbles are phase-shift bubbles. In Figure \ref{Fig:bubble_kinetic} and the corresponding video, the phase-shift bubble dynamics are visible. A phase-shift bubble is getting stuck, releases, and attaches to another location. The adhesion and activation is also clearly visible in Figure \ref{bubble_zoom}. The activation and dynamics imply that the identified bubble is a phase-shift bubble. The number density curves as a function of time (Figure \ref{Fig:tic_count}) display how the phase-shift bubbles stay for several minutes. This may increase the drug retention time by fully or partially blocking the vasculature.

\subsubsection{Effect of high power ultrasound}
The high power ultrasound has clearly an effect on the counted number of bubbles (Figure \ref{High power US non-lin} and \ref{High power US lin}). In general, there is not a large decrease. This implies that most of the identified bubbles are not Sonazoid\texttrademark{}. As mentioned earlier, they are disrupted by high power ultrasound. The high ultrasound burst is short (\SI{\sim 0.1}{\second}) in linear imaging mode, and long (\SI{\sim 2}{\second}) in non-linear mode.

In linear imaging mode (Figure \ref{High power US lin}) there is a large spread in the relative change. The reasons are not fully understood, but there are a few possibilities. Some bubbles are obviously activated (\ref{Fig:high_power_US}). The decrease observed in three of the data sets may come from destroyed Sonazoid\texttrademark{}, which may have been stuck in vascular dead ends or inflammation areas~\cite{Healey_pc}. The radiation force from the high-power ultrasound may be able to push bubbles in or out of the image plane, and this may also cause a change in number of counted bubbles. In non-linear imaging mode (Figure \ref{High power US non-lin}), there is a large increase in two thirds of the data sets. This reason for this is not known.

\subsection{Synthesized data set for validation}
As there exist no 'Gold standard' for counting of phase-shift bubbles, a synthesized data set was needed to get a quantitative evaluation of the performance of the counting algorithm. By knowing the number of synthesized phase-shift bubbles we could determine the accuracy and precision for different number densities.

The synthesized data set was constructed to imitate an administration of ACT\texttrademark{} and Sonazoid\texttrademark{} microbubbles. To replicate the inflow, time of appearance, maximum intensity and position were chosen randomly from distributions generated from real data. In the real data some phase-shift bubbles are getting stuck after activation, or are released from the their location before the size is reduced. This is not accounted for in the synthesized data. 

The synthesized data set was based upon three different background videos, and there is clearly difference in terms of counting performance (Figure \ref{Fig:close comparison}). We observe a lower number density of counted bubbles in Background 2, compared to Background 1 and 3. In the corresponding videos we find that this background has less visible flow of contrast agent. The increased flow of contrast agent in Background 1 and 3 create variations in the signal intensity. This may help in differentiating adjacent stuck bubbles from each other. Three different backgrounds are hardly enough to make a proper validation. The number of backgrounds were limited by the available data and the long processing  time. 

\subsection{Performance of counting algorithm}

We observe a good correlation between the automatically and manually counted number density of phase-shift bubbles (Figure \ref{Fig:Number density of counted bubbles}). This supports the credibility of the developed algorithm.

The performance of the counting algorithm for the full synthesized data set was shown in Figure \ref{Fig:counted_vs_inserted_all}. The counting algorithm experiences a saturation when the inserted density passes \SI{\sim 2}{bubbles\per\milli\meter\squared} (Figure \ref{Fig:counted_vs_inserted_inverse}). This is an expected behaviour. When the density of bubbles increases, the chance of two or more bubbles being adjacent increases as well, and they may be counted as one large bubble (Figure \ref{Fig:saturation}). Although this is a drawback, it is important to note that most ($\sim 86\%$) of the counted number densities in the real data set is below \SI{2}{bubbles\per\milli\meter} (Figure \ref{Fig:number density distribution}). Hence, the error from saturation in the real data is limited.


\begin{figure}[h]
	\centering
	\includegraphics[width=\linewidth]{saturation.png}
	 \cprotect\caption{Saturation is clearly seen in as large clusters of bubbles recognized as one bubble. The corresponding movie is found at  \path{F:\usb\avi\2014-05-01-11-44-15_PS_counted_258.avi}.}
	\label{Fig:saturation}
\end{figure}

\begin{figure}[h]
	\centering
	\includegraphics[width=\linewidth]{number_density_distribution.png}
	\caption{The number density distribution of phase-shift bubbles in the real data set. We observe that $\sim 86\%$ of the counted densities is below \SI{2}{bubbles\per\milli\meter}.}
	\label{Fig:number density distribution}
\end{figure} 


If we only consider the lower end of the inserted density (Figure \ref{Fig:counted_vs_inserted_all_small}), there is almost a linear relationship between the counted and inserted number of bubbles. By inverting the axes and fitting a linear curve to the log-transformed data, we obtain an estimate of the accuracy and precision of the counting algorithm (Figure\ref{Fig:counted_vs_inserted_inverse}). We find that the algorithm is slightly under-counting. The relative standard deviation of the counting algorithm is fairly constant, with a mean of 0.016 (Figure \ref{Fig:rsd}).

\subsection{Comparison to existing program (Visual Sonics)}
In Figure \ref{Fig:compare VisualSonics} we present a comparison of the Visual Sonics and our program's performance in terms of motion correction and background subtraction. An improved background subtraction makes the single phase-shift bubbles more visible, and the noise and motion artefacts are strongly reduced.

\begin{figure}[h]
	\centering
	\begin{minipage}[b]{0.42\textwidth}
		\centering
		\includegraphics[width=\textwidth]{vevo_10_40_53.png}\\
		(a)
	\end{minipage}%
	\begin{minipage}[b]{0.35\textwidth}
		\centering
		\includegraphics[width=\textwidth]{10_40_53_snorre.png}\\
		(b)
	\end{minipage}%
	 \cprotect\caption{The same video sequence processed with the Visual Sonics (a) and our program (b). The improved background subtraction increases the visibility of the individual bubbles, and there is a reduction in motion artefacts. The corresponding movies are found at \path{F:\usb\avi\#01 2014-04-28-10-40-53.avi} and \path{F:\usb\avi\2014-04-28-10-40-53_count_and_color_11_to_1000dilate_1_intensity_1000ct_0.85running_avg.avi}.}
	\label{Fig:compare VisualSonics}
\end{figure}

\subsection{Non-linear imaging}
The counting algorithm was tailored to linear contrast images. This imaging mode is suppose to enhance the phase-shift bubbles compared to Sonazoid\texttrademark{} microbubbles and tissue. From the time-intensity curves (Appendix \ref{tic appendix}) we find that the total contrast is slightly larger in linear contrast mode, compared to non-linear contrast mode. On the other hand, the number density curves (APPENDIX ?/Figure?) display no difference between the two imaging modes, in terms of the counting performance. Hence, it is not possible to make a statement regarding choice of imaging mode based on our results.

We suspect that the raw ultrasound data from the non-linear mode contains some pulse encoding. Information about the encoding is not publicly available. We are therefore not able to fully reproduce the non-linear images produced by the Visual Sonics software. This complicates the use of the non-linear data.  


%\subsection{Findings in a broader context}


%The focal plane was measured to \SI{400}{\micro\meter}.If the total volume of the tumor is estimated, it is possible to compute the total number of deposited ACT\texttrademark{} bubbles. Assuming a linear relationship between total amount of drug delivered and number of stuck bubbles, it is possible to make a relative estimate of the drug delivered. This can be compared to results from ongoing treatment studies.

%
%Relate to amount of bubbles introduced.
%
%Relate to concentration for each animal.


%conclusion
%Short and to the point
%No discussion
%Further work

\section{Conclusion}
A program for motion correction, background subtraction and counting of phase-shift bubbles is successfully developed, validated and applied to a data set containing 125 ultrasound image sequences of 16 different prostate tumor xenografts. The program has been validated qualitatively against prejudices regarding the phase-shift bubble behaviour. Our results have been compared to the results from manual counting, performed by Andrew Healey~\cite{2014}. There is a very good correlation between the results obtained from automatic and manual counting. This support the credibility of the program.

A synthesized data set of in total 81 videos, based on three different backgrounds, was created. For each video, a chosen number of synthesized phase-shift bubbles was inserted at a random time and position, and with a random intensity. The random values were drawn from appropriate distributions based on the real data set. This provided a quantitative validation of the program performance. For number densities below \SI{\sim2}{bubbles\per\milli\meter\squared}, the relation between the inserted and the counted number density is close to linear (Figure \ref{Fig:counted_vs_inserted_inverse}). Above \SI{\sim2}{bubbles\per\milli\meter\squared} the program experience a saturation, and the accuracy decreases as the number density increases. 

The output of the developed program fulfil the project requirements, and include a video (.avi-file) displaying the counted phase-shift bubbles and a data file (.mat-file) containing number and number density of counted phase-shift bubbles. The quality of the video is better than existing software, in terms of motion correction and visualization of phase-shift bubbles.


\subsection{Further work}
If this program is to be applied on data with number density of phase-shift bubbles above \SI{2}{bubbles\per\milli\meter\squared}, effort should be put into reducing the experienced saturation. This saturation occurs because single bubbles are situated too close to each other, and therefore recognized as one large bubble. It should be possible to distinguish these separate bubbles by counting the number of local maxima within each bubble. This was tested without satisfying result, due to the inherent variations and noise present in ultrasound images. A more robust method may be achieved by smoothing the shape of the identified bubbles, before the number of local maxima is counted. Another option is to increasing the imaging frequency, and thereby the spatial resolution. Another obvious improvement of the program is to speed up the motion correction.
	
This program is based on linear contrast images. Some non-linear data were processed, but the model validation is based solely on linear contrast images. Based on our results, the algorithm seems to work equally well on non-linear and linear contrast images. But, in order to accurately determine the performance on non-linear images, a validation based on this imaging mode is necessary.

The high power ultrasound bursts turned out to produce unexpected results. More research and examination of single bubble intensity curves is necessary to understand the physics behind the large increase in counted phase-shift bubbles in non-linear imaging mode. 




% Bibliografi/referanseliste skal komme før appendiks
\bibliography{C:/Users/Snorre/Documents/bibtex/library}
\bibliographystyle{unsrtnat}

% En latex-kommando for å si fra at kapitlene/seksjonene fra nå 
% av skal nummereres med store bokstaver:
\appendix
%\begin{appendices}

\section{Derivation of the Rayleigh-Plesset equation}
The Rayleigh-Plesset equation is an ordinary differential equation which describe the non-linear oscillation of a gas bubble suspended in an infinite liquid, subject to an external sound wave. This equation is in the following derived using the energy balance between the liquid and the gas bubble \cite{Moss2014}. The equation can also be derived from the Navier-Stokes equations\cite{leighton2007derivation}.

A few assumptions are required for the following derivation to be valid. We assume the wavelength of the pressure field to be way larger then the size of the gas bubble, i.e. $d \ll \lambda$. The bubble is spherical and spatially uniform conditions within the bubble exist at all times. We can neglect gravity and bulk viscosity. There is no flow of either matter of heat through the boundary of the bubble. The density of the gas is significantly smaller then the liquid density, and the gas within the bubble can be considered an ideal gas. Using these assumptions the oscillation will be an adiabatic process.  Newton's notation for the time-derivative is used.

Consider an oscillating bubble suspended in incompressible fluid. Because of the incompressibility of the fluid, the fluid velocity $u(r,t)$ has to follow the inverse square law, i.e. 

\begin{equation}
\label{eq:1}
u(r,t) = \frac{R(t)^2}{r(t)^2}\dot{R}(t).
\end{equation}
Here $r_b(t)$ is the bubble radius, $\dot{r}_b(t)$ the velocity of the boundary and $r$ any radius larger or equal $r_b$. 

The kinetic energy $E_k$ of the fluid caused by the oscillating bubble is then
\begin{equation}
\label{kinetic energy}
E_k = \frac{\rho_l}{2}\int_{r_b}^\infty u(r,t)4\pi r^2 dr = 2\pi\rho_l r_b^3 \dot{r}_b^2,
\end{equation}
where $\rho_l$ is the liquid density.

Far from the bubble the liquid pressure is given by $p_{\infty}(t) = p_0 + p(t)$, where $p(t) = p_a e^{i\omega t}$ is the time varying pressure caused by the sound wave and $p_0$ the hydrostatic pressure. For an adiabatic process we have that $pV^{\gamma}=\mathrm{constant}$. Here $V$ is the volume, $p$ the pressure and $\gamma$ the adiabatic index. The pressure is then only a function the bubble radius $r_b$,
\begin{equation}
p(r_b) = p_{r_0}\left(\frac{V_{r_0}}{V(r_b)}\right)^{\gamma} =  p_{r_0}\left(\frac{r_0}{r_b}\right)^{3\gamma},
\end{equation}

where $r_0$ and $p_{r_0}$ are the equilibrium radius and pressure. 

The work done to expand the bubble is only carried out by the net pressure, $\Delta p = P(r_b)-P_{\infty}(t)$, and the total work $W$ is

\begin{equation}
\label{bubble work}
W = \int \Delta p dV = \int _{r_0}^{r_b} (p(r_b)-p_{\infty(t)})4\pi r_b^2 dr_b.
\end{equation}

The kinetic energy of the liquid must equal the work, and the Rayleigh-Plesset equation is obtained by equating and differentiating Equation \eqref{kinetic energy} and \eqref{bubble work} with respect to $r_b$, 

\begin{equation}
\label{RPE}
\frac{p_{r_0}\left(\frac{r_0}{r_b}\right)^{3\gamma}-p_0 - p(t)}{\rho_l} = \frac{3\dot{r}_b^2}{2}+r_b\ddot{r}_b.
\end{equation}

Note that $\frac{d\dot{r}_b^2}{dr-b} = 2\ddot{r_b}$. 
\clearpage
\section{Raw counting results}
\label{raw counting}
The raw counting results are stored in Excel-sheets found in \path{F:\usb\excel\}. The first document (\path{counting sheet.xlsx}) contains the result and information about the real data set. The second document (\path{counting sheet synthesized data.xlsx}) contains the result and information about the synthesized data set.
	


\begin{center}
	\begin{longtable}{@{}l l l l@{}}
		\caption{Counted raw ultrasound data.}
		\label{counting sheet}
		\toprule
		File id & Animal nr & Density of bubbles (automatic) & Density of bubbles (manual) \\ 
		\midrule \endhead
		2014-11-04-15-29-07 & 0 &  &  \\ 
		2014-11-04-15-31-39 & 0 &  &  \\ 
		2014-11-04-15-36-35 & 0 &  &  \\ 
		2014-11-04-15-38-42 & 0 &  &  \\ 
		2014-11-04-15-40-50 & 0 &  &  \\ 
		2014-04-28-10-37-22.rf & 1 & 0.00 &  \\ 
		2014-04-28-10-40-53.rf & 1 & 1.54 & 0.75 \\ 
		2014-04-28-10-44-00.rf & 1 & 1.54 &  \\ 
		2014-04-28-10-45-33.iq & 1 & 0.73 &  \\ 
		2014-04-28-10-47-00.rf & 1 & 1.22 &  \\ 
		2014-04-28-10-47-31.rf & 1 & 0.57 &  \\ 
		2014-04-28-10-54-52.iq & 1 & 0.41 &  \\ 
		2014-04-29-09-43-36.iq & 2 & 0.00 &  \\ 
		2014-04-29-09-45-50.iq & 2 & 0.96 & 1.35 \\ 
		2014-04-29-09-48-08.iq & 2 & 0.73 &  \\ 
		2014-04-29-09-48-38.iq & 2 & 0.63 &  \\ 
		2014-04-29-09-48-53.rf & 2 & 0.60 &  \\ 
		2014-04-29-09-49-36.iq & 2 & 0.70 &  \\ 
		2014-04-29-09-50-22.rf & 2 & 0.17 &  \\ 
		2014-04-29-09-51-05.iq & 2 & 0.63 &  \\ 
		2014-04-29-09-52-06.iq & 2 & 0.17 &  \\ 
		2014-04-29-09-55-24.rf & 2 & 0.13 &  \\ 
		2014-04-29-09-56-56.rf & 2 & 0.20 &  \\ 
		2014-04-29-09-58-15.rf & 2 & 0.23 &  \\ 
		2014-04-30-08-45-53.iq & 3 & 0.00 &  \\ 
		2014-04-30-08-48-40.iq & 3 & 0.52 & 0.36 \\ 
		2014-04-30-08-51-24.iq & 3 & 0.31 &  \\ 
		2014-04-30-08-52-05.rf & 3 & 0.26 &  \\ 
		2014-04-30-08-52-39.iq & 3 & 0.52 &  \\ 
		2014-04-30-08-53-11.rf & 3 & 0.21 &  \\ 
		2014-04-30-08-54-43.rf & 3 & 0.16 &  \\ 
		2014-04-30-08-55-20.iq & 3 & 0.37 &  \\ 
		2014-04-30-08-58-37.rf & 3 & 0.42 &  \\ 
		2014-04-30-09-00-03.rf & 3 & 0.21 &  \\ 
		2014-04-30-09-02-46.rf & 3 & 0.42 &  \\ 
		2014-04-30-09-03-48.rf & 3 & 0.31 &  \\ 
		2014-05-04-11-37-15.iq & 4 & 0.00 &  \\ 
		2014-05-04-11-38-18.iq & 4 & 0.96 & 0.5 \\ 
		2014-05-04-11-40-41.iq & 4 & 0.72 &  \\ 
		2014-05-04-11-41-03.rf & 4 & 1.20 &  \\ 
		2014-05-04-11-41-36.iq & 4 & 0.72 &  \\ 
		2014-05-04-11-42-00.rf & 4 & 0.72 &  \\ 
		2014-05-04-11-42-32.iq & 4 & 0.24 &  \\ 
		2014-05-04-11-43-11.rf & 4 & 0.96 &  \\ 
		2014-05-04-11-44-01.rf & 4 & 0.96 &  \\ 
		2014-05-04-11-49-10.rf & 4 & 0.48 &  \\ 
		2014-05-02-10-39-41.iq & 5 & 0.00 &  \\ 
		2014-05-02-10-40-23.iq & 5 & 2.16 & 1.625 \\ 
		2014-05-02-10-42-36.iq & 5 & 1.44 &  \\ 
		2014-05-02-10-43-07.rf & 5 & 2.94 &  \\ 
		2014-05-02-10-43-44.iq & 5 & 1.22 &  \\ 
		2014-05-02-10-44-08.rf & 5 & 2.55 &  \\ 
		2014-05-02-10-44-41.iq & 5 & 1.16 &  \\ 
		2014-05-03-09-52-43.iq & 6 & 0.00 &  \\ 
		2014-05-03-09-53-31.iq & 6 & 1.96 & 1.6 \\ 
		2014-05-03-09-55-51.iq & 6 & 1.13 &  \\ 
		2014-05-03-09-56-15.rf & 6 & 2.66 &  \\ 
		2014-05-03-09-56-48.iq & 6 & 1.05 &  \\ 
		2014-05-03-09-57-12.rf & 6 & 2.53 &  \\ 
		2014-05-03-09-57-49.iq & 6 & 0.65 &  \\ 
		2014-05-03-10-28-05.iq & 7 & 0.00 &  \\ 
		2014-05-03-10-28-44.iq & 7 & 0.76 & 0.886 \\ 
		2014-05-03-10-31-01.iq & 7 & 0.61 &  \\ 
		2014-05-03-10-31-27.rf & 7 & 0.41 &  \\ 
		2014-05-03-10-31-58.iq & 7 & 0.71 &  \\ 
		2014-05-03-10-32-20.rf & 7 & 0.71 &  \\ 
		2014-05-03-10-33-01.iq & 7 & 0.81 &  \\ 
		2014-05-03-11-26-50.iq & 8 & 0.00 &  \\ 
		2014-05-03-11-27-31.iq & 8 & 1.26 & 1.34 \\ 
		2014-05-03-11-29-50.iq & 8 & 1.15 &  \\ 
		2014-05-03-11-30-14.rf & 8 & 0.92 &  \\ 
		2014-05-03-11-30-55.iq & 8 & 1.15 &  \\ 
		2014-05-03-11-31-16.rf & 8 & 1.29 &  \\ 
		2014-05-03-11-31-54.iq & 8 & 1.11 &  \\ 
		2014-04-30-10-02-26.iq & 9 & 0.00 &  \\ 
		2014-04-30-10-03-54.iq & 9 & 2.03 & 2.2 \\ 
		2014-04-30-10-06-12.iq & 9 & 2.30 &  \\ 
		2014-04-30-10-06-38.rf & 9 & 1.90 &  \\ 
		2014-04-30-10-07-13.iq & 9 & 1.62 &  \\ 
		2014-04-30-10-07-42.rf & 9 & 1.22 &  \\ 
		2014-04-30-10-08-15.iq & 9 & 1.35 &  \\ 
		2014-04-30-10-53-51.iq & 10 & 0.00 &  \\ 
		2014-04-30-10-54-40.iq & 10 & 3.79 & 3.83 \\ 
		2014-04-30-10-57-03.iq & 10 & 3.48 &  \\ 
		2014-04-30-10-57-30.rf & 10 & 3.10 &  \\ 
		2014-04-30-10-58-11.iq & 10 & 2.71 &  \\ 
		2014-04-30-10-58-44.rf & 10 & 2.63 &  \\ 
		2014-04-30-10-59-28.iq & 10 & 2.48 &  \\ 
		2014-05-04-10-33-26.iq & 11 & 0.00 &  \\ 
		2014-05-04-10-34-26.iq & 11 & 0.69 & 0.76 \\ 
		2014-05-04-10-36-50.iq & 11 & 0.55 &  \\ 
		2014-05-04-10-37-12.rf & 11 & 0.27 &  \\ 
		2014-05-04-10-37-50.iq & 11 & 0.55 &  \\ 
		2014-05-04-10-38-16.rf & 11 & 0.55 &  \\ 
		2014-05-04-10-38-53.iq & 11 & 0.27 &  \\ 
		2014-05-04-12-09-24.rf & 12 & 0.00 &  \\ 
		2014-05-04-12-10-35.iq & 12 & 1.69 & 1.76 \\ 
		2014-05-04-12-12-56.iq & 12 & 1.51 &  \\ 
		2014-05-04-12-13-20.rf & 12 & 1.25 &  \\ 
		2014-05-04-12-13-54.iq & 12 & 1.46 &  \\ 
		2014-05-04-12-14-19.rf & 12 & 1.02 &  \\ 
		2014-05-04-12-14-55.iq & 12 & 1.77 &  \\ 
		2014-05-01-11-22-10.iq & 13 & 0.00 &  \\ 
		2014-05-01-11-23-04.iq & 13 & 1.95 & 2.42 \\ 
		2014-05-01-11-25-25.iq & 13 & 1.79 &  \\ 
		2014-05-01-11-26-07.rf & 13 & 2.44 &  \\ 
		2014-05-01-11-26-42.iq & 13 & 1.25 &  \\ 
		2014-05-01-11-27-07.rf & 13 & 2.66 &  \\ 
		2014-05-01-11-27-48.iq & 13 & 1.41 &  \\ 
		2014-05-01-13-29-22.iq & 14 & 0.00 &  \\ 
		2014-05-01-13-30-02.iq & 14 & 1.30 & 0.82 \\ 
		2014-05-01-13-32-22.iq & 14 & 0.92 &  \\ 
		2014-05-01-13-32-55.rf & 14 & 1.78 &  \\ 
		2014-05-01-13-33-28.iq & 14 & 0.92 &  \\ 
		2014-05-01-13-33-53.rf & 14 & 1.51 &  \\ 
		2014-05-01-13-34-40.rf & 14 & 0.87 &  \\ 
		2014-05-02-09-33-06.iq & 15 & 0.00 &  \\ 
		2014-05-02-09-33-52.iq & 15 & 4.32 & 4.6 \\ 
		2014-05-02-09-36-09.iq & 15 & 3.56 &  \\ 
		2014-05-02-09-36-32.rf & 15 & 4.10 &  \\ 
		2014-05-02-09-37-13.iq & 15 & 2.81 &  \\ 
		2014-05-02-09-37-39.rf & 15 & 3.13 &  \\ 
		2014-05-02-09-38-20.iq & 15 & 2.70 &  \\ 
		2014-05-03-11-57-58.iq & 16 & 0.00 &  \\ 
		2014-05-03-11-58-39.iq & 16 & 2.53 & 2.21 \\ 
		2014-05-03-12-00-54.iq & 16 & 2.02 &  \\ 
		2014-05-03-12-01-19.rf & 16 & 1.84 &  \\ 
		2014-05-03-12-01-52.iq & 16 & 1.93 &  \\ 
		2014-05-03-12-02-14.rf & 16 & 2.20 &  \\ 
		2014-05-03-12-02-50.iq & 16 & 1.76 &  \\ 
		\bottomrule
	\end{tabular}
\end{center}


	
%
%%SKRIVE NOE MER?? \cite{Plesset1977} 
\clearpage
\setcounter{LTchunksize}{10}
\section{Matlab files}
%\begin{table}[htbp]
%	\caption{List of Matlab files.}

The program developed in this work is written in Matlab. All written Matlab files and a short description is given in the table below (\ref{longtable}). Note that the key files are written in bold text. The actual Matlab files are found in \path{F:\usb\mfiles\}. A simple description on how to use the program is given in the following paragraph.

The main file in this program is \textit{run\_program.m}. The first input is the full file name of the raw ultrasound data (e.g. \path{D:\Ultrasound image data\2014-04-28-10-23-41.rf}). The second input argument is an array of frame numbers from which the background should be computed (e.g. \verb|1:10|). The third  and fourth argument are true/false statements. The decide if counting or background subtraction is performed, respectively. The fifth and sixth argument may be omitted, if the background file is based on frames within the video given in argument one. The sixth argument is the full file name, of a .mat where the correlation matrix for the previous video is stored (e.g.\path{C:\Phaseshiftcounting\2014-05-02-10-44-08\_PS\_count.mat}). This allow the counting of phase-shift bubble to continue from where the last video ended. The \textit{run\_program.m} perform motion correction, background subtraction and counting, in the respective order. The computation time will depend on the computer performance and the size of the raw ultrasound file, but between one and ten hours per file must be expected. Note that this script load and save several files to directories available on my computer. Directories and file names in the Matlab files may therefore be corrected for the program to work properly. the  An example on how to run through a large set of ultrasound image data is given in \textit{batch\_process.m} 

	\begin{center}
		\begin{longtable}{@{} p{3cm} p{3cm} p{2cm} p{4cm} @{}}
			\caption{List of Matlab files. The most important files are written in bold text.}
			\label{longtable}
			\toprule
			Name & Input & Output & Function\\  
			\hline \endhead
			\textbf{align\_image.m} & fixed image, moving image, transformation type,  max iterations, initial transform, max step length) &  & Align the moving image to the fixed image using MATLAB functions imregtform and imwarp. \\ \hline 
			\textbf{batch\_process.m} &  &  & This script perform batch processing of the rf-data specified in \path{counting sheet.xlsx}. For all file names, the function run\_program is launched. \\ \hline 
			bubble\_count\_curve.m &  & .mat files & Produce .mat files containing data later used for plotting of Number density as a function of time  \\ \hline 
			bubble\_density.m & Number of bubbles, region of interest, Bmode parameters & number density & Compute number density from a given number of bubbles and a ROI. \\ \hline 
			bubble\_growth.m &  &  & Estimate bubble growth curve from existing single bubble tic curve \\ \hline 
			bubble\_tic.m &  & .mat files & Produce .mat files with data later used to plot tic-curves \\ \hline 
			bubble\_zoom.m & frames, chosen pixels, show images(true/false) & maximum intensity & Zoom in on a region defined by the chosen pixels, and produce a smoothed close up video of these pixels. \\ \hline 
			\textbf{color\_code.m} & frame, contrast mask & RGB image & Color the contrast green in the .avi files. \\ \hline 
			compare\_manual\_and\_auto\_count.m &  & Figures & Make histograms comparing manually and automatically counted data. \\ \hline 
			count\_max\_real\_data.m &  & .txt file & Count the maximum counted number of phase-shift bubble for each video, and store in .txt file. \\ \hline 
			\textbf{count\_PS\_bubbles.m} & motion corrected file name, subtracted file name & .avi file and .mat file & Count the number of phase-shift bubbles and make the final .avi file. \\ \hline 
			\textbf{do\_subtraction.m} & frames & subtracted frames and background frame & Compute background and subtract the backgrounf from all frames. \\ \hline 
			draw\_roi.m & frame & ROI & Allow the user to draw a ROI on a frame. \\ \hline 
			evalc2decimal.m & string & decimal number & Convert a string to a decimal number \\ \hline 
			frame\_counter.m & Full file name of raw US data & Total number of frames in file & Count the number ov video frames in a raw file (.rf/.iq) \\ \hline 
			\textbf{get\_background.m} & Frames & background & Compute background by maximum projection of the given frames \\ \hline 
			get\_correlation.m & frame A, frame B, & correlation & Compute correlation between two frames \\ \hline 
			get\_growth\_intensity.m & max intensity, start frame, length & intensity array & Multiply the intensity obtained with growth\_fun.m to obtain correct intensity function \\ \hline 
			get\_param.m & file name raw US data & image parameter & Obtain image parameters \\ \hline 
			\textbf{get\_RF\_from\_IQ.m} & Full file name of raw US data, frame number & RF data, parameters & Obtain RF-data from IQ data.  \\ \hline 
			get\_tic.m & file\_reference, t & intensity array & Compute time intensity curve for a file given by the file reference.  t(seconds) is a time array.  \\ \hline 
			growth\_fun.m & frame numbers & growth function & Calculate the bubble growth function for the given frame numbers \\ \hline 
			hms2sec.m & d1,h1, m1, s1, d2, h2, m2, s2 & seconds & Compute seconds elapsed between two timestamps.  \\ \hline 
			im\_sub.m & frame A, frame B, type & Difference frame & Subtract frame B from A, and set all values less than zero equal to zero. \\ \hline 
			insert\_bubble.m & Intensity, x-position, y-position, frame & frame with bubble & Insert a synthesized bubble at a given position and and intensity given in the given frame \\ \hline 
			investigate\_pixels.m &  & Figures & Plot the intensity as a function of frame for a given pixel. \\ \hline 
			log\_compress\_2.m & array, 'compress/decompress' & array & Approximation to log-compression(envelope) in log\_compress.m Can compress or decompress. \\ \hline 
			log\_compress.m & RF-data & compressed data & Perform hilbert transform and log  compression on RF- data \\ \hline 
			make\_intensity\_distribution.m &  & probability object & Compute probability density function for the intensity distribution. \\ \hline 
			make\_ref\_frame.m & Full filename, frame indexes, extension(rf/iq) & reference frame and parameters & Make reference frame for motion correction. \\ \hline 
			mat2avi.m & filename & .avi file & Contruct .avi file from .mat file \\ \hline 
			\textbf{motion\_correction.m} & Filename(mat file) & motion corrected .mat file & Compute motion corrected .mat file from a .mat file \\ \hline 
			mse.m & frame A, frame B & Mean square error & Compute mean square error between two frames. \\ \hline 
			plot\_bubble\_count\_curve.m &  & Figures & Plot bubble count curves from .mat files produced with bubble\_count\_curve.m \\ \hline 
			plot\_bubble\_tic.m &  & Figures & Plot tic curves from the .mat files produced with bubble\_tic.m \\ \hline 
			plot\_tic\_curves.m & Counted .mat file name, subtracted .mat file name & time array, intensity curves & Calculate intensity curves for single bubbles and show close-up video of bubbles. \\ \hline 
			point\_spread\_fun.m &  &  & Calculate the PSF from an identified bubble. \\ \hline 
			progressbar.m & ratio & Figure & Show progress bar.  Public available script. \\ \hline 
			psf.m &  &  & return the PSF \\ \hline 
			random\_intensity.m & intensity distribution & intensity & Draw a random intensity from the intensity distribution. \\ \hline 
			random\_position.m & Roi, N & x and y coordinate & Draw N random positions within ROI. \\ \hline 
			ReadRF.m & full filename, mode name, and frame number & RF data, parameters & Read in RF data. Written by A.Healey. \\ \hline 
			\textbf{RF2mat.m} & Full filename & .mat file & Convert raw US data to .mat file. \\ \hline 
			\textbf{run\_program.m} & File name raw US data, background frame index, count(true/false), bg\_subtraction(true/false),  background file name, correlation array file name. & .avi file and . Mat file & Perform all processing from raw US data to counted .avi file and .mat file. If count or background\_subtraction is false, the program will not perform these tasks. Background file name must be included if the background is supposed to be made from another file. Correlation array file name is the name of the previous video sequence. \\ \hline 
			\textbf{subtraction\_fun.m} & File name motion corrected .mat file, indexes for background & .mat file(subtracted data) & Create background file, and subtract the background from all frames. Save subtracted frames as .mat file \\ \hline 
			time\_array\_from\_ time\_stamps.m & time stamp 1, time stamp 2. & time array(seconds) & Make time array from two time stamps \\ \hline 
			VsiBModeIQ.m & Full file name, mode name, and frame number & IQ data and parameters & Compute IQ data from RF-data. Written by A. Needles, J. Mehi. Copyright VisualSonics 1999-2010 \\

		\end{longtable}
	\end{center}
%	\label{Matlab files}
%\end{table}

\section{Number density curves}
\label{App:number_density_curves}
Number density curves as a function of time for all 16 animals are given below. Black and red data points represent data from linear contrast and non-linear imaging modes, respectively.

\begin{figure}
\includegraphics[width=\linewidth]{bubble_count1.jpg}
\caption{Number density curve for animal number 1. Black and red data points represent data from linear contrast and non-linear imaging modes, respectively.}
\end{figure}
\begin{figure}
\includegraphics[width=\linewidth]{bubble_count2.jpg}
\caption{Number density curve for animal number 2. Black and red data points represent data from linear contrast and non-linear imaging modes, respectively.}
\end{figure}
\begin{figure}
\includegraphics[width=\linewidth]{bubble_count3.jpg}
\caption{Number density curve for animal number 3. Black and red data points represent data from linear contrast and non-linear imaging modes, respectively.}
\end{figure}
\begin{figure}
\includegraphics[width=\linewidth]{bubble_count4.jpg}
\caption{Number density curve for animal number 4. Black and red data points represent data from linear contrast and non-linear imaging modes, respectively.}
\end{figure}
\begin{figure}
\includegraphics[width=\linewidth]{bubble_count5.jpg}
\caption{Number density curve for animal number 5. Black and red data points represent data from linear contrast and non-linear imaging modes, respectively.}
\end{figure}
\begin{figure}
\includegraphics[width=\linewidth]{bubble_count6.jpg}
\caption{Number density curve for animal number 6. Black and red data points represent data from linear contrast and non-linear imaging modes, respectively.}
\end{figure}
\begin{figure}
\includegraphics[width=\linewidth]{bubble_count7.jpg}
\caption{Number density curve for animal number 7. Black and red data points represent data from linear contrast and non-linear imaging modes, respectively.}
\end{figure}
\begin{figure}
\includegraphics[width=\linewidth]{bubble_count8.jpg}
\caption{Number density curve for animal number 8. Black and red data points represent data from linear contrast and non-linear imaging modes, respectively.}
\end{figure}\begin{figure}
\includegraphics[width=\linewidth]{bubble_count9.jpg}
\caption{Number density curve for animal number 9. Black and red data points represent data from linear contrast and non-linear imaging modes, respectively.}
\end{figure}
\begin{figure}
\includegraphics[width=\linewidth]{bubble_count10.jpg}
\caption{Number density curve for animal number 10. Black and red data points represent data from linear contrast and non-linear imaging modes, respectively.}
\end{figure}
\begin{figure}
\includegraphics[width=\linewidth]{bubble_count11.jpg}
\caption{Number density curve for animal number 11. Black and red data points represent data from linear contrast and non-linear imaging modes, respectively.}
\end{figure}
\begin{figure}
\includegraphics[width=\linewidth]{bubble_count12.jpg}
\caption{Number density curve for animal number 12. Black and red data points represent data from linear contrast and non-linear imaging modes, respectively.}
\end{figure}\begin{figure}
\includegraphics[width=\linewidth]{bubble_count13.jpg}
\caption{Number density curve for animal number 13. Black and red data points represent data from linear contrast and non-linear imaging modes, respectively.}
\end{figure}
\begin{figure}
\includegraphics[width=\linewidth]{bubble_count14.jpg}
\caption{Number density curve for animal number 14. Black and red data points represent data from linear contrast and non-linear imaging modes, respectively.}
\end{figure}
\begin{figure}
\includegraphics[width=\linewidth]{bubble_count15.jpg}
\caption{Number density curve for animal number 15. Black and red data points represent data from linear contrast and non-linear imaging modes, respectively.}
\end{figure}
\begin{figure}
	\includegraphics[width=\linewidth]{bubble_count16.jpg}
	\caption{Number density curve for animal number 16. Black and red data points represent data from linear contrast and non-linear imaging modes, respectively.}
\end{figure}
\clearpage
\section{Time-intensity curves}
\label{tic appendix}
Time-intensity curves for all 16 animals are presented below. Black and red data points represent data from linear contrast and non-linear imaging modes, respectively.
\begin{figure}
	\includegraphics[width=\linewidth]{tic_1.jpg}
	\caption{Time-intensity curve for animal number 1. Black and red data points represent data from linear contrast and non-linear imaging modes, respectively.}
\end{figure}
\begin{figure}
	\includegraphics[width=\linewidth]{tic_2.jpg}
	\caption{Time-intensity curve for animal number 2. Black and red data points represent data from linear contrast and non-linear imaging modes, respectively.}
\end{figure}
\begin{figure}
	\includegraphics[width=\linewidth]{tic_3.jpg}
	\caption{Time-intensity curve for animal number 3. Black and red data points represent data from linear contrast and non-linear imaging modes, respectively.}
\end{figure}
\begin{figure}
	\includegraphics[width=\linewidth]{tic_4.jpg}
	\caption{Time-intensity curve for animal number 4. Black and red data points represent data from linear contrast and non-linear imaging modes, respectively.}
\end{figure}
\begin{figure}
	\includegraphics[width=\linewidth]{tic_5.jpg}
	\caption{Time-intensity curve for animal number 5. Black and red data points represent data from linear contrast and non-linear imaging modes, respectively.}
\end{figure}
\begin{figure}
	\includegraphics[width=\linewidth]{tic_6.jpg}
	\caption{Time-intensity curve for animal number 6. Black and red data points represent data from linear contrast and non-linear imaging modes, respectively.}
\end{figure}
\begin{figure}
	\includegraphics[width=\linewidth]{tic_7.jpg}
	\caption{Time-intensity curve for animal number 7. Black and red data points represent data from linear contrast and non-linear imaging modes, respectively.}
\end{figure}
\begin{figure}
	\includegraphics[width=\linewidth]{tic_8.jpg}
	\caption{Time-intensity curve for animal number 8. Black and red data points represent data from linear contrast and non-linear imaging modes, respectively.}
\end{figure}\begin{figure}
	\includegraphics[width=\linewidth]{tic_9.jpg}
	\caption{Time-intensity curve for animal number 9. Black and red data points represent data from linear contrast and non-linear imaging modes, respectively.}
\end{figure}
\begin{figure}
	\includegraphics[width=\linewidth]{tic_10.jpg}
	\caption{Time-intensity curve for animal number 10. Black and red data points represent data from linear contrast and non-linear imaging modes, respectively.}
\end{figure}
\begin{figure}
	\includegraphics[width=\linewidth]{tic_11.jpg}
	\caption{Time-intensity curve for animal number 11. Black and red data points represent data from linear contrast and non-linear imaging modes, respectively.}
\end{figure}
\begin{figure}
	\includegraphics[width=\linewidth]{tic_12.jpg}
	\caption{Time-intensity curve for animal number 12. Black and red data points represent data from linear contrast and non-linear imaging modes, respectively.}
\end{figure}\begin{figure}
	\includegraphics[width=\linewidth]{tic_13.jpg}
	\caption{Time-intensity curve for animal number 13. Black and red data points represent data from linear contrast and non-linear imaging modes, respectively.}
\end{figure}
\begin{figure}
	\includegraphics[width=\linewidth]{tic_14.jpg}
	\caption{Time-intensity curve for animal number 14. Black and red data points represent data from linear contrast and non-linear imaging modes, respectively.}
\end{figure}
\begin{figure}
	\includegraphics[width=\linewidth]{tic_15.jpg}
	\caption{Time-intensity curve for animal number 15. Black and red data points represent data from linear contrast and non-linear imaging modes, respectively.}
\end{figure}
\begin{figure}
	\includegraphics[width=\linewidth]{tic_16.jpg}
	\caption{Time-intensity curve for animal number 16. Black and red data points represent data from linear contrast and non-linear imaging modes, respectively.}
\end{figure}
\clearpage
\section{Raw counting results}
The raw counting results are stored in Excel-sheets found in \verb|F:\usb\excel\|. The first document (counting sheet.xlsx) contains the result and information about the real data set. The second document (counting sheet synthesized data.xlsx) contains the result and information about the synthesized data set.

\section{Matlab files}
\begin{table}[htbp]
	\caption{List of Matlab files.}
	\begin{center}
		\begin{tabular}{l l l p{5cm} }
			\hline
			Name & Input & Output & Function \\ \hline
			\textbf{align\_image.m} & fixed image, moving image, transformation type,  max iterations, intital transform, max step length) &  & Align the moving image to the fixed image using MATLAB functions imregtform and imwarp. \\ \hline
			\textbf{batch\_process.m} &  &  & This script perform batch processing of the rf-data specified in counting sheet.xlsx. For all filenames, the function run\_program is launched. \\ \hline
			bubble\_count\_curve.m &  & .mat files & Produce .mat files containing data later used for plotting of Number density as a function of time  \\ \hline
			bubble\_density.m & Number of bubbles, region of interest, Bmode parameters & number density & Compute number density from a given number of bubbles and a ROI. \\ \hline
			bubble\_growth.m &  &  & Estimate bubble growth curve from existing single bubble tic curve \\ \hline
			bubble\_tic.m &  & .mat files & Produce .mat files with data later used to plot tic-curves \\ \hline
			bubble\_zoom.m & frames, chosen pixels, show images(true/false) & maximum intensity & Zoom in on a region defined by the chosen pixels, and produce a smoothed close up video of these pixels. \\ \hline
			\textbf{color\_code.m} & frame, contrast mask & RGB image & Color the contrast green in the .avi files. \\ \hline
			compare\_manual\_and\_auto\_count.m &  & Figures & Make histograms comparing manually and automatically counted data. \\ \hline
			count\_max\_real\_data.m &  & .txt file & Count the maximum counted number of phase-shift bubble for each video, and store in .txt file. \\ \hline
			\textbf{count\_PS\_bubbles.m} & motion corrected filename, subtracted filename & .avi file and .mat file & Count the number of phase-shift bubbles and make the final .avi file. \\ \hline
			\textbf{do\_subtraction.m} & frames & subtracted frames and background frame & Compute background and subtract the backgrounf from all frames. \\ \hline
			draw\_roi.m & frame & ROI & Allow the user to draw a ROI on a frame. \\ \hline
			evalc2decimal.m & string & decimal number & Convert a string to a decimal number \\ \hline
			frame\_counter.m & Full file name of raw US data & Total number of frames in file & Count the number ov video frames in a raw file (.rf/.iq) \\ \hline
			\textbf{get\_background.m} & Frames & background & Compute background by maximum projection of the given frames \\ \hline
			get\_correlation.m & frame A, frame B, & correlation & Compute correlation between two frames \\ \hline
			get\_growth\_intensity.m & max intensity, start frame, length & intensity array & Multiply the intensity obtained with growth\_fun.m to obtain correct intensity function \\ \hline
			get\_param.m & file name raw US data & image parameter & Obtain image parameters \\ \hline
			\textbf{get\_RF\_from\_IQ.m} & Full file name of raw US data, frame number & RF data, parameters & Obtain RF-data from IQ data.  \\ \hline
			get\_tic.m & file\_reference, t & intensity array & Compute time intensity curve for a file given by the file reference.  t(seconds) is a time array.  \\ \hline
			growth\_fun.m & frame numbers & growth function & Calculate the bubble growth function for the given frame numbers \\ \hline
			hms2sec.m & d1,h1, m1, s1, d2, h2, m2, s2 & seconds & Compute seconds elapsed between two timestamps.  \\ \hline
			im\_sub.m & frame A, frame B, type & Difference frame & Subtract frame B from A, and set all values less than zero equal to zero. \\ \hline
			insert\_bubble.m & Intensity, x-position, y-position, frame & frame with bubble & Insert a synthesized bubble at a given position and and intensity given in the given frame \\ \hline
			investigate\_pixels.m &  & Figures & Plot the intensity as a function of frame for a given pixel. \\ \hline
			log\_compress\_2.m & array, 'compress/decompress' & array & Approximation to log-compression(envelope) in log\_compress.m Can compress or decompress. \\ \hline
			log\_compress.m & RF-data & compressed data & Perform hilbert transform and log  compression on RF- data \\ \hline
			make\_intensity\_distribution.m &  & probability object & Compute probability density function for the intensity distribution. \\ \hline
			make\_ref\_frame.m & Full filename, frame indexes, extension(rf/iq) & reference frame and parameters & Make reference frame for motion correction. \\ \hline
			mat2avi.m & filename & .avi file & Contruct .avi file from .mat file \\ \hline
			\textbf{motion\_correction.m} & Filename(mat file) & motion corrected .mat file & Compute motion corrected .mat file from a .mat file \\ \hline
			mse.m & frame A, frame B & Mean square error & Compute mean square error between two frames. \\ \hline
			plot\_bubble\_count\_curve.m &  & Figures & Plot bubble count curves from .mat files produced with bubble\_count\_curve.m \\ \hline
			plot\_bubble\_tic.m &  & Figures & Plot tic curves from the .mat files produced with bubble\_tic.m \\ \hline
			plot\_tic\_curves.m & Counted .mat file name, subtracted .mat file name & time array, intensity curves & Calculate intensity curves for single bubbles and show close-up video of bubbles. \\ \hline
			point\_spread\_fun.m &  &  & Calculate the PSF from an identified bubble. \\ \hline
			progressbar.m & ratio & Figure & Show progress bar.  Public available script. \\ \hline
			psf.m &  &  & return the PSF \\ \hline
			random\_intensity.m & intensity distribution & intensity & Draw a random intensity from the intensity distribution. \\ \hline
			random\_position.m & Roi, N & x and y coordinate & Draw N random positions within ROI. \\ \hline
			ReadRF.m & full filename, mode name, and frame number & RF data, parameters & Read in RF data. Written by A.Healey. \\ \hline
			\textbf{RF2mat.m} & Full filename & .mat file & Convert raw US data to .mat file. \\ \hline
			\textbf{run\_program.m} & Filename raw US data, background frame index, count(true/false), bg\_subtraction(true/false),  background file name, correlation array file name. & .avi file and . Mat file & Perform all processing from raw US data to counted .avi file and .mat file. If count or background\_subtraction is false, the program will not perform these tasks. Background file name must be included if the background is supposed to be made from another file. Correlation array file name is the name of the previous video sequence. \\ \hline
			\textbf{subtraction\_fun.m} & filename notion corrected .mat file, indexes for background & .mat file(subtracted data) & Create background file, and subtract the background from all frames. Save subtracted frames as .mat file \\ \hline
			time\_array\_from\_time\_stamps.m & time\_stamp1, time\_stamp 2. & time array(seconds) & Make time array from two time stamps \\ \hline
			VsiBModeIQ.m & full filename, mode name, and frame number & IQ data and parameters & Compute IQ data from RF-data. Written by A. Needles, J. Mehi. Copyright VisualSonics 1999-2010 \\ \hline
			\end{tabular}
			\end{center}
			\label{Matlab files}
			\end{table}

%\section{Matlab files}
\begin{table}[htbp]
	\caption{List of Matlab files.}
	\begin{center}
		\begin{tabular}{l l l p{5cm} }
			\hline
			Name & Input & Output & Function \\ \hline
			\textbf{align\_image.m} & fixed image, moving image, transformation type,  max iterations, intital transform, max step length) &  & Align the moving image to the fixed image using MATLAB functions imregtform and imwarp. \\ \hline
			\textbf{batch\_process.m} &  &  & This script perform batch processing of the rf-data specified in counting sheet.xlsx. For all filenames, the function run\_program is launched. \\ \hline
			bubble\_count\_curve.m &  & .mat files & Produce .mat files containing data later used for plotting of Number density as a function of time  \\ \hline
			bubble\_density.m & Number of bubbles, region of interest, Bmode parameters & number density & Compute number density from a given number of bubbles and a ROI. \\ \hline
			bubble\_growth.m &  &  & Estimate bubble growth curve from existing single bubble tic curve \\ \hline
			bubble\_tic.m &  & .mat files & Produce .mat files with data later used to plot tic-curves \\ \hline
			bubble\_zoom.m & frames, chosen pixels, show images(true/false) & maximum intensity & Zoom in on a region defined by the chosen pixels, and produce a smoothed close up video of these pixels. \\ \hline
			\textbf{color\_code.m} & frame, contrast mask & RGB image & Color the contrast green in the .avi files. \\ \hline
			compare\_manual\_and\_auto\_count.m &  & Figures & Make histograms comparing manually and automatically counted data. \\ \hline
			count\_max\_real\_data.m &  & .txt file & Count the maximum counted number of phase-shift bubble for each video, and store in .txt file. \\ \hline
			\textbf{count\_PS\_bubbles.m} & motion corrected filename, subtracted filename & .avi file and .mat file & Count the number of phase-shift bubbles and make the final .avi file. \\ \hline
			\textbf{do\_subtraction.m} & frames & subtracted frames and background frame & Compute background and subtract the backgrounf from all frames. \\ \hline
			draw\_roi.m & frame & ROI & Allow the user to draw a ROI on a frame. \\ \hline
			evalc2decimal.m & string & decimal number & Convert a string to a decimal number \\ \hline
			frame\_counter.m & Full file name of raw US data & Total number of frames in file & Count the number ov video frames in a raw file (.rf/.iq) \\ \hline
			\textbf{get\_background.m} & Frames & background & Compute background by maximum projection of the given frames \\ \hline
			get\_correlation.m & frame A, frame B, & correlation & Compute correlation between two frames \\ \hline
			get\_growth\_intensity.m & max intensity, start frame, length & intensity array & Multiply the intensity obtained with growth\_fun.m to obtain correct intensity function \\ \hline
			get\_param.m & file name raw US data & image parameter & Obtain image parameters \\ \hline
			\textbf{get\_RF\_from\_IQ.m} & Full file name of raw US data, frame number & RF data, parameters & Obtain RF-data from IQ data.  \\ \hline
			get\_tic.m & file\_reference, t & intensity array & Compute time intensity curve for a file given by the file reference.  t(seconds) is a time array.  \\ \hline
			growth\_fun.m & frame numbers & growth function & Calculate the bubble growth function for the given frame numbers \\ \hline
			hms2sec.m & d1,h1, m1, s1, d2, h2, m2, s2 & seconds & Compute seconds elapsed between two timestamps.  \\ \hline
			im\_sub.m & frame A, frame B, type & Difference frame & Subtract frame B from A, and set all values less than zero equal to zero. \\ \hline
			insert\_bubble.m & Intensity, x-position, y-position, frame & frame with bubble & Insert a synthesized bubble at a given position and and intensity given in the given frame \\ \hline
			investigate\_pixels.m &  & Figures & Plot the intensity as a function of frame for a given pixel. \\ \hline
			log\_compress\_2.m & array, 'compress/decompress' & array & Approximation to log-compression(envelope) in log\_compress.m Can compress or decompress. \\ \hline
			log\_compress.m & RF-data & compressed data & Perform hilbert transform and log  compression on RF- data \\ \hline
			make\_intensity\_distribution.m &  & probability object & Compute probability density function for the intensity distribution. \\ \hline
			make\_ref\_frame.m & Full filename, frame indexes, extension(rf/iq) & reference frame and parameters & Make reference frame for motion correction. \\ \hline
			mat2avi.m & filename & .avi file & Contruct .avi file from .mat file \\ \hline
			\textbf{motion\_correction.m} & Filename(mat file) & motion corrected .mat file & Compute motion corrected .mat file from a .mat file \\ \hline
			mse.m & frame A, frame B & Mean square error & Compute mean square error between two frames. \\ \hline
			plot\_bubble\_count\_curve.m &  & Figures & Plot bubble count curves from .mat files produced with bubble\_count\_curve.m \\ \hline
			plot\_bubble\_tic.m &  & Figures & Plot tic curves from the .mat files produced with bubble\_tic.m \\ \hline
			plot\_tic\_curves.m & Counted .mat file name, subtracted .mat file name & time array, intensity curves & Calculate intensity curves for single bubbles and show close-up video of bubbles. \\ \hline
			point\_spread\_fun.m &  &  & Calculate the PSF from an identified bubble. \\ \hline
			progressbar.m & ratio & Figure & Show progress bar.  Public available script. \\ \hline
			psf.m &  &  & return the PSF \\ \hline
			random\_intensity.m & intensity distribution & intensity & Draw a random intensity from the intensity distribution. \\ \hline
			random\_position.m & Roi, N & x and y coordinate & Draw N random positions within ROI. \\ \hline
			ReadRF.m & full filename, mode name, and frame number & RF data, parameters & Read in RF data. Written by A.Healey. \\ \hline
			\textbf{RF2mat.m} & Full filename & .mat file & Convert raw US data to .mat file. \\ \hline
			\textbf{run\_program.m} & Filename raw US data, background frame index, count(true/false), bg\_subtraction(true/false),  background file name, correlation array file name. & .avi file and . Mat file & Perform all processing from raw US data to counted .avi file and .mat file. If count or background\_subtraction is false, the program will not perform these tasks. Background file name must be included if the background is supposed to be made from another file. Correlation array file name is the name of the previous video sequence. \\ \hline
			\textbf{subtraction\_fun.m} & filename notion corrected .mat file, indexes for background & .mat file(subtracted data) & Create background file, and subtract the background from all frames. Save subtracted frames as .mat file \\ \hline
			time\_array\_from\_time\_stamps.m & time\_stamp1, time\_stamp 2. & time array(seconds) & Make time array from two time stamps \\ \hline
			VsiBModeIQ.m & full filename, mode name, and frame number & IQ data and parameters & Compute IQ data from RF-data. Written by A. Needles, J. Mehi. Copyright VisualSonics 1999-2010 \\ \hline
			\end{tabular}
			\end{center}
			\label{Matlab files}
			\end{table}

%\end{appendices}

% Indeks for rapporten. Ta bort prosenttegn hvis du vil ha det med.
%\printindex

% Avslutter dokumentet vårt:
\end{document}

% Local Variables:
% TeX-master: "master"
% End: