\section{Introduction}
%Skriv om annen forskning innen samme område. ref mail fra andy 30.okt. 
%Skriv noe om forskning ved NTNU
%short about background/motivation
%Migth mension methods used
%1-1 1/2 page

%Review of background information enabling the reader to understandt objective and significance
Cancer is currently the most common cause of disease, with approximately 14.1 million new incidents worldwide every year (per 2012)\cite{cancer1}. In the United states, 1 of 4 deaths are caused by cancer\cite{Siegel2014}, and only 50\% of the people diagnosed with cancer survive for 10 years. 

The available treatments are rarely satisfying, giving limited results and negative side effects\cite{doi:10.1056/NEJM200106283442607}. Chemotherapy, surgery and radiotherapy are the main treatment options, often combined to achieve the best result. Chemotherapy is based on systemic administration of toxic drugs. These drugs damage healthy as well as cancerous tissue, limiting the dose to a survivable amount. This amount may be too low for the cancer to be fully treated.%[ref]

Targeted cancer treatment is a hot field of research, where the goal is increased uptake of drug in cancerous tissue without harming healthy tissue. Several targeted treatments are already available, but none have so far been able to produce a precise, satisfying treatment. A common weakness is that the drug does not travel far enough into the tumor after leaving the vasculature\cite{Bae2009}. This leave remote cancerous cells untreated. Slow accumulation is another drawback. 

Ultrasound may be the solution for precise, effective drug delivery. Microbubbles are already used as contrast agents for ultrasound, and work has been conducted to use these as drug carriers as well. The main idea is to use ultrasound to trigger the release of the carried drug. The drug can be loaded onto the shell encapsulating the microbubble, but this research has given limited results due to low drug load capacity\cite{Ibsen2011}%(obs. review article). 
Another concept is the use of nanoparticles to achieve the necessary properties. In previous work by \citeauthor{Eggen2013} the drug is encapsulated in nanoparticles\cite{Eggen2013}. The nanoparticles are then used to stabilize the shell of the gas microbubbles. The shell of the microbubble can also be developed with ligands able to attach to receptors present in cancer cells\cite{Davis2008}. This is known as active targeting and can increase accumulation of drug in the cancerous tissue.

Phoenix Solutions is developing a new concept for ultrasound mediated drug delivery. This concept is to use clusters consisting of encapsulated gas microbubbles and drug-loaded emulsion droplets. When exposed to ultrasound a phase-shift fro liquid to gas occurs, and the drug is released immediately. The emulsion droplets vaporizes, creating new gas bubbles of \SI{\sim 30}{\micro\meter}, roughly ten times the initial size. These bubbles are large enough to block small vessels, keeping the drug present for an extended time. Low frequency ultrasound is applied, initiating oscillations of the large bubbles. Applied ultrasound together with microbubbles are known to enhance drug delivery through enhanced vessel permeability and sonoporation\cite{VanWamel2006a}. Similar mechanisms may be valid for the phase-shift bubbles. This new drug delivery concept is called Acoustic Cluster Therapy (ACT\texttrademark{}), and can be regarded a theranostic (combining therapy and diagnostics) product. It is possible to image the clusters and phase-shift bubbles while performing drug delivery.

The function of these microbubbles has to be proved \textit{in vivo}. This is carried through using mice with prostate cancer xenografts, where the administration and activation is imaged using high-frequency (\SIrange{16}{18}{\mega\hertz}) ultrasound. The main goal of this thesis is to develop a method process the recorded data and estimate the number of activated and stuck phase-shift bubbles within the tumor. The developed method should produce high quality display and quantification of the phase-shift bubbles. Image processing involves motion correction, background subtraction and counting. The algorithm performing counting of phase-shift bubbles is validated, including an estimate of accuracy and precision. The algorithm is applied to a data set containing ultrasound images of 16 mice. A proper evaluation of the performance of the ACT\texttrademark{} concept is essential for further development toward clinical trials. The phase-shift bubbles have previously been counted manually. 

Counting of stuck bubbles in ultrasound images is a field of image processing with limited available literature. One approach is described by \citeauthor{Needles2009}, where microbubbles bound to vessel walls are differentiated from tissue and flowing microbubbles\cite{Needles2009}. Subharmonic imaging is used to separate tissue and microbubbles, before a low-pass inter-frame filter is applied to remove the free flowing microbubbles. 






%What have been done?

%Summary of conflicting findings in literature

%What I want to do

%Purpose and significance of study




%\subsection{History}
%Ultrasound is sound with frequency above the upper limit of human hearing, considered to be at 20 kHz, and has a wide range of use. The first work on ultrasound related to spatial orientation was written in 1794 by Lazaro Spallanzini, after discovering bats ability to navigate, only using ultrasound. Yet, almost hundred years passed before Jacques and Pierre Curie discovered the piezoelectric effect and Sir Francis Galton invented a machine able to produce ultrasound at 40 kHz, both during 1880. The piezoelectric effect is the ability of some crystals to generate electrical charge, when subjected to mechanical stress.
%
%During the beginning of the 20th century the echo-locator was invented, and the first application was detecting submarines during World War 1. Use of ultrasound in medical imaging, also known as sonography, was first used in 1956 when Ian Donald measured the parietal diameter of a fetal head. Seven years later commercial sonography devices were available.
%
%The last decades there has been continuous development, and from the simple display modes used in the beginning, there is now possible to get real-time imaging in both two and three dimensions, and Doppler imaging enables continuous measurement and visualization of blood movement in vessels.




%\subsection*{Outline of thesis}

 
   