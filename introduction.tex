\section{Introduction}

\subsection{History}
Ultrasound is sound with frequency above the upper limit of human hearing, considered to be at 20 kHz, and has a wide range of use. The first work on ultrasound related to spatial orientation was written in 1794 by Lazaro Spallanzini, after discovering bats ability to navigate, only using ultrasound. Yet, almost hundred years passed before Jacques and Pierre Curie discovered the piezoelectric effect and Sir Francis Galton invented a machine able to produce ultrasound at 40 kHz, both during 1880. The piezoelectric effect is the ability of some crystals to generate electrical charge, when subjected to mechanical stress.

During the beginning of the 20th century the echo-locator was invented, and the first application was detecting submarines during World War 1. Use of ultrasound in medical imaging, also known as sonography, was first used in 1956 when Ian Donald measured the parietal diameter of a fetal head. Seven years later commercial sonography devices were available.

The last decades there has been continuous development, and from the simple display modes used in the beginning, there is now possible to get real-time imaging in both two and three dimensions, in and Doppler imaging enables continuous measurement and visualization of blood movement in blood vessels and tissues.

\subsection{} 
   