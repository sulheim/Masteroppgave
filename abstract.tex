%abstract
%Summary of Background/motivation for work
%Material and methods
%results
%Conclusion
%Approx 1 page.
%No references
\section{Abstract}
Ultrasound mediated drug delivery is an important tool in the fight against cancer. A new concept known  as ACT\texttrademark{} is under development, and two pilot imaging studies have been performed on prostate cancer xenografts in mice.  A large amount of raw ultrasound data has been recorded, but existing software can not perform the required image processing. The ACT\texttrademark{} concept is based on clusters of microbubbles and microdroplets that experience a phase-shift from liquid to gas when exposed to ultrasound. The phase-shift increases the bubble size and, make they are caught in the vasculature of the tumor. 

A complete program has been developed in Matlab\textregistered{} to process the raw ultrasound data. The program is tailored to the unique properties of the phase-shift bubbles, and is able to reduce noise and motion artefacts, to visualize the contrast agent, and to count the number of ultrasound activated phase-shift bubbles. The program produces high quality videos, displaying both free flowing contrast agent and identified, stuck phase-shift bubbles. 

The program showed a very good correlation to a manually counted data set. The program was validated against a synthesized data set, and we found that the program counted accurately up to \SI{\sim2}{bubbles\per\milli\meter}. A saturation was experienced above this threshold, and too few bubbles were counted.  

