%keywords.tex
%author: Snorre Sulheim 
\section*{Keywords}
\textit{Pulse-echo imaging} is the basis of ultrasound imaging. An image is formed by sending an US pulse, and note the time of the received echos. The echos are plotted om a grid with position according to line of measurement and position proportional to the return travel time.

\textit{Transmit frequencies} for US are all above 20kHz. In imaging of humans, frequencies between 2 and 18 MHz are common. For imaging of small animals, also know as micro-ultrasound, the imaging frequency is in the range from 15 to 80 MHz. Choice of transmit frequency depend on how deep you want to image. High frequencies give a better axial resolution, but is more attenuated than lower frequencies.

\textit{Focal zones} is where the US beam is focused, and thus the point in the image with best lateral resolution and highest intensity. The focusing can be performed using either an acoustic lens or a phased array element. 

\textit{Envelope detection} is the detection technique used to determine the brightness in B-mode imaging. The received echos at the transducer has the same sinusoidal shape as the transmitted pulse, and the envelope of this curve is extracted by first rectifying the signal. The envelope of the signal is then obtained by applying a low-pass filter to remove the high frequency oscillations. The remaining signal is the envelope, and the height of the envelope(after TGC) determines the brightness at that point in the B-mode display. 

\textit{Time-gain compensation(TGC)} is a technique used to compensate for decreasing echo intensity from deep tissue interfaces. After TGC is applied, echoes from similar interfaces should have the same intensity, regardless of depth.

\textit{Dynamic range} of signals at the transducer is the ratio between the largest echo amplitude which does not cause distortion and the smallest echo distinguishable from signal noise.

\textit{Compression} is necessary to allow echos from both tissue interfaces and organ parenchyma(the inside) to be visible in the display, the range of amplitude echos has to be compressed from 60 dB to 2 dB, which is maximum brightness ratio on a common screen. The data is compressed by applying a non-linear amplifier which provide more gain for the weak echoes. 

\textit{Apodization} is a method used to reduce the effect of side lobes in some arrays. This is usually accomplished by decreasing the vibration of crystals far from the center of the transducer.



