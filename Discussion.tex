%Discussion
%Discuss methods then results
\section{Discussion}
%\subsection{Why existing algorithm are not applicable}
%Several methods for distinguish stuck microbubbles from free flowing microbubbles and tissue were described in section \ref{existing algorithms}. The size of the ACT\textregistered phase-shift bubbles distinguish them from other microbubbles, and allow a new approach for segmentation. The method described in \cite{Rychak2006} has its obvious drawback of waiting for free bubbles to clear. During this time most of the ACT\textregistered will disappear as well. The first method described in \cite{Zhao2007} assume that no bubbles are stuck before the first image. 

\subsection{Synthesized data set for validation}
As there exist no 'Gold standard' for counting of phase-shift bubbles, a synthesized data set was needed to get a quantitative evaluation of the performance of the counting algorithm. By knowing the number of synthesized phase-shift bubbles we could determine the accuracy and precision for different number densities.

The synthesized data set was constructed to replicate an administration of ACT\textregistered and Sonazoid\texttrademark microbubbles. 


\subsection{Comparison to existing program(VisualSonics)}
\subsection{Comparison to Andy's manual counting}

\subsection{Decreasing performance with increasing bubble density}
\subsection{Performance of motion correction}
The motion correction applied in this work use a linear affine transformation to correct for motion in image. This can not correct for local deformations in the image. Hence, some videos will not be free of motion. A tumor is in general quite rigid, and there is thus rarely local deformations within the tumor region. Non-linear(local) image registration is therefore not necessary. Linear transformation has the benefit of being faster and more robust than non-linear transformations. 

The current motion correction program is slow. Number of pyramidal steps and iterations can be more closely evaluated to increase the speed, but this was not prioritized in this work. 

\subsection{Counting performance}
\subsection{Findings in broader context}


