%Discussion
%Discuss methods then results
\section{Discussion}
The developed and validated algorithm is a tool that might increase the understanding of the activation, administration and \texttrademark{} of the ACT\texttrademark{} bubbles. Compared to manual counting this program give increased efficiency, and a result with a known accuracy and precision. It is also shown that the chosen method of motion correction, background subtraction and stuck bubble identification provides satisfying results. Although this program is tailored to the ACT\texttrademark{} bubbles, these methods may apply to similar problems.
%\subsection{Why existing algorithm are not applicable}
%Several methods for distinguish stuck microbubbles from free flowing microbubbles and tissue were described in section \ref{existing algorithms}. The size of the ACT\texttrademark{} phase-shift bubbles distinguish them from other microbubbles, and allow a new approach for segmentation. The method described in \cite{Rychak2006} has its obvious drawback of waiting for free bubbles to clear. During this time most of the ACT\texttrademark{} will disappear as well. The first method described in \cite{Zhao2007} assume that no bubbles are stuck before the first image. 
\subsection{Performance of motion correction}
The motion correction applied in this work, uses a linear affine transformation to correct for motion in image. This can not correct for local deformations in the tissue or movements out of the image plane. Hence, some videos are not free of motion. A tumor is in general quite rigid, and there is thus rarely local deformations within the tumor region. Non-linear (local) image registration is therefore unnecessary. Linear transformation has the benefit of being faster and more robust than non-linear transformations. The current motion correction program is slow. The number of pyramidal steps and iterations can be more closely evaluated to increase the speed, but this was not prioritized in this work. 

\subsection{Qualitative validation}
In Section \ref{sec:qualitative} there are several results which support the claim that the counted bubbles are phase-shift bubbles. In Figure \ref{Fig:bubble_kinetic} and the corresponding video, the phase-shift bubble dynamics are visible. A phase-shift bubble is getting stuck, releases, and attaches to another location. The adhesion and activation is also clearly visible in Figure \ref{bubble_zoom}. The activation and dynamics imply that the identified bubble is a phase-shift bubble. The number density curves as a function of time (Figure \ref{Fig:tic_count}) display how the phase-shift bubbles stay for several minutes. This may increase the drug retention time by fully or partially blocking the vasculature.

\subsubsection{Effect of high power ultrasound}
The high power ultrasound has clearly an effect on the counted number of bubbles (Figure \ref{High power US non-lin} and \ref{High power US lin}). In general, there is not a large decrease. This implies that most of the identified bubbles are not Sonazoid\texttrademark{}. As mentioned earlier, they are disrupted by high power ultrasound. The high ultrasound burst is short (\SI{\sim 0.1}{\second}) in linear imaging mode, and long (\SI{\sim 2}{\second}) in non-linear mode.

In linear imaging mode (Figure \ref{High power US lin}) there is a large spread in the relative change. The reasons are not fully understood, but there are a few possibilities. Some bubbles are obviously activated (\ref{Fig:high_power_US}). The decrease observed in three of the data sets may come from destroyed Sonazoid\texttrademark{}, which may have been stuck in vascular dead ends or inflammation areas\cite{Healey_pc}. The radiation force from the high-power ultrasound may be able to push bubbles in or out of the image plane, and this may also cause a change in number of counted bubbles. In non-linear imaging mode (Figure \ref{High power US non-lin}), there is a large increase in two thirds of the data sets. This reason for this is not known.

\subsection{Synthesized data set for validation}
As there exist no 'Gold standard' for counting of phase-shift bubbles, a synthesized data set was needed to get a quantitative evaluation of the performance of the counting algorithm. By knowing the number of synthesized phase-shift bubbles we could determine the accuracy and precision for different number densities.

The synthesized data set was constructed to imitate an administration of ACT\texttrademark{} and Sonazoid\texttrademark{} microbubbles. To replicate the inflow, time of appearance, maximum intensity and position were chosen randomly from distributions generated from real data. In the real data some phase-shift bubbles are getting stuck after activation, or are released from the their location before the size is reduced. This is not accounted for in the synthesized data. 

The synthesized data set was based upon three different background videos, and there is clearly difference in terms of counting performance (Figure \ref{Fig:close comparison}). We observe a lower number density of counted bubbles in Background 2, compared to Background 1 and 3. In the corresponding videos we find that this background has less visible flow of contrast agent. The increased flow of contrast agent in Background 1 and 3 create variations in the signal intensity. This may help in differentiating adjacent stuck bubbles from each other. 

\subsection{Performance of counting algorithm}

We observe a good correlation between the automatically and manually counted number density of phase-shift bubbles (Figure \ref{Fig:Number density of counted bubbles}). This supports the credibility of the developed algorithm.

The performance of the counting algorithm for the full synthesized data set was shown in Figure \ref{Fig:counted_vs_inserted_all}. The counting algorithm experiences a saturation when the inserted density passes \SI{\sim 2}{bubbles\per\milli\meter\squared} (Figure \ref{Fig:counted_vs_inserted_inverse}). This is an expected behaviour. When the density of bubbles increases, the chance of two or more bubbles being adjacent increases as well, and they may be counted as one large bubble (Figure \ref{Fig:saturation}). Although this is a drawback, it is important to note that most ($\sim 86\%$) of the counted number densities in the real data set is below \SI{2}{bubbles\per\milli\meter} (Figure \ref{Fig:number density distribution}). Hence, the error from saturation in the real data is limited.


\begin{figure}[h]
	\centering
	\includegraphics[width=\linewidth]{saturation.png}
	 \cprotect\caption{Saturation is clearly seen in as large clusters of bubbles recognized as one bubble. The corresponding movie is found at  \path{F:\usb\avi\2014-05-01-11-44-15_PS_counted_258.avi}.}
	\label{Fig:saturation}
\end{figure}

\begin{figure}[h]
	\centering
	\includegraphics[width=\linewidth]{number_density_distribution.png}
	\caption{The number density distribution of phase-shift bubbles in the real data set. We observe that $\sim 86\%$ of the counted densities is below \SI{2}{bubbles\per\milli\meter}.}
	\label{Fig:number density distribution}
\end{figure} 


If we only consider the lower end of the inserted density (Figure \ref{Fig:counted_vs_inserted_all_small}), there is almost a linear relationship between the counted and inserted number of bubbles. By inverting the axes and fitting a linear curve to the log-transformed data, we obtain an estimate of the accuracy and precision of the counting algorithm (Figure\ref{Fig:counted_vs_inserted_inverse}). We find that the algorithm is slightly under-counting. The relative standard deviation of the counting algorithm is fairly constant, with a mean of 0.016 (Figure \ref{Fig:rsd}).

\subsection{Comparison to existing program (Visual Sonics)}
In Figure \ref{Fig:compare VisualSonics} we present a comparison of the Visual Sonics and our program's performance in terms of motion correction and background subtraction. An improved background subtraction makes the single phase-shift bubbles more visible, and the noise and motion artefacts are strongly reduced.

\begin{figure}[h]
	\centering
	\begin{minipage}[b]{0.42\textwidth}
		\centering
		\includegraphics[width=\textwidth]{vevo_10_40_53.png}\\
		(a)
	\end{minipage}%
	\begin{minipage}[b]{0.35\textwidth}
		\centering
		\includegraphics[width=\textwidth]{10_40_53_snorre.png}\\
		(b)
	\end{minipage}%
	 \cprotect\caption{The same video sequence processed with the Visual Sonics (a) and our program (b). The improved background subtraction increases the visibility of the individual bubbles, and there is a reduction in motion artefacts. The corresponding movies are found at \path{F:\usb\avi\#01 2014-04-28-10-40-53.avi} and \path{F:\usb\avi\2014-04-28-10-40-53_count_and_color_11_to_1000dilate_1_intensity_1000ct_0.85running_avg.avi}.}
	\label{Fig:compare VisualSonics}
\end{figure}

\subsection{Non-linear imaging}
The counting algorithm was tailored to linear contrast images. This imaging mode is suppose to enhance the phase-shift bubbles compared to Sonazoid\texttrademark{} microbubbles and tissue. From the time-intensity curves (APPENDIX ?/Figure?) we find that the total contrast is slightly larger in linear contrast mode, compared to non-linear contrast mode. On the other hand, the number density curves (APPENDIX ?/Figure?) display no difference between the two imaging modes, in terms of the counting performance. Hence, it is not possible to make a statement regarding choice of imaging mode based on our results.


%\subsection{Findings in a broader context}


%The focal plane was measured to \SI{400}{\micro\meter}.If the total volume of the tumor is estimated, it is possible to compute the total number of deposited ACT\texttrademark{} bubbles. Assuming a linear relationship between total amount of drug delivered and number of stuck bubbles, it is possible to make a relative estimate of the drug delivered. This can be compared to results from ongoing treatment studies.

%
%Relate to amount of bubbles introduced.
%
%Relate to concentration for each animal.

