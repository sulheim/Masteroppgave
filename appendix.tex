
\section{Derivation of the Rayleigh-Plesset equation}
The Rayleigh-Plesset equation is an ordinary differential equation which describe the non-linear oscillation of a gas bubble suspended in an infinite liquid, subject to an external sound wave. This equation is in the following derived using the energy balance between the liquid and the gas bubble \cite{Moss2014}. The equation can also be derived from the Navier-Stokes equations\cite{leighton2007derivation}.

A few assumptions are required for the following derivation to be valid. We assume the wavelength of the pressure field to be way larger then the size of the gas bubble, i.e. $d \ll \lambda$. The bubble is spherical and spatially uniform conditions within the bubble exist at all times. We can neglect gravity and bulk viscosity. There is no flow of either matter of heat through the boundary of the bubble. The density of the gas is significantly smaller then the liquid density, and the gas within the bubble can be considered an ideal gas. Using these assumptions the oscillation will be an adiabatic process.  Newton's notation for the time-derivative is used.

Consider an oscillating bubble suspended in incompressible fluid. Because of the incompressibility of the fluid, the fluid velocity $u(r,t)$ has to follow the inverse square law, i.e. 

\begin{equation}
\label{eq:1}
u(r,t) = \frac{R(t)^2}{r(t)^2}\dot{R}(t).
\end{equation}
Here $r_b(t)$ is the bubble radius, $\dot{r}_b(t)$ the velocity of the boundary and $r$ any radius larger or equal $r_b$. 

The kinetic energy $E_k$ of the fluid caused by the oscillating bubble is then
\begin{equation}
\label{kinetic energy}
E_k = \frac{\rho_l}{2}\int_{r_b}^\infty u(r,t)4\pi r^2 dr = 2\pi\rho_l r_b^3 \dot{r}_b^2,
\end{equation}
where $\rho_l$ is the liquid density.

Far from the bubble the liquid pressure is given by $p_{\infty}(t) = p_0 + p(t)$, where $p(t) = p_a e^{i\omega t}$ is the time varying pressure caused by the sound wave and $p_0$ the hydrostatic pressure. For an adiabatic process we have that $pV^{\gamma}=\mathrm{constant}$. Here $V$ is the volume, $p$ the pressure and $\gamma$ the adiabatic index. The pressure is then only a function the bubble radius $r_b$,
\begin{equation}
p(r_b) = p_{r_0}\left(\frac{V_{r_0}}{V(r_b)}\right)^{\gamma} =  p_{r_0}\left(\frac{r_0}{r_b}\right)^{3\gamma},
\end{equation}

where $r_0$ and $p_{r_0}$ are the equilibrium radius and pressure. 

The work done to expand the bubble is only carried out by the net pressure, $\Delta p = P(r_b)-P_{\infty}(t)$, and the total work $W$ is

\begin{equation}
\label{bubble work}
W = \int \Delta p dV = \int _{r_0}^{r_b} (p(r_b)-p_{\infty(t)})4\pi r_b^2 dr_b.
\end{equation}

The kinetic energy of the liquid must equal the work, and the Rayleigh-Plesset equation is obtained by equating and differentiating Equation \eqref{kinetic energy} and \eqref{bubble work} with respect to $r_b$, 

\begin{equation}
\label{RPE}
\frac{p_{r_0}\left(\frac{r_0}{r_b}\right)^{3\gamma}-p_0 - p(t)}{\rho_l} = \frac{3\dot{r}_b^2}{2}+r_b\ddot{r}_b.
\end{equation}

Note that $\frac{d\dot{r}_b^2}{dr-b} = 2\ddot{r_b}$. 
\clearpage
\section{Raw counting results}
\label{raw counting}
The raw counting results are stored in Excel-sheets found in \path{F:\usb\excel\}. The first document (\path{counting sheet.xlsx}) contains the result and information about the real data set. The second document (\path{counting sheet synthesized data.xlsx}) contains the result and information about the synthesized data set.
	


\begin{center}
	\begin{longtable}{@{}l l l l@{}}
		\caption{Counted raw ultrasound data.}
		\label{counting sheet}
		\toprule
		File id & Animal nr & Density of bubbles (automatic) & Density of bubbles (manual) \\ 
		\midrule \endhead
		2014-11-04-15-29-07 & 0 &  &  \\ 
		2014-11-04-15-31-39 & 0 &  &  \\ 
		2014-11-04-15-36-35 & 0 &  &  \\ 
		2014-11-04-15-38-42 & 0 &  &  \\ 
		2014-11-04-15-40-50 & 0 &  &  \\ 
		2014-04-28-10-37-22.rf & 1 & 0.00 &  \\ 
		2014-04-28-10-40-53.rf & 1 & 1.54 & 0.75 \\ 
		2014-04-28-10-44-00.rf & 1 & 1.54 &  \\ 
		2014-04-28-10-45-33.iq & 1 & 0.73 &  \\ 
		2014-04-28-10-47-00.rf & 1 & 1.22 &  \\ 
		2014-04-28-10-47-31.rf & 1 & 0.57 &  \\ 
		2014-04-28-10-54-52.iq & 1 & 0.41 &  \\ 
		2014-04-29-09-43-36.iq & 2 & 0.00 &  \\ 
		2014-04-29-09-45-50.iq & 2 & 0.96 & 1.35 \\ 
		2014-04-29-09-48-08.iq & 2 & 0.73 &  \\ 
		2014-04-29-09-48-38.iq & 2 & 0.63 &  \\ 
		2014-04-29-09-48-53.rf & 2 & 0.60 &  \\ 
		2014-04-29-09-49-36.iq & 2 & 0.70 &  \\ 
		2014-04-29-09-50-22.rf & 2 & 0.17 &  \\ 
		2014-04-29-09-51-05.iq & 2 & 0.63 &  \\ 
		2014-04-29-09-52-06.iq & 2 & 0.17 &  \\ 
		2014-04-29-09-55-24.rf & 2 & 0.13 &  \\ 
		2014-04-29-09-56-56.rf & 2 & 0.20 &  \\ 
		2014-04-29-09-58-15.rf & 2 & 0.23 &  \\ 
		2014-04-30-08-45-53.iq & 3 & 0.00 &  \\ 
		2014-04-30-08-48-40.iq & 3 & 0.52 & 0.36 \\ 
		2014-04-30-08-51-24.iq & 3 & 0.31 &  \\ 
		2014-04-30-08-52-05.rf & 3 & 0.26 &  \\ 
		2014-04-30-08-52-39.iq & 3 & 0.52 &  \\ 
		2014-04-30-08-53-11.rf & 3 & 0.21 &  \\ 
		2014-04-30-08-54-43.rf & 3 & 0.16 &  \\ 
		2014-04-30-08-55-20.iq & 3 & 0.37 &  \\ 
		2014-04-30-08-58-37.rf & 3 & 0.42 &  \\ 
		2014-04-30-09-00-03.rf & 3 & 0.21 &  \\ 
		2014-04-30-09-02-46.rf & 3 & 0.42 &  \\ 
		2014-04-30-09-03-48.rf & 3 & 0.31 &  \\ 
		2014-05-04-11-37-15.iq & 4 & 0.00 &  \\ 
		2014-05-04-11-38-18.iq & 4 & 0.96 & 0.5 \\ 
		2014-05-04-11-40-41.iq & 4 & 0.72 &  \\ 
		2014-05-04-11-41-03.rf & 4 & 1.20 &  \\ 
		2014-05-04-11-41-36.iq & 4 & 0.72 &  \\ 
		2014-05-04-11-42-00.rf & 4 & 0.72 &  \\ 
		2014-05-04-11-42-32.iq & 4 & 0.24 &  \\ 
		2014-05-04-11-43-11.rf & 4 & 0.96 &  \\ 
		2014-05-04-11-44-01.rf & 4 & 0.96 &  \\ 
		2014-05-04-11-49-10.rf & 4 & 0.48 &  \\ 
		2014-05-02-10-39-41.iq & 5 & 0.00 &  \\ 
		2014-05-02-10-40-23.iq & 5 & 2.16 & 1.625 \\ 
		2014-05-02-10-42-36.iq & 5 & 1.44 &  \\ 
		2014-05-02-10-43-07.rf & 5 & 2.94 &  \\ 
		2014-05-02-10-43-44.iq & 5 & 1.22 &  \\ 
		2014-05-02-10-44-08.rf & 5 & 2.55 &  \\ 
		2014-05-02-10-44-41.iq & 5 & 1.16 &  \\ 
		2014-05-03-09-52-43.iq & 6 & 0.00 &  \\ 
		2014-05-03-09-53-31.iq & 6 & 1.96 & 1.6 \\ 
		2014-05-03-09-55-51.iq & 6 & 1.13 &  \\ 
		2014-05-03-09-56-15.rf & 6 & 2.66 &  \\ 
		2014-05-03-09-56-48.iq & 6 & 1.05 &  \\ 
		2014-05-03-09-57-12.rf & 6 & 2.53 &  \\ 
		2014-05-03-09-57-49.iq & 6 & 0.65 &  \\ 
		2014-05-03-10-28-05.iq & 7 & 0.00 &  \\ 
		2014-05-03-10-28-44.iq & 7 & 0.76 & 0.886 \\ 
		2014-05-03-10-31-01.iq & 7 & 0.61 &  \\ 
		2014-05-03-10-31-27.rf & 7 & 0.41 &  \\ 
		2014-05-03-10-31-58.iq & 7 & 0.71 &  \\ 
		2014-05-03-10-32-20.rf & 7 & 0.71 &  \\ 
		2014-05-03-10-33-01.iq & 7 & 0.81 &  \\ 
		2014-05-03-11-26-50.iq & 8 & 0.00 &  \\ 
		2014-05-03-11-27-31.iq & 8 & 1.26 & 1.34 \\ 
		2014-05-03-11-29-50.iq & 8 & 1.15 &  \\ 
		2014-05-03-11-30-14.rf & 8 & 0.92 &  \\ 
		2014-05-03-11-30-55.iq & 8 & 1.15 &  \\ 
		2014-05-03-11-31-16.rf & 8 & 1.29 &  \\ 
		2014-05-03-11-31-54.iq & 8 & 1.11 &  \\ 
		2014-04-30-10-02-26.iq & 9 & 0.00 &  \\ 
		2014-04-30-10-03-54.iq & 9 & 2.03 & 2.2 \\ 
		2014-04-30-10-06-12.iq & 9 & 2.30 &  \\ 
		2014-04-30-10-06-38.rf & 9 & 1.90 &  \\ 
		2014-04-30-10-07-13.iq & 9 & 1.62 &  \\ 
		2014-04-30-10-07-42.rf & 9 & 1.22 &  \\ 
		2014-04-30-10-08-15.iq & 9 & 1.35 &  \\ 
		2014-04-30-10-53-51.iq & 10 & 0.00 &  \\ 
		2014-04-30-10-54-40.iq & 10 & 3.79 & 3.83 \\ 
		2014-04-30-10-57-03.iq & 10 & 3.48 &  \\ 
		2014-04-30-10-57-30.rf & 10 & 3.10 &  \\ 
		2014-04-30-10-58-11.iq & 10 & 2.71 &  \\ 
		2014-04-30-10-58-44.rf & 10 & 2.63 &  \\ 
		2014-04-30-10-59-28.iq & 10 & 2.48 &  \\ 
		2014-05-04-10-33-26.iq & 11 & 0.00 &  \\ 
		2014-05-04-10-34-26.iq & 11 & 0.69 & 0.76 \\ 
		2014-05-04-10-36-50.iq & 11 & 0.55 &  \\ 
		2014-05-04-10-37-12.rf & 11 & 0.27 &  \\ 
		2014-05-04-10-37-50.iq & 11 & 0.55 &  \\ 
		2014-05-04-10-38-16.rf & 11 & 0.55 &  \\ 
		2014-05-04-10-38-53.iq & 11 & 0.27 &  \\ 
		2014-05-04-12-09-24.rf & 12 & 0.00 &  \\ 
		2014-05-04-12-10-35.iq & 12 & 1.69 & 1.76 \\ 
		2014-05-04-12-12-56.iq & 12 & 1.51 &  \\ 
		2014-05-04-12-13-20.rf & 12 & 1.25 &  \\ 
		2014-05-04-12-13-54.iq & 12 & 1.46 &  \\ 
		2014-05-04-12-14-19.rf & 12 & 1.02 &  \\ 
		2014-05-04-12-14-55.iq & 12 & 1.77 &  \\ 
		2014-05-01-11-22-10.iq & 13 & 0.00 &  \\ 
		2014-05-01-11-23-04.iq & 13 & 1.95 & 2.42 \\ 
		2014-05-01-11-25-25.iq & 13 & 1.79 &  \\ 
		2014-05-01-11-26-07.rf & 13 & 2.44 &  \\ 
		2014-05-01-11-26-42.iq & 13 & 1.25 &  \\ 
		2014-05-01-11-27-07.rf & 13 & 2.66 &  \\ 
		2014-05-01-11-27-48.iq & 13 & 1.41 &  \\ 
		2014-05-01-13-29-22.iq & 14 & 0.00 &  \\ 
		2014-05-01-13-30-02.iq & 14 & 1.30 & 0.82 \\ 
		2014-05-01-13-32-22.iq & 14 & 0.92 &  \\ 
		2014-05-01-13-32-55.rf & 14 & 1.78 &  \\ 
		2014-05-01-13-33-28.iq & 14 & 0.92 &  \\ 
		2014-05-01-13-33-53.rf & 14 & 1.51 &  \\ 
		2014-05-01-13-34-40.rf & 14 & 0.87 &  \\ 
		2014-05-02-09-33-06.iq & 15 & 0.00 &  \\ 
		2014-05-02-09-33-52.iq & 15 & 4.32 & 4.6 \\ 
		2014-05-02-09-36-09.iq & 15 & 3.56 &  \\ 
		2014-05-02-09-36-32.rf & 15 & 4.10 &  \\ 
		2014-05-02-09-37-13.iq & 15 & 2.81 &  \\ 
		2014-05-02-09-37-39.rf & 15 & 3.13 &  \\ 
		2014-05-02-09-38-20.iq & 15 & 2.70 &  \\ 
		2014-05-03-11-57-58.iq & 16 & 0.00 &  \\ 
		2014-05-03-11-58-39.iq & 16 & 2.53 & 2.21 \\ 
		2014-05-03-12-00-54.iq & 16 & 2.02 &  \\ 
		2014-05-03-12-01-19.rf & 16 & 1.84 &  \\ 
		2014-05-03-12-01-52.iq & 16 & 1.93 &  \\ 
		2014-05-03-12-02-14.rf & 16 & 2.20 &  \\ 
		2014-05-03-12-02-50.iq & 16 & 1.76 &  \\ 
		\bottomrule
	\end{tabular}
\end{center}


	
%
%%SKRIVE NOE MER?? \cite{Plesset1977} 
\clearpage
\setcounter{LTchunksize}{10}
\section{Matlab files}
%\begin{table}[htbp]
%	\caption{List of Matlab files.}

The program developed in this work is written in Matlab. All written Matlab files and a short description is given in the table below (\ref{longtable}). Note that the key files are written in bold text. The actual Matlab files are found in \path{F:\usb\mfiles\}. A simple description on how to use the program is given in the following paragraph.

The main file in this program is \textit{run\_program.m}. The first input is the full file name of the raw ultrasound data (e.g. \path{D:\Ultrasound image data\2014-04-28-10-23-41.rf}). The second input argument is an array of frame numbers from which the background should be computed (e.g. \verb|1:10|). The third  and fourth argument are true/false statements. The decide if counting or background subtraction is performed, respectively. The fifth and sixth argument may be omitted, if the background file is based on frames within the video given in argument one. The sixth argument is the full file name, of a .mat where the correlation matrix for the previous video is stored (e.g.\path{C:\Phaseshiftcounting\2014-05-02-10-44-08\_PS\_count.mat}). This allow the counting of phase-shift bubble to continue from where the last video ended. The \textit{run\_program.m} perform motion correction, background subtraction and counting, in the respective order. The computation time will depend on the computer performance and the size of the raw ultrasound file, but between one and ten hours per file must be expected. Note that this script load and save several files to directories available on my computer. Directories and file names in the Matlab files may therefore be corrected for the program to work properly. the  An example on how to run through a large set of ultrasound image data is given in \textit{batch\_process.m} 

	\begin{center}
		\begin{longtable}{@{} p{3cm} p{3cm} p{2cm} p{4cm} @{}}
			\caption{List of Matlab files. The most important files are written in bold text.}
			\label{longtable}
			\toprule
			Name & Input & Output & Function\\  
			\hline \endhead
			\textbf{align\_image.m} & fixed image, moving image, transformation type,  max iterations, initial transform, max step length) &  & Align the moving image to the fixed image using MATLAB functions imregtform and imwarp. \\ \hline 
			\textbf{batch\_process.m} &  &  & This script perform batch processing of the rf-data specified in \path{counting sheet.xlsx}. For all file names, the function run\_program is launched. \\ \hline 
			bubble\_count\_curve.m &  & .mat files & Produce .mat files containing data later used for plotting of Number density as a function of time  \\ \hline 
			bubble\_density.m & Number of bubbles, region of interest, Bmode parameters & number density & Compute number density from a given number of bubbles and a ROI. \\ \hline 
			bubble\_growth.m &  &  & Estimate bubble growth curve from existing single bubble tic curve \\ \hline 
			bubble\_tic.m &  & .mat files & Produce .mat files with data later used to plot tic-curves \\ \hline 
			bubble\_zoom.m & frames, chosen pixels, show images(true/false) & maximum intensity & Zoom in on a region defined by the chosen pixels, and produce a smoothed close up video of these pixels. \\ \hline 
			\textbf{color\_code.m} & frame, contrast mask & RGB image & Color the contrast green in the .avi files. \\ \hline 
			compare\_manual\_and\_auto\_count.m &  & Figures & Make histograms comparing manually and automatically counted data. \\ \hline 
			count\_max\_real\_data.m &  & .txt file & Count the maximum counted number of phase-shift bubble for each video, and store in .txt file. \\ \hline 
			\textbf{count\_PS\_bubbles.m} & motion corrected file name, subtracted file name & .avi file and .mat file & Count the number of phase-shift bubbles and make the final .avi file. \\ \hline 
			\textbf{do\_subtraction.m} & frames & subtracted frames and background frame & Compute background and subtract the backgrounf from all frames. \\ \hline 
			draw\_roi.m & frame & ROI & Allow the user to draw a ROI on a frame. \\ \hline 
			evalc2decimal.m & string & decimal number & Convert a string to a decimal number \\ \hline 
			frame\_counter.m & Full file name of raw US data & Total number of frames in file & Count the number ov video frames in a raw file (.rf/.iq) \\ \hline 
			\textbf{get\_background.m} & Frames & background & Compute background by maximum projection of the given frames \\ \hline 
			get\_correlation.m & frame A, frame B, & correlation & Compute correlation between two frames \\ \hline 
			get\_growth\_intensity.m & max intensity, start frame, length & intensity array & Multiply the intensity obtained with growth\_fun.m to obtain correct intensity function \\ \hline 
			get\_param.m & file name raw US data & image parameter & Obtain image parameters \\ \hline 
			\textbf{get\_RF\_from\_IQ.m} & Full file name of raw US data, frame number & RF data, parameters & Obtain RF-data from IQ data.  \\ \hline 
			get\_tic.m & file\_reference, t & intensity array & Compute time intensity curve for a file given by the file reference.  t(seconds) is a time array.  \\ \hline 
			growth\_fun.m & frame numbers & growth function & Calculate the bubble growth function for the given frame numbers \\ \hline 
			hms2sec.m & d1,h1, m1, s1, d2, h2, m2, s2 & seconds & Compute seconds elapsed between two timestamps.  \\ \hline 
			im\_sub.m & frame A, frame B, type & Difference frame & Subtract frame B from A, and set all values less than zero equal to zero. \\ \hline 
			insert\_bubble.m & Intensity, x-position, y-position, frame & frame with bubble & Insert a synthesized bubble at a given position and and intensity given in the given frame \\ \hline 
			investigate\_pixels.m &  & Figures & Plot the intensity as a function of frame for a given pixel. \\ \hline 
			log\_compress\_2.m & array, 'compress/decompress' & array & Approximation to log-compression(envelope) in log\_compress.m Can compress or decompress. \\ \hline 
			log\_compress.m & RF-data & compressed data & Perform hilbert transform and log  compression on RF- data \\ \hline 
			make\_intensity\_distribution.m &  & probability object & Compute probability density function for the intensity distribution. \\ \hline 
			make\_ref\_frame.m & Full filename, frame indexes, extension(rf/iq) & reference frame and parameters & Make reference frame for motion correction. \\ \hline 
			mat2avi.m & filename & .avi file & Contruct .avi file from .mat file \\ \hline 
			\textbf{motion\_correction.m} & Filename(mat file) & motion corrected .mat file & Compute motion corrected .mat file from a .mat file \\ \hline 
			mse.m & frame A, frame B & Mean square error & Compute mean square error between two frames. \\ \hline 
			plot\_bubble\_count\_curve.m &  & Figures & Plot bubble count curves from .mat files produced with bubble\_count\_curve.m \\ \hline 
			plot\_bubble\_tic.m &  & Figures & Plot tic curves from the .mat files produced with bubble\_tic.m \\ \hline 
			plot\_tic\_curves.m & Counted .mat file name, subtracted .mat file name & time array, intensity curves & Calculate intensity curves for single bubbles and show close-up video of bubbles. \\ \hline 
			point\_spread\_fun.m &  &  & Calculate the PSF from an identified bubble. \\ \hline 
			progressbar.m & ratio & Figure & Show progress bar.  Public available script. \\ \hline 
			psf.m &  &  & return the PSF \\ \hline 
			random\_intensity.m & intensity distribution & intensity & Draw a random intensity from the intensity distribution. \\ \hline 
			random\_position.m & Roi, N & x and y coordinate & Draw N random positions within ROI. \\ \hline 
			ReadRF.m & full filename, mode name, and frame number & RF data, parameters & Read in RF data. Written by A.Healey. \\ \hline 
			\textbf{RF2mat.m} & Full filename & .mat file & Convert raw US data to .mat file. \\ \hline 
			\textbf{run\_program.m} & File name raw US data, background frame index, count(true/false), bg\_subtraction(true/false),  background file name, correlation array file name. & .avi file and . Mat file & Perform all processing from raw US data to counted .avi file and .mat file. If count or background\_subtraction is false, the program will not perform these tasks. Background file name must be included if the background is supposed to be made from another file. Correlation array file name is the name of the previous video sequence. \\ \hline 
			\textbf{subtraction\_fun.m} & File name motion corrected .mat file, indexes for background & .mat file(subtracted data) & Create background file, and subtract the background from all frames. Save subtracted frames as .mat file \\ \hline 
			time\_array\_from\_ time\_stamps.m & time stamp 1, time stamp 2. & time array(seconds) & Make time array from two time stamps \\ \hline 
			VsiBModeIQ.m & Full file name, mode name, and frame number & IQ data and parameters & Compute IQ data from RF-data. Written by A. Needles, J. Mehi. Copyright VisualSonics 1999-2010 \\

		\end{longtable}
	\end{center}
%	\label{Matlab files}
%\end{table}
